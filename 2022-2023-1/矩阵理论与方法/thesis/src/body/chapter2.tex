\section{预备知识}

本章是正式开始研究两个大问题之前的准备工作, 主要介绍了欧氏空间、线性变
换、向量范数、矩阵范数以及矩阵函数, 这些都是之后研究问题的基石.

\subsection{欧式空间与线性变换}

本节研究欧氏空间及线性变换的性质, 特别是几种重要的线性变换、对若尔当标准形的求解和线性变换的求法.

\subsubsection{欧式空间与线性变换介绍}

\paragraph*{线性空间} \

\par 线性空间是线性代数中最基本的概念之一, 也是学习现代矩阵论的重要基础. 线性空间的概念, 是某类事物从量的方面的一个抽象. 我们不考虑集合的对象, 抽去它们的具体内容的含义, 来研究这类集合的公共性质, 并引进一个概括性名词. 于是就有如下的线性空间的概念.

\paragraph*{定义 2.1} 设 $V$ 是一个非空集合, 他的元素用 $\bm{x},\bm{y},\bm{z}$ 等表示, 并称之为向量; $K$ 是一个数域, 它的元素用 $k, l, m$ 等表示. 如果 $V$ 满足条件:
\par (1) 在$V$中定义一个加法运算, 即当 $\bm{x}, \bm{y} \in V$ 时, 有唯一的和 $\bm{x} + \bm{y} \in V$, 且加法运算满足以下性质:
\par 1) 结合律 $\bm{x} + (\bm{y} + \bm{x}) = (\bm{x} + \bm{y}) = \bm{z}$;
\par 2) 交换律 $\bm{x} + \bm{y} = \bm{y} + \bm{x}$;
\par 3) 存在零元素 $\bm{0}$, 使 $\bm{x} + \bm{0} = \bm{x}$;
\par 4) 存在负元素, 即对任一向量 $\bm{x}\in V$, 存在向量 $\bm{y} \in V$, 使 $\bm{x + y = 0}$, 则称$\bm{y}$ 为 $\bm{x}$
的负元素, 记为$\bm{-x}$, 于是有 $\bm{x + (- x) = 0}$.
\par (2) 在 $V$ 中定义数乘(数与向量的乘法)运算, 即当 $\bm{x} \in V, k \in K$ 时, 有唯一的乘
积 $k\bm{x} \in V$, 且数乘运算满足以下性质:
\par 5) 数因子分配律 $k(\bm{x + y} = k \bm{x} + k \bm{y})$;
\par 6) 分配律 $(k + l) \bm{x} = k \bm{x} + l \bm{x}$;
\par 7) 结合律 $k(l \bm{x}) = (kl) \bm{x}$;
\par 8) $1 \cdot \bm{x} = \bm{x}$ \\
则称 $V$ 为数域 $K$ 上的线性空间或向量空间.

\par $V$中所定义的加法及数乘运算统称为 $V$ 的线性运算. 在不致产生混淆时, 将数域 $K$ 上的
线性空间简称为线性空间. 数 $k$ 与向量 $\bm{x}$ 的乘积$k\bm{x}$也可写成$\bm{x}k$.

\par 需要指出, 不管$V$的元素如何, 当$K$为实数域$\mathbb{R}$时, 则称$V$为实线性空间; 当$K$为复
数域$\mathbb{C}$时, 就称$V$为复线性空间.

\paragraph*{例 2.1} 设$\mathbb{R^{+}}$为所有正实数组成的数集, 其加法与数乘运算分别定义为
$$
    m \oplus n = mn, \quad k \circ m = m^k
$$
证明$\mathbb{R^{+}}$是$\mathbb{R}$是上的线性空间. \par

\paragraph*{证} 设$m,n\in \mathbb{R^+}, \quad k \in R$,则有
$$
    m \oplus n = mn \in \mathbb{R^+}, \quad k \circ m = m^k \in \mathbb{R^+}
$$
即$\mathbb{R^+}$对所定义的加法运算“$\oplus$”与数乘运算“$\circ$”是封闭的,且有

\par (1) $(m \oplus n) \oplus p = (mn) \oplus p = mnp = m \oplus (np) = m \oplus (n \oplus p)$
\par (2) $m \oplus n = mn = nm = n \oplus m$
\par (3) 1是零元素,因为$m \oplus 1 = m \times 1 = m$
\par (4) $m$的负元素是$\dfrac{1}{m}$,因为$m \oplus \dfrac{1}{m} = m \oplus \dfrac{1}{m} = 1$
\par (5) $k \circ (m \oplus n) = k \circ (mn) = (mn)^k = m^kn^k = (k \circ m) \oplus (k \circ n)$
\par (6) $(k + l) \circ m = m^{k + l} = m^km^l = (k \circ m) \oplus (l \circ m)$
\par (7) $k \circ (l \circ m) = k \circ m^l = m^{lk} = m^{kl} = (kl) \circ m$
\par (8) $1 \circ m = m^1 = m$
\\ 成立, 故$\mathbb{R^+}$是实线性空间.

\paragraph*{定义 2.2} 设 $V$ 是数域 $K$ 上的线性空间, $\bm{x}_1,\bm{x}_2,\cdots,\bm{x}_r(r\geqslant 1)$ 是属于 $V$ 的任意 $r$ 个
向量, 如果它满足
\par (1) $\bm{x}_1,\bm{x}_2,\cdots,\bm{x}_r$ 线性无关;
\par (2) $V$ 中任一向量 $x$ 都是 $\bm{x}_1,\bm{x}_2,\cdots,\bm{x}_r$ 的线性组合.
\\ 则称 $\bm{x}_1,\bm{x}_2,\cdots,\bm{x}_r$ 为 $V$ 的一个基或基底, 并称 $\bm{x}_i(i=1,2,\cdots,r)$ 为基向量.

\par 且由上述定义可见, 线性空间的维数是其基中所含向量的个数.

\par 同时, 一个线性空间的基不是唯一的. 例如, $n$ 维向量组
\begin{gather}
    \begin{cases}
        \bm{e}_1 = (1, 0, \cdots, 0) \\
        \bm{e}_2 = (0, 1, \cdots, 0) \\
        \cdots                       \\
        \bm{e}_n = (0, 0, \cdots, 1)
    \end{cases}
    \begin{cases}
        \bm{y}_1 = (1, 1, \cdots, 1, 1) \\
        \bm{y}_2 = (0, 1, \cdots, 1, 1) \\
        \cdots                          \\
        \bm{y}_n = (0, 0, \cdots, 0, 1)
    \end{cases}
    \tag{2.1.1} \label{2.1.1}
\end{gather}
都实线性空间 $\bm{R}^n$ 的基. 这是因为
$$
    \begin{vmatrix}
        1      & 0      & \cdots & 0      \\
        0      & 1      & \cdots & 0      \\
        \vdots & \vdots &        & \vdots \\
        0      & 0      & \cdots & 1
    \end{vmatrix} \neq 0, \quad
    \begin{vmatrix}
        1      & 1      & \cdots & 1      & 1      \\
        0      & 1      & \cdots & 1      & 1      \\
        \vdots & \vdots &        & \cdots & \vdots \\
        0      & 0      & \cdots & 0      & 1
    \end{vmatrix} \neq 0, \quad
$$
从而它们各自都线性无关, 而且对于任一向量 $\bm{x} = (\xi_1, \xi_2, \cdots, \xi_n) \in R^n$, 分别有
\begin{gather*}
    \bm{x} = \xi_1\bm{e}_1 + \xi_2\bm{e}_2 + \cdots + \xi_n\bm{e}_n \\
    \bm{x} = \xi_1\bm{y}_1 + (\xi_2 - \xi_1)\bm{y}_2 + \cdots + (\xi_n - \xi_{n - 1})\bm{y}_n
\end{gather*}

\paragraph*{定义 2.3} 称线性空间 $V^n$ 的一个基 $\bm{x}_1,\bm{x}_2,\cdots,\bm{x}_r$ 为 $V^n$ 的一个坐标系. 设向量 $\bm{x} \in V^n$, 它在该基下的线性表示式为
$$
    \bm{x} = \xi_1 \bm{x}_1 + \xi_2 \bm{x}_2 + \cdots + \xi_n \bm{x}_n
$$
则称$\xi_1, \xi_2, \cdots, \xi_n$ 为 $\bm{x}$ 在该坐标系中的坐标或分量, 记为
$$
    (\xi_1, \xi_2, \cdots, \xi_n)^T
$$

\par 需要指出, 在不同的坐标系(或基)中, 同一向量的坐标一般是不同的.

\paragraph*{欧氏空间} \

\par 欧式空间是一种特殊的线性空间, 定义如下.

\paragraph*{定义 2.4} 设 $V$ 是实数域 $\mathbb{R}$ 上的线性空间, 对于 $V$ 中任意两个向量 $\bm{x}$ 与 $\bm{y}$, 按照某种规则定义一个实数, 用 $(\bm{x}, \bm{y})$ 来表示, 且它满足下述4个条件:
\par (1) 交换律: $(\bm{x}, \bm{y}) = (\bm{y}, \bm{x})$;
\par (2) 分配律: $(\bm{x}, \bm{y} + \bm{z}) = (\bm{x}, \bm{y}) + (\bm{x}, \bm{z})$;
\par (3) 齐次性: $(k\bm{x}, \bm{y}) = k(\bm{x}, \bm{y})(\forall k \in \mathbb{R})$;
\par (4) 非负性: $(\bm{x}, \bm{x}) \geqslant 0$, 当且仅当 $\bm{x} = \bm{0}$ 时, $(\bm{x}, \bm{x}) = 0$.
\\ 则称实数 $(\bm{x}, \bm{y})$ 为向量 $\bm{x}$ 与 $\bm{y}$ 的内积, 而称 $V$ 为Euclid空间, 简称欧氏空间或实内积空间.

\par 显然, 欧氏空间是定义了内积的实线性空间. 因此, 又有内积空间之称, 可见, 欧氏空间是一个特殊的实线性空间.

\par 因为向量的内积与向量的线性运算是彼此无关的运算, 所以不论内积如何规定, 都
不会影响该实线性空间的维数. 欧氏空间的子空间显然也是欧氏空间.

\paragraph[]{线性变换} \

\par 线性空间是某类客观事物从量的方面的一个抽象, 而线性变换则研究线性空间中元素之间的最基本联系.

\paragraph*{定义 2.5} 设 $V$ 是数域 $K$ 上的线性空间, $T$ 是 $V$ 到自身的一个映射, 使对任意向量
$\bm{x} \in V$, $V$ 中都有唯一的向量 $\bm{y}$ 与之对应, 则称 $T$ 是 $V$ 的一个变换或算子, 记为 $T\bm{x} = \bm{y}$,
称 $\bm{y}$ 为 $\bm{x}$ 在 $T$ 下的象, 而 $\bm{x}$ 是 $\bm{y}$ 的原象(或象源).

\par 例如, 平面上所有起点在原点的向量的集合, 形成实二维线性空间 $R^2$. 在 $R^2$ 中绕
原点的旋转就是 $R^2$ 的一个变换.

\paragraph*{定义 2.6} 如果数域 $K$ 上的线性空间 $V$ 的一个变换 $T$ 具有以下性质:
$$
    T(k\bm{x} + l\bm{y}) = k(T\bm{x}) + l(T\bm{y})
$$
其中 $\bm{x}, \bm{y} \in V, k, l \in K$. 则称 $T$ 为 $V$ 的一个线性变换或线性算子.

\paragraph*{例 2.2} 把线性空间 $R^2$ 的所有向量均绕原点依顺(或逆)时针方向旋转 $\theta$ 角的变换, 就是一个线性变换. 这时象 $(\eta_1, \eta_2)$ 与原象 $(\xi_1, \xi_2)$之间的关系为
$$
    \begin{bmatrix}
        \eta_1 \\
        \eta_2
    \end{bmatrix} = \begin{bmatrix}
        \cos\theta  & \sin\theta \\
        -\sin\theta & \cos\theta
    \end{bmatrix} \begin{bmatrix}
        \xi_1 \\
        \xi_2
    \end{bmatrix}
$$

\paragraph*{例 2.3} 在线性空间 $P_n$ 中, 求微分是其一个线性变换, 这里用 $D$ 表示, 即
$$
    Df(t) = f'(t) \quad (\forall f(t) \in P_n)
$$
\par 事实上, 对任意的 $f(t), g(t) \in P_n$ 及 $k, l \in \mathbb{R}$, 有
$$
    D(kf(t) + lg(t)) = (kf(t) + lg(t))' = kf'(t) + lg'(t) = k(Df(t)) + l(Dg(t))
$$

\par 不过需要注意, 线性变换可能把线性无关的向量组变为线性相关的向量组. 如零变
换 $T_0$ 就是这样.

\paragraph*{定义 2.6} 设 $T$ 是线性空间 $V$ 的线性变换, $V$ 中所有向量的象形成的集合, 称为 $T$ 的
值域, 用 $R(T)$ 表示, 即
$$
    R(T) = \{T\bm{x} \mid \bm{x} \in V \}
$$
$V$ 中所有被 $T$ 变为零向量的原象构成的集合, 称为 $T$ 的核, 用 $N(T)$ 表示, 即
$$
    N(T) = \{\bm{x} \mid T\bm{x} = \bm{0}, \bm{x} \in V \}
$$

\paragraph[]{线性变换的运算} \

\par 下面, 为了讨论线性变换的运算, 引入单位变换的零变换的概念.

\par 把线性空间 $V$ 的任一向量都变为其自身的变换是一个线性变换, 称为单位变换或
恒等变换, 记为 $T_e$, 于是有
$$
    T_e \bm{x} = \bm{0} \quad (\forall \bm{x} \in V)
$$

\par 如果 $T_1, T_2$ 是 $V$ 的两个变换, 且对于任意向量 $\bm{x} \in V$, 都有 $T_1\bm{x} = T_2\bm{x}$, 那么就称
$T_1$ 与 $T_2$ 相等, 记为
$$
    T_1 = T_2
$$

\par 对于线性空间的线性变换, 下面定义它们的集中运算方式.

\par{1.加法}

\par 设 $T_1, T_2$ 是线性空间 $V$ 的两个线性变换, 定义它们的和 $T_1 + T_2$ 为
$$
    (T_1 + T_2) \bm{x} = T_1 \bm{x} + T_2 \bm{x} \quad (\forall \bm{x} \in V)
$$
\par 下面证明, 线性变换 $T_1$ 与 $T_2$ 的和 $T_1 + T_2$ 仍是 $V$ 的线性变换. 事实上, 对任意
$\bm{x}, \bm{y} \in V, k, l \in K$, 由定义2.6有
\begin{align*}
    (T_1 + T_2)(k\bm{x} + l\bm{y}) = & T_1(k\bm{x} + l\bm{y}) + T_2(k\bm{x} + l\bm{y}) =           \\
                                     & k(T_1\bm{x}) + l(T_1\bm{y}) + k(T_2\bm{x}) + l(T_2\bm{y}) = \\
                                     & k(T_1\bm{x} + T_2\bm{x}) + l(T_1\bm{y} + T_2\bm{y}) =       \\
                                     & k(T_1 + T_2)\bm{x} + l(T_1 + T_2)\bm{y}
\end{align*}
这就表明 $T_1 + T_2$ 是 $V$ 的线性变换.

\par 另外, 定义线性变换 $T$ 的负变换 $-T$ 定义为
$$
    (-T)\bm{x} = -(T\bm{x}) \quad (\forall \bm{x} \in V)
$$

\par 因此, 线性变换的加法具有以下基本性质:
\par (1) $T_1 + T_2 = T_2 + T_1$;
\par (2) $(T_1 + T_2) + T_3 = T_1 + (T_2 + T_3)$;
\par (3) $T + T_0 = T$;
\par (4) $T + (-T) = T_0$.

\par{2. 线性变换与数的乘法}

\par 设 $k \in K$, $T$ 为线性空间 $V$ 中的线性变换, 定义数 $k$ 与 $T$ 的乘积 $kT$ 为
$$
    (kT)\bm{x} = k(T\bm{x}) \quad (\forall \bm{x} \in V)
$$

\par 因此, 线性变换的数乘具有以下基本性质:
\par (1) $k(T_1 + T_2) = kT_1 + k$;
\par (2) $(k + l)T = kT + lT$;
\par (3) $(kl)T= k(lT)$;
\par (4) $1\ T = T$.

\par{3. 线性变换的乘法}

\par 设 $T_1, T_2$ 是线性空间 $V$ 的两个线性变换, 定义 $T_1$ 与 $T_2$ 的乘积 $T_1T_2$ 为
$$
    (T_1T_2)\bm{x} = T_1(T_2)\bm{x} \quad (\forall \bm{x} \in V)
$$
即 $T_1T_2$ 是先施行 $T_2$, 然后施行 $T_1$ 的变换. 且
\begin{align*}
    (T_1T_2)(k\bm{x} + l\bm{y}) = & T_1(T_2(k\bm{x} + l\bm{y})) =           \\
                                  & k(T_1(T_2\bm{x})) + l(T_1(T_2\bm{y})) = \\
                                  & k(T_1T_2)\bm{x} + l(T_1T_2)\bm{y}
\end{align*}

\par 因此, 线性变换的乘积具有以下基本性质:
\par (1) $(T_1T_2)T_3 = T_1(T_2T_3)$;
\par (2) $T_eT = TT_e = T$;
\par (3) 一般地, $T_1T_2 \neq T_2T_1$;
\par (4) $T_1(T_2 + T_3) = T_1T_2 + T_1T_3$;
\par (5) $(T_1 + T_2)T_3 = T_1T_3 + T_2T_3$.

\par{4. 逆变换}

\par 同逆矩阵的概念类似, 若 $T$ 是 $V$ 的线性变换, 且存在线性变换 $S$, 使得
$$
    (ST)\bm{x} = (TS)\bm{x} = \bm{x} \quad (\forall \bm{x} \in V)
$$
则称 $S$ 是 $T$ 的逆变换, 记为 $S = T^{-1}$. 且有
$$
    T^{-1}T = TT^{-1} = T_e
$$

\par{5. 线性变换的多项式}

\par 设 $n$ 是正整数, $T$ 是线性空间 $V$ 的线性变换. 定义 $T$ 的 $n$ 次幂为
$$
    T^n = T^{n-1}T \quad (n = 2,3,\cdots)
$$
定义 $T$ 的零次幂为
$$
    T^0 = T_e
$$

\par 设 $f(T) = a_0t^m + a_1t^{m-1} + \cdots + a_{m-1}t + a_m$ 是纯量 $t$ 的 $m$ 次多项式, $T$ 是 $V$ 的一个线性变换, 则由线性变换的运算可知
$$
    f(T) = a_0T^m + a_1T^{m-1} + \cdots + a_{m-1}T + a_mT_e
$$
也是 $V$ 的一个线性变换或线性算子, 称其为线性变换 $T$ 的多项式.

\paragraph[]{线性变换的矩阵} \

\par 有限维线性空间的向量可以用坐标表示, 更进一步, 这里将通过坐标把线性变换用
矩阵表示出来, 从而可以把比较抽象的线性变换转化为具体的矩阵来处理.

\paragraph*{定义 2.7} 设 $x_1, x_2, \cdots, x_n$ 为数域 $P$ 上线性空间 $V$ 的一组基, $T$ 为 $V$ 的线性变换. 基
向量的象可以被基线性表出, 设
$$
    \begin{cases}
        T(x_1) = a_{11}\bm{x}_1 + a_{21}\bm{x}_2 + \cdots + a_{n1}\bm{x}_n \\
        T(x_1) = a_{11}\bm{x}_1 + a_{21}\bm{x}_2 + \cdots + a_{n1}\bm{x}_n \\
        \cdots                                                             \\
        T(x_1) = a_{11}\bm{x}_1 + a_{21}\bm{x}_2 + \cdots + a_{n1}\bm{x}_n \\
    \end{cases}
$$
用矩阵表示即为
$$
    T(\bm{x}_1, \bm{x}_2, \cdots, \bm{x}_n) = (T\bm{x}_1, T\bm{x}_2, \cdots, T\bm{x}_n) = (\bm{x}_1, \bm{x}_2, \cdots, \bm{x}_n)\bm{A}
$$
其中, $\bm{A} = \begin{bmatrix}
        a_{11} & a_{12} & \cdots & a_{1n} \\
        a_{21} & a_{22} & \cdots & a_{2n} \\
        \cdots & \cdots & \cdots & \cdots \\
        a_{n1} & a_{n2} & \cdots & a_{nn}
    \end{bmatrix}$, 矩阵 $\bm{A}$ 称为线性变换 $T$ 在基 $\bm{x}_1, \bm{x}_2, \cdots, \bm{x}_n$ 下的矩
阵.

\par 需要注意
\par (1) 给定 $V^n$ 的基 $\bm{x}_1, \bm{x}_2, \cdots, \bm{x}_n$ 和线性变换 $T$, 矩阵 $A$ 是唯一的;
\par (2) 单位变换在任意一组基下的矩阵皆为单位矩阵; 零变换在任意一组基下的
矩阵皆为零矩阵; 数乘变换在任意一组基下的矩阵皆为数量矩阵.

\paragraph*{例 2.4} 在矩阵空间 $R^{2\times 2}$ 中, 给定矩阵
$$
    \bm{B} = \begin{bmatrix}
        0 & 1 \\
        4 & 0
    \end{bmatrix}
$$
线性变换 $T(\bm{X}) = \bm{XB}(\forall \bm{X} \in R^{2\times 2})$, $R^{2\times 2}$ 的两个基为
\par (1): $\bm{E}_{11}, \bm{E}_{12}, \bm{E}_{21}, \bm{E}_{22}$;
\par (2): $\bm{B}_1 = \begin{bmatrix}
        1 & 1 \\
        1 & 1
    \end{bmatrix}, \bm{B}_2 = \begin{bmatrix}
        1 & 1 \\
        1 & 0
    \end{bmatrix}, \bm{B}_3 = \begin{bmatrix}
        1 & 1 \\
        0 & 0
    \end{bmatrix}, \bm{B}_4 = \begin{bmatrix}
        1 & 0 \\
        0 & 0
    \end{bmatrix}$.
\\分别求 $T$ 在这两个基下的矩阵.

\paragraph*{解} 计算(1)的基象组, 有
$$
    T(\bm{E}_{11}) = \bm{E}_{11}\bm{B} = \begin{bmatrix}
        0 & 1 \\
        0 & 0
    \end{bmatrix} = 0 \bm{E}_{11} + 1 \bm{E}_{12} + 0 \bm{E}_{21} + 0 \bm{E}_{22} \\
$$
$$
    T(\bm{E}_{12}) = \bm{E}_{12}\bm{B} = \begin{bmatrix}
        4 & 0 \\
        0 & 0
    \end{bmatrix} = 4 \bm{E}_{11} + 0 \bm{E}_{12} + 0 \bm{E}_{21} + 0 \bm{E}_{22} \\
$$
$$
    T(\bm{E}_{21}) = \bm{E}_{21}\bm{B} = \begin{bmatrix}
        0 & 0 \\
        0 & 1
    \end{bmatrix} = 0 \bm{E}_{11} + 0 \bm{E}_{12} + 0 \bm{E}_{21} + 1 \bm{E}_{22} \\
$$
$$
    T(\bm{E}_{22}) = \bm{E}_{22}\bm{B} = \begin{bmatrix}
        0 & 0 \\
        4 & 0
    \end{bmatrix} = 0 \bm{E}_{11} + 0 \bm{E}_{12} + 4 \bm{E}_{21} + 0 \bm{E}_{22}
$$
故 $T$ 在基(1)下的矩阵为
$$
    \bm{A}_1 = \begin{bmatrix}
        0 & 4 & 0 & 0 \\
        1 & 0 & 0 & 0 \\
        0 & 0 & 0 & 4 \\
        0 & 0 & 1 & 0
    \end{bmatrix}
$$
计算(2)的基象组, 有
$$
    T(\bm{B}_{1}) = \bm{B}_{1}\bm{B} = \begin{bmatrix}
        4 & 1 \\
        4 & 1
    \end{bmatrix} = 1 \bm{B}_{1} + 3 \bm{B}_{2} - 3 \bm{B}_{3} + 3 \bm{B}_{4} \\
$$
$$
    T(\bm{B}_{2}) = \bm{B}_{2}\bm{B} = \begin{bmatrix}
        4 & 1 \\
        0 & 1
    \end{bmatrix} = 1 \bm{B}_{1} - 1 \bm{B}_{2} + 1 \bm{B}_{3} + 3 \bm{B}_{4} \\
$$
$$
    T(\bm{B}_{3}) = \bm{B}_{3}\bm{B} = \begin{bmatrix}
        4 & 1 \\
        0 & 0
    \end{bmatrix} = 0 \bm{B}_{1} + 0 \bm{B}_{2} + 1 \bm{B}_{3} + 3 \bm{B}_{4} \\
$$
$$
    T(\bm{B}_{4}) = \bm{B}_{4}\bm{B} = \begin{bmatrix}
        0 & 1 \\
        0 & 0
    \end{bmatrix} = 0 \bm{B}_{1} + 0 \bm{B}_{2} + 1 \bm{B}_{3} - 1 \bm{B}_{4}
$$
故 $T$ 在基(2)下的矩阵为
$$
    \bm{A}_2 = \begin{bmatrix}
        1  & 1  & 0 & 0  \\
        3  & -1 & 0 & 0  \\
        -3 & 1  & 1 & 1  \\
        3  & 3  & 3 & -1
    \end{bmatrix}
$$
显然 $\bm{A}_1 \neq \bm{A}_2$.

\paragraph*{定理 2.1} 设$\bm{x}_1,\bm{x}_2,\cdots,\bm{x}_n$是数域$K$上的线性空间$V^n$的一个基, 线性变换$T_1, T_2$在该基下的矩阵
依次是$\bm{A}, \bm{B}$.则有如下结论:
\par(1) $(T_1 + T_2)(\bm{x}_1,\bm{x}_2,\cdots,\bm{x}_n) = (\bm{x}_1,\bm{x}_2,\cdots,\bm{x}_n)(\bm{A} + \bm{B})$;
\par(2) $(kT_1)(\bm{x}_1,\bm{x}_2,\cdots,\bm{x}_n) = (\bm{x}_1,\bm{x}_2,\cdots,\bm{x}_n)(k\bm{A})$;
\par(3) $(T_1T_2)(\bm{x}_1,\bm{x}_2,\cdots,\bm{x}_n) = (\bm{x}_1,\bm{x}_2,\cdots,\bm{x}_n)\bm{AB}$;
\par(4) $T_1^{-1}(\bm{x}_1,\bm{x}_2,\cdots,\bm{x}_n) = (\bm{x}_1,\bm{x}_2,\cdots,\bm{x}_n) \bm{A}^{-1}$.

\paragraph*{定理 2.2} 设线性变换 $T$ 在线性空间 $V^n$ 的基 $\bm{x}_1, \bm{x}_2, \cdots, \bm{x}_n$ 下的矩阵是 $\bm{A}$, 向量 $\bm{x}$ 在
该基下的坐标是 $\bm{\alpha}$, 则 $T\bm{x}$ 在该基下的坐标是
\begin{gather}
    \bm{\beta} = \bm{A\alpha}
    \tag{2.1.2}
\end{gather}

\par 利用(2.1.2)式, 可以直接由线性变换的矩阵 $\bm{A}$, 来计算一个向量 $\bm{x}$ 的象 $T\bm{x}$ 的坐
标.

\par 为了利用矩阵研究线性变换, 有必要弄清楚线性变换的矩阵是怎么样随基的改变而改变的, 从而建立矩阵相似的概念.

\paragraph*{定理 2.3} 设线性空间 $V^n$ 的线性变换为 $T$, $T$ 在 $V^n$ 的两个基 $\bm{x}_1, \bm{x}_2, \cdots, \bm{x}_n$ 和 $\bm{y}_1$,
$\bm{y}_2, \cdots, \bm{y}_n$ 下的矩阵依次是 $\bm{A}$ 和 $\bm{B}$, 并且
$$
    (\bm{y}_1, \bm{y}_2, \cdots, \bm{y}_n) = (\bm{x}_1, \bm{x}_2, \cdots, \bm{x}_n) \bm{C}
$$
那么
\begin{equation}
    \bm{B} = \bm{C}^{-1}\bm{AC}
    \tag{2.1.3}
\end{equation}

\par 式(2.1.3)给出的两个矩阵 $\bm{A}$ 和 $\bm{B}$ 之间的关系, 在矩阵论中将起极其重要的作用. 引入如下定义.

\paragraph*{定义 2.8} 设 $\bm{A}, \bm{B}$ 为数域 $K$ 上的两个 $n$ 阶矩阵, 如果存在 $K$ 上的 $n$ 阶可逆矩阵 $\bm{P}$,
使得 $\bm{B} = \bm{P}^{-1}\bm{AP}$, 则称 $\bm{A}$ 相似于 $\bm{B}$, 记为 $\bm{A} \sim \bm{B}$.

\par 那么, 相似矩阵有如下基本性质:
\par 反身性: $\bm{A} \sim \bm{A}$.
\par 对称性: 如果 $\bm{A} \sim \bm{B}$, 那么 $\bm{B} \sim \bm{A}$.
\par 传递性: 如果 $\bm{A} \sim \bm{B}, \bm{B} \sim \bm{C}$, 那么 $\bm{A} \sim \bm{C}$.

\paragraph[]{特征值与特征向量} \

\par 现在讨论如何选择线性空间的基, 使线性变换的该基下的矩阵形状最简单的问题.
为此, 先论述线性变换的特征值和特征向量的概念. 它们对于线性变换的研究, 起着
十分重要的作用.

\paragraph*{定义 2.9} 设 $T$ 是数域 $K$ 上的线性空间 $V^n$ 的线性变换, 且对 $K$ 中某一数 $\lambda_0$, 存在非零
向量 $\bm{x} \in V^n$, 使得
\begin{gather}
    T\bm{x} = \lambda_0 \bm{x}
    \tag{2.1.4}
\end{gather}
成立, 则称 $\lambda_0$ 为 $T$ 的特征值, $\bm{x}$ 为 $T$ 的属于 $\lambda_0$ 的特征向量.

\par 式(2.1.4)表明, 在几何上, 特征向量 $\bm{x}$ 的方位, 经过线性变换后保持不变.

\paragraph*{定义 2.10} 设 $\bm{A} = (a_{ij})_{n\times n}$ 是数域 $K$ 上的 $n$ 阶矩阵, $\lambda$ 是参数, $\bm{A}$ 的特征矩阵 $\lambda\bm{I} - \bm{A}$ 的
行列式
$$
    \det (\lambda\bm{I} - \bm{A}) = \begin{vmatrix}
        \lambda - a_{11} & -a_{12}          & \cdots & -a_{1n}          \\
        -a_{21}          & \lambda - a_{22} & \cdots & -a_{1n}          \\
        \vdots           & \vdots           &        & \vdots           \\
        -a_{n1}          & -a_{n2}          & \cdots & \lambda - a_{nn}
    \end{vmatrix}
$$
称为矩阵 $\bm{A}$ 的特征多项式, 它是 $K$ 上的一个 $n$ 次多项式, 记为 $\varphi(\lambda)$. $\varphi(\lambda)$ 的根(或零点)
$\lambda_0$ 称为 $\bm{A}$ 的特征值(根); 而相对应与方程组
\begin{gather}
    (\lambda_0\bm{I} - \bm{A})\begin{bmatrix}
        \xi_1  \\
        \xi_2  \\
        \vdots \\
        \xi_n
    \end{bmatrix} = \bm{0}
    \tag{2.1.5}
\end{gather}
的非零解向量 $(\xi_1, \xi_2, \cdots, \xi_n)^T$ 称为 $\bm{A}$ 的属于特征值 $\lambda_0$ 的特征向量.

\par 那么, 计算特征值和特征向量的步骤如下:
\par (1) 取定数域 $K$ 上的线性空间 $V^n$ 的一个基, 写出线性变换 $T$ 在该基下的矩阵 $\bm{A}$;
\par (2) 求出 $\bm{A}$ 的特征多项式 $\varphi(\lambda)$ 在数域 $K$ 上的全部根, 他们就是 $T$ 的全部特征值;
\par (3) 把求得的特征值逐个带入方程组(2.1.5)中, 解出矩阵 $\bm{A}$ 属于每个特征值的全部线性无关的特征向量;
\par (4) 以 $\bm{A}$ 的属于每个特征值的特征向量为 $V^n$ 中取定基下的坐标, 即得 $T$ 的相应特征向量.

\par 由行列式的展开法则可得 $n$ 阶矩阵 $\bm{A} = (a_{ij})_{n\times n}$ 的特征多项式
\begin{align*}
    \varphi(\lambda) = & \det(\lambda\bm{I} - \bm{A}) =                                                           \\
                       & \lambda^n - (a_{11} + a_{22} + \cdots + a_{nn})\lambda^{n-1} + \cdots + (-1)^n\det\bm{A}
\end{align*}
如果 $\bm{A}$ 有 $n$ 个特征值 $\lambda_1, \lambda_2, \cdots, \lambda_n$, 则由上式可知
\begin{gather}
    \sum\limits_{i=1}^n \lambda_i = \sum\limits_{i=1}^n a_{ii}, \quad \lambda_1\lambda_2\cdots\lambda_n = \det\bm{A}
    \tag{2.1.6}
\end{gather}
引入记号
$$
    \mathrm{tr}\bm{A} = \sum\limits_{i=1}^n a_{ii}
$$
称为矩阵 $\bm{A}$ 的迹或追迹. 式(2.1.6)表明, 矩阵 $\bm{A}$ 的所有特征值的和等于 $\bm{A}$ 的迹, 而 $\bm{A}$
的全体特征值的乘积等于 $\det\bm{A}$.

\paragraph*{定理 2.4(Hamilton-Cayley)} $n$ 阶矩阵 $\bm{A}$ 是其特征多项式的矩阵根(零点), 即令
$$
    \varphi(\lambda) = \det(\lambda\bm{I} - \bm{A}) = \lambda^n + a_1\lambda^{n-1} + \cdots + a_{n-1}\lambda + a_n
$$
则有
\begin{gather}
    \varphi(\bm{A}) = \bm{A}^n + a_1\bm{A}^{n-1} + \cdots + a_{n-1}\bm{A} + a_n\bm{I} + \bm{O}
    \tag{2.1.7}
\end{gather}

\par 当 $n$ 阶矩阵 $\bm{A}$ 可逆时, 它的特征多项式中的常数项 $a_n \neq 0$. 由(2.1.7)可得
$$
    \bm{A}^{-1} = -\dfrac{1}{a_n}(\bm{A}^{n-1} + a_1\bm{A}^{n-1} + \cdots + a_{n-2}\bm{A} + a_{n-1}\bm{I})
$$
即 $\bm{A}$ 的逆矩阵能够由它的 $n - 1$ 次矩阵多项式表示. 此外, 无论 $\bm{A}$ 是否可逆, 它的 $n$ 次
幂也能够由它的次数不超过 $n - 1$ 的矩阵多项式表示. 根据后一结论, 可以简化矩阵多
项式的计算问题.

\paragraph*{例 2.5} 计算矩阵多项式 $\bm{A}^{100} + 2\bm{A}^{50}$, 其中
$$
    \bm{A} = \begin{bmatrix}
        1 & 1  & -1 \\
        1 & 1  & 1  \\
        0 & -1 & 2
    \end{bmatrix}
$$

\paragraph*{解} 令 $\psi(\lambda) = \lambda^{100} + 2\lambda^{50}$, 可求得 $\bm{A}$ 的特征多项式为
$$
    \varphi(\lambda) = \det(\lambda\bm{I} - \bm{A}) = (\lambda - 1)^2(\lambda - 2)
$$
用 $\varphi(\lambda)$ 除 $\phi(\lambda)$, 可得
$$
    \psi(\lambda) = \varphi(\lambda)q(\lambda) + b_0 + b_1\lambda + b_2\lambda^2
$$
将 $\lambda = 1,2$ 分别带入上式, 则有
$$
    \begin{cases}
        b_0 + b_1 + b_2 = 3
        b_0 + 2b_1 + 4b_2 = 2^{100} + 2^{51}
    \end{cases}
$$
对 $\psi(\lambda)$ 关于 $\lambda$ 求导, 得到
$$
    \psi'(\lambda) = [2(\lambda - 1)(\lambda - 2) + (\lambda - 1)^2]q(\lambda) + \varphi(\lambda)q'(\lambda) + b_1 + 2b_2\lambda
$$
将 $\lambda = 1$ 带入上式, 可得
$$
    b_1 + 2b_2 = 200
$$
从而求得
$$
    b_0 = 2^{100} + 2^[51] - 400, \quad b_1 = 606 - 2^{101} - 2^{52}, \quad b_2 = -203 + 2^{100} + 2^{51}
$$
故
$$
    \bm{A}^{100} + 2\bm{A}^{50} = \psi(\bm{A}) = b_0\bm{I} + b_1\bm{A} + b_2\bm{A}^2
$$

\par 以矩阵 $\bm{A}$ 为根的多项式有时是很多的, 但是它们之间却有一定的关系. 引入如下定义.

\paragraph*{定义 2.11} 首项系数是1(简称首1), 次数最小, 且以矩阵 $\bm{A}$ 为根的 $\lambda$ 的多项式, 称为 $\bm{A}$
的最小多项式, 常用 $m(\lambda)$ 表示.

\par 根据定理2.4, 显然 $\bm{A}$ 的最小多项式 $m(\lambda)$ 的次数不大于它的特征多项式 $\varphi(\lambda)$ 的次数.

\paragraph*{例 2.6} 求矩阵 $\bm{A} = \begin{bmatrix}
        3  & -3 & 2  \\
        -1 & 5  & -2 \\
        -1 & 3  & 0
    \end{bmatrix}$ 的最小多项式.

\paragraph*{解} 设 $f(\lambda) = \lambda + k (k \in \mathbb{R})$, 由于 $f(\bm{A}) = \bm{A} + k\bm{I} \neq \bm{O}$, 所以任何一次多项式都不是 $\bm{A}$ 的最小多项式. 注意到 $\bm{A}$ 的特征多项式
$$
    \varphi(\lambda) = (\lambda - 2)^2(\lambda - 4)
$$
且对于它的二次因式
$$
    \psi(\lambda) = (\lambda - 2)(\lambda - 4) = \lambda^2 - 6\lambda + 8
$$
有
$$
    \psi(\bm{A}) = \bm{A}^2 - 6\bm{A} + 8\bm{I} + \bm{O}
$$
于是由定义2.11, 有 $m(\lambda) = \psi(\lambda)$.

\par 这就是说, $\bm{A}$ 的最小多项式是其特征多项式的因式. 这个事实具有一般性, 引出如下定理。

\paragraph*{定理 2.5} 矩阵 $\bm{A}$ 的最小多项式 $m(\lambda)$ 可以整除以 $\bm{A}$ 为根的任意首1多项式 $\psi(\lambda)$, 且
$m(\lambda)$ 是唯一的.

\paragraph*{证} 假若 $m(\lambda)$ 不能整除 $\psi(\lambda)$, 则有
$$
    \psi(\lambda) = m(\lambda)q(\lambda) + r(\lambda)
$$
其中 $r(\lambda)$ 的次数小于 $m(\lambda)$ 的次数. 于是由
$$
    \psi(\bm{A}) = m(\bm{A})q(\bm{A}) + r(\bm{A})
$$
知 $r(\bm{A}) = \bm{O}$, 这就与 $m(\lambda)$ 是 $\bm{A}$ 的最小多项式相矛盾.

\paragraph*{定理 2.6} 矩阵 $\bm{A}$ 的最小多项式 $m(\lambda)$ 与其特征多项式 $\varphi(\lambda)$ 的零点相同(不计重数).

\paragraph*{证} 由定理2.4知 $\varphi(\bm{A}) = \bm{O}$, 再由定理2.5知 $m(\lambda)$ 能够整除 $\varphi(\lambda)$, 所以 $m(\lambda)$ 的零点是
$\varphi(\lambda)$ 的零点.

\paragraph*{定理 2.7} 设 $n$ 阶矩阵 $\bm{A}$ 的特征多项式为 $\varphi(\lambda)$, 特征矩阵 $\lambda\bm{I} - \bm{A}$ 的全体 $n - 1$ 阶子式的
最大公因式为 $d(\lambda)$, 则 $\bm{A}$ 的最小多项式为
$$
    m(\lambda) = \dfrac{\varphi(\lambda)}{d(\lambda)}
$$

\par 需要指出, 最小多项式相同是矩阵相似的必要条件, 并非充分条件.

\paragraph[]{对角矩阵} \

\par 对角矩阵是较简单的矩阵之一. 无论是计算它的乘积、逆矩阵还是特征值等, 都甚
为方便. 下面讨论哪些线性变换在适当的基下的矩阵是对角矩阵的问题.

\paragraph*{定理 2.8} 设 $T$ 是线性空间 $V^n$ 的线性变换, $T$ 在某一基下的矩阵 $\bm{A}$ 可以为对角矩阵的
充要条件是 $T$ 有 $n$ 个线性无关的特征向量.

\paragraph*{证} 设 $T$ 在 $V^n$ 的基 $\bm{x}_1, \bm{x}_2, \cdots, \bm{x}_n$ 下的矩阵是对角矩阵
$$
    \bm{A} = \mathrm{diag}(\lambda_1, \lambda_2, \cdots, \lambda_n)
$$
这就意味着有
$$
    T\bm{x}_i = \lambda_i\bm{x}_i \quad (i = 1, 2, \cdots, n)
$$
因而 $\bm{x}_1, \bm{x}_2, \cdots, \bm{x}_n$ 就是 $T$ 的 $n$ 个线性无关的特征向量.
\par 反之, 如果 $T$ 有 $n$ 个线性无关的特征向量 $\bm{x}_1, \bm{x}_2, \cdots, \bm{x}_n$, 即有
$$
    T\bm{x}_i = \lambda_i\bm{x}_i \quad (i = 1, 2, \cdots, n)
$$
那么就取 $\bm{x}_1, \bm{x}_2, \cdots, \bm{x}_n$ 为 $V^n$ 的基, 于是在这个基下 $T$ 的矩阵是对角矩阵.

\paragraph*{定理 2.9} $n$ 阶矩阵 $\bm{A}$ 与对角矩阵相似的充要条件是, $\bm{A}$ 有 $n$ 个线性无关的特征向量,
或 $\bm{A}$ 有完备的特征向量系.

\subsubsection{若尔当标准型的求解}

\par 前面指出, 一切 $n$ 阶矩阵 $\bm{A}$ 可以分解成许多相似类. 现在需要在与 $\bm{A}$ 相似的全体
矩阵中, 找出一个较简单的矩阵来作为这个相似类的标准形. 当然以对角矩阵作为标准
形最好, 可惜不是每一个矩阵都能与对角矩阵相似.

\paragraph*{定理 2.10} 设 $T$ 是复数域 $\mathbb{C}$ 上的线性空间 $V^n$ 的线性变换, 任取 $V^n$ 的一个基, $T$ 在该
基下的矩阵是 $\bm{A}$, $T$(或$\bm{A}$)的特征多项式可分解因式为
\begin{gather}
    \varphi(\lambda) = (\lambda - \lambda_1)^{m_1}(\lambda - \lambda_2)^{m_2}\cdots(\lambda - \lambda_s)^{m_s} \quad (m_1 + m_2 + \cdots + m_n = n)
    \tag{2.1.8}
\end{gather}
则 $V^n$ 可分解成不变子空间的直和
$$
    V^n = V_1 \oplus V_2 \oplus \cdots \oplus V_s
$$
其中 $V_i = \{\bm{x} \mid (T - \lambda_iT_e)^{m_i}\bm{x} = \bm{0}, \bm{x} \in V^n\}$ 是线性变换 $(T - \lambda_iT_e)^{m_i}$ 的核子空间.

\par 如果给每个子空间 $V_i$ 选一适当的基, 每个子空间的基合并起来即为 $V^n$ 的基, 且 $T$ 在该基下的矩阵为以下形式的准对角矩阵
\begin{gather}
    \bm{J} = \begin{bmatrix}
        \bm{J}_1(\lambda_1) &                     &        &                     \\
                            & \bm{J}_2(\lambda_2) &        &                     \\
                            &                     & \ddots &                     \\
                            &                     &        & \bm{J}_s(\lambda_s)
    \end{bmatrix}
    \tag{2.1.9}
\end{gather}
其中
$$
    \bm{J}_i(\lambda_i) = \begin{bmatrix}
        \lambda_i & 1         &           &        &           \\
                  & \lambda_i & 1         &        &           \\
                  &           & \lambda_i & \ddots &           \\
                  &           &           & \ddots & 1         \\
                  &           &           &        & \lambda_i
    \end{bmatrix}_{m_i\times m_i} \quad (i = 1, 2, \cdots, s)
$$

\paragraph*{定义 2.12} 由式(2.1.9)给出的矩阵 $\bm{J}$ 称为矩阵 $\bm{A}$ 的 Jordan 标准形, $\bm{J}_i(\lambda_i)$ 称为因式
$(\lambda - \lambda_i)^{m_i}$ 对应的 Jordan 块.

\par 由相似矩阵的的定义可得到如下定理.

\paragraph*{定理 2.11} 设 $\bm{A}$ 是 $n$ 阶复矩阵, 且其特征多项式的某种分解式是(2.1.8), 则存在 $n$ 阶复可逆矩阵 $\bm{P}$, 使
$$
    \bm{P}^{-1}\bm{AP} = \bm{J}
$$
即, 设 $T$ 是复数域 $\mathbb{C}$ 上线性空间 $V^n$ 的线性变换, 在 $V^n$ 中必存在一个基, 使 $T$ 在该基
下的矩阵是 Jordan 标准形 $\bm{J}$.

\par 虽然上述定理已经肯定了一般矩阵的 Jordan 标准形是存在的, 但是仍旧无法准确
地求出矩阵的 Jordan 标准形. 而讨论矩阵的 Jordan 标准形的求法, 涉及以下形式的多
项式矩阵或 $\lambda -$ 矩阵
$$
    \bm{A}(\lambda) = \begin{bmatrix}
        a_{11}(\lambda) & a_{12}(\lambda) & \cdots & a_{1n}(\lambda) \\
        a_{21}(\lambda) & a_{22}(\lambda) & \cdots & a_{2n}(\lambda) \\
        \vdots          & \vdots          &        & \vdots          \\
        a_{n1}(\lambda) & a_{n2}(\lambda) & \cdots & a_{nn}(\lambda)
    \end{bmatrix}
$$
的理论, 其中 $a{ij}(\lambda)(i, j = 1, 2, \cdots, n)$ 为数域 $K$ 上的纯量 $\lambda$ 的多项式. 如果 $\bm{A} = (a_{ij})_{n\times n}$
是数域 $K$ 上的 $n$ 阶矩阵, 则 $\bm{A}$ 的特征矩阵
$$
    \lambda\bm{I} - \bm{A} = \begin{bmatrix}
        \lambda - a_{11} & -a_{12}          & \cdots & -a_{1n}          \\
        -a_{21}          & \lambda - a_{22} & \cdots & -a_{2n}          \\
        \vdots           & \vdots           &        & \vdots           \\
        -a_{n1}          & -a_{n2}          & \cdots & \lambda - a_{nn}
    \end{bmatrix}
$$
就是一个特殊的多项式矩阵.
\par 多项式矩阵 $\bm{A}(\lambda)$ 的标准形, 是指使用矩阵的初等变换将 $\bm{A}(\lambda)$ 化为多项式矩阵, 有
$$
    \bm{A}(\lambda) \to \begin{bmatrix}
        d_1(\lambda) &              &        &              &   &        &   \\
                     & d_2(\lambda) &        &              &   &        &   \\
                     &              & \ddots &              &   &        &   \\
                     &              &        & d_s(\lambda) &   &        &   \\
                     &              &        &              & 0 &        &   \\
                     &              &        &              &   & \ddots &   \\
                     &              &        &              &   &        & 0
    \end{bmatrix}
$$
其中
$$
    d_1(\lambda) \mid d_2(\lambda), d_2(\lambda) \mid d_3(\lambda), \cdots d_{s-1}(\lambda) \mid d_s(\lambda) \quad (s \geqslant n)
$$
且 $d_i(\lambda)(i = 1, 2, \cdots, s)$ 是首1多项式.

\paragraph*{例 2.7} 用初等变化化多项式矩阵
$$
    \bm{A}(\lambda) = \begin{bmatrix}
        -\lambda + 1  & 2\lambda - 1            & \lambda    \\
        \lambda       & \lambda^2               & -\lambda   \\
        \lambda^2 + 1 & \lambda^2 + \lambda - 1 & -\lambda^2
    \end{bmatrix}
$$
为标准形.

\paragraph*{解}

\begin{align*}
    \bm{A}(\lambda) & \to \begin{bmatrix}
                              -\lambda + 1  & 2\lambda - 1            & 1 \\
                              \lambda       & \lambda^2               & 0 \\
                              \lambda^2 + 1 & \lambda^2 + \lambda - 1 & 1
                          \end{bmatrix} \to \\
                    & \begin{bmatrix}
                          1 & 2\lambda - 1            & -\lambda + 1  \\
                          0 & \lambda^2               & \lambda       \\
                          1 & \lambda^2 + \lambda - 1 & \lambda^2 + 1
                      \end{bmatrix} \to
\end{align*}
\begin{align*}
     & \begin{bmatrix}
           1 & 2\lambda - 1        & -\lambda + 1        \\
           0 & \lambda^2           & \lambda             \\
           0 & \lambda^2 - \lambda & \lambda^2 + \lambda
       \end{bmatrix} \to                   \\
     & \begin{bmatrix}
           1 & 0                   & 0                   \\
           0 & \lambda^2           & \lambda             \\
           0 & \lambda^2 - \lambda & \lambda^2 + \lambda
       \end{bmatrix} \to \begin{bmatrix}
                             1 & 0                   & 0                   \\
                             0 & \lambda             & \lambda^2           \\
                             0 & \lambda^2 + \lambda & \lambda^2 - \lambda
                         \end{bmatrix} \to \\
     & \begin{bmatrix}
           1 & 0                   & 0                    \\
           0 & \lambda             & 0                    \\
           0 & \lambda^2 + \lambda & -\lambda^3 - \lambda
       \end{bmatrix} \to \begin{bmatrix}
                             1 & 0       & 0                   \\
                             0 & \lambda & 0                   \\
                             0 & 0       & \lambda^3 + \lambda
                         \end{bmatrix}
\end{align*}
最后所得的矩阵是 $\bm{A}(\lambda)$ 的标准形, 此时 $d_1(\lambda) = 1, d_2(\lambda) = \lambda, d_3(\lambda) = \lambda^3 + \lambda$.

\par 可以证明, 一个多项式矩阵 $\bm{A}(\lambda)$ 的标准形式的对角线上的非零元素$d_i(\lambda)$ 不随矩阵
的初等变换而改变. 因此, 通常称 $d_i(\lambda)(i = 1, 2, \cdots, s)$ 为 $\bm{A}(\lambda)$ 的不变因子或不变因式.

\par 如果以 $D_i(\lambda)(i = 1, 2, \cdots, s)$ 表示 $\bm{A}(\lambda)$ 的一切 $i$ 阶子式的最大(高)公因式(常称之
为 $\bm{A}(\lambda)$ 的 $i$ 阶行列式因子, 由行列式性质知 $D_i(\lambda)$ 不随初等变换而改变), 则 $\bm{A}(\lambda)$ 的不
变因子的计算公式为
$$
    d_i(\lambda) = \dfrac{D_i(\lambda)}{D_{i-1}(\lambda)}, \quad D_0(\lambda) = 1 \quad (i = 1, 2, \cdots, s)
$$
表明, $\bm{A}(\lambda)$ 的标准形式被 $D_i(\lambda) (i = 1, 2, \cdots, s)$ 唯一决定.

\par 把 $\bm{A}(\lambda)$ 的每个次数大于零的不变因子 $d_i(\lambda)$ 分解为不可约因式的乘积, 这样的不可
约因式(连同它们的幂指数)称为 $\bm{A}(\lambda)$ 的一个初等因子, 初等因子的全体称为 $\bm{A}(\lambda)$ 的
初等因子组.

\par 确定 $\bm{A}(\lambda)$ 的初等因子组的一个简便方法是: 用初等变换将 $\bm{A}(\lambda)$ 化为对角矩阵, 若
记对角线上的非零多项式为 $f_i(\lambda)(i = 1,2,\cdots,s)$, 那么诸次数大于零的 $f_i(\lambda)$ 的全体不可
约因式, 就是 $\bm{A}(\lambda)$ 的初等因子组.

\par 要注意的是, 初等因子组是随系数域不同而不同的. 因为有些不变因子在有理数域
上不可约, 但在实数域 $\mathbb{R}$ 或复数域 $\mathbb{C}$ 上确是可约的.

\par 那么, 可以得到, 在复数域 $\mathbb{C}$ 上, 求 $n$ 阶矩阵 $\bm{A}$ 的 Jordan 标准形的步骤如下:
\par 第一步: 求特征矩阵 $\lambda\bm{I} - \bm{A}$ 的初等因子组, 设为
$$
    (\lambda - \lambda_1)^{m_1}, (\lambda - \lambda_2)^{m_2}, \cdots, (\lambda - \lambda_s)^{m_s}
$$
其中, $\lambda_1, \lambda_2, \cdots, \lambda_s$ 可能由相同的, 指数 $m_1, m_2, \cdots, m_s$ 也可能有相同的, 且
$$
    m_1 + m_2 + \cdots + m_s = n
$$
\par 第二步: 写出每个初等因子 $(\lambda - \lambda_i)^{m_i}(i = 1, 2, \cdots, s)$ 对应的 Jordan 块
$$
    \bm{J}_i(\lambda_i) = \begin{bmatrix}
        \lambda_i & 1         &           &        &           \\
                  & \lambda_i & 1         &        &           \\
                  &           & \lambda_i & \ddots &           \\
                  &           &           & \ddots & 1         \\
                  &           &           &        & \lambda_i
    \end{bmatrix} \quad (i = 1, 2, \cdots, s)
$$
\par 第三步: 写出以这些 Jordan 块构成的 Jordan 标准形
$$
    \bm{J} = \begin{bmatrix}
        \bm{J}_1(\lambda_1) &                       &        &                     \\
                            & \bm{J}_2(\lambda_2) 7 &                              \\
                            &                       & \ddots &                     \\
                            &                       &        & \bm{J}_s(\lambda_s)
    \end{bmatrix}
$$

\paragraph*{例 2.8} 求矩阵 $\bm{A}$ 的 Jordan 标准形, 其中
$$
    \bm{A} = \begin{bmatrix}
        -1 & 1 & 0 \\
        -4 & 3 & 0 \\
        1  & 0 & 2
    \end{bmatrix}
$$

\paragraph*{解} 求 $\lambda\bm{I} - \bm{A}$ 的初等因子组

\begin{align*}
    \lambda\bm{I} - \bm{A} = & \begin{bmatrix}
                                   \lambda + 1 & -1          & 0           \\
                                   4           & \lambda - 3 & 0           \\
                                   -1          & 0           & \lambda - 2
                               \end{bmatrix} \to \\
                             & \begin{bmatrix}
                                   1 & 0               & 0           \\
                                   0 & (\lambda - 1)^2 & 0           \\
                                   0 & -1              & \lambda - 2
                               \end{bmatrix} \to       \\
                             & \begin{bmatrix}
                                   1 & 0 & 0                            \\
                                   0 & 1 & 0                            \\
                                   0 & 0 & (\lambda - 2)(\lambda - 1)^2
                               \end{bmatrix}
\end{align*}
因此, 所求的初等因子组为 $\lambda - 2, (\lambda - 1)^2$. 于是有
$$
    \bm{A} \sim \bm{J} = \begin{bmatrix}
        2 & 0 & 0 \\
        0 & 1 & 1 \\
        0 & 0 & 1
    \end{bmatrix}
$$

\paragraph*{定理 2.12} 每个 $n$ 阶复矩阵 $\bm{A}$ 都与一个 Jordan 标准形相似, 这个 Jordan 标准形除去
其中 Jordan 块的排列次序外, 是被 $\bm{A}$ 唯一确定的.

\par 对于所需要的可逆矩阵 $\bm{P}$, 下面给出特殊情况下 $\bm{P}$ 的计算方法.

$$
    \bm{P}^{-1}\bm{AP} = \bm{J} = \begin{bmatrix}
        \lambda_1 & 0         & 0         \\
                  & \lambda_2 & 1         \\
                  &           & \lambda_2
    \end{bmatrix}
$$
其中 $\bm{P} = (\bm{x}_1, \bm{x}_2, \bm{x}_3)$, 于是有
$$
    \bm{A}(\bm{x}_1, \bm{x}_2, \bm{x}_3) = (\bm{x}_1, \bm{x}_2, \bm{x}_3)\begin{bmatrix}
        \lambda_1 & 0         & 0         \\
                  & \lambda_2 & 1         \\
                  &           & \lambda_2
    \end{bmatrix}
$$
即
$$
    (\bm{Ax}_1, \bm{Ax}_2, \bm{Ax}_3) = (\lambda_1\bm{x}_1, \lambda_2\bm{x}_2, \bm{x}_2 + \lambda_2\bm{x}_3)
$$
由此可得
\begin{gather}
    \begin{cases}
        (\lambda_1\bm{I} - \bm{A})\bm{x}_1 = \bm{0} \\
        (\lambda_2\bm{I} - \bm{A})\bm{x}_2 = \bm{0} \\
        (\lambda_2\bm{I} - \bm{A})\bm{x}_3 = -\bm{x}-2
    \end{cases}
    \tag{2.1.10}
\end{gather}
从而 $\bm{x}_1, \bm{x}_2$ 依次是 $\bm{A}$ 的属于 $\lambda_1, \lambda_2$ 的特征向量. $\bm{x}_3$ 是最后一个非齐次线性方程组的解
向量. 求出这些解向量就得到了所需要的矩阵 $\bm{P}$.

\par 在一般情况下, 如果 $\lambda_1$ 是 $\bm{A}$ 的 $k$ 重特征值, 则 $\bm{x}_1, \bm{x}_2, \cdots, \bm{x}_k$ 可由解下面各方程组
$$
    (\lambda_1\bm{I} - \bm{A})\bm{x}_1 = \bm{0} \\
    (\lambda_1\bm{I} - \bm{A})\bm{x}_i = -\bm{x}_{i-1} \quad (i = 2, 3, \cdots k)
$$
而获得. 这样得到的 $\bm{x}_1, \bm{x}_2, \cdots, \bm{x}_k$ 线性无关, 于是
$$
    \bm{P} = (\bm{x}_2, \bm{x}_3, \cdots, \bm{x}_k, \cdots)
$$
称 $\bm{x}_2, \bm{x}_3, \cdots, \bm{x}_k$ 为 $\bm{A}$ 属于 $\lambda_1$ 的广义特征向量.

\paragraph*{例 2.9} 求例2.8中, 使矩阵 $\bm{A}$ 相似于 Jordan 标准形时所用的可逆矩阵 $\bm{P}$.

\paragraph*{解} 因为 $\lambda_1 = 2, \lambda_2 = 1$ 分别是其单特征值和二重特征值, 所以可用式(2.1.10)求 $\bm{P} =
    (p_{ij})_{3\times 3}$, 这里 $(p_{1i},p_{2i},p_{3i})^T = \bm{x}_i(i = 1, 2, 3)$.解方程组
$$
    (2\bm{I} - \bm{A})\bm{x}_1 = \bm{0},\quad (\bm{I} - \bm{A})\bm{x}_2 = \bm{0}, \quad (\bm{I} - \bm{A})\bm{x}_3 = -\bm{x}_2
$$
得特征向量 $\bm{x}_1, \bm{x}_2$ 及广义特征向量 $\bm{x}_3$ 依次为
$$
    \bm{x}_1 = (0, 0, 1)^T,\quad \bm{x}_2 = (1, 2, -1)^T, \quad \bm{x}_3 = (0, 1, -1)^T
$$
故所求矩阵 $\bm{P}$ 为
$$
    \bm{P} = \begin{bmatrix}
        0 & 1  & 0  \\
        0 & 2  & -1 \\
        1 & -1 & -1
    \end{bmatrix}
$$


\subsubsection{欧式空间中线性变换的求法}

本节讨论在欧氏空间中的线性变换, 特别是正交变换和对称变换, 它们与几何密切
相关.

\paragraph[]{欧氏空间的性质} \

\par 假定 $\bm{x}_1, \bm{x}_2, \cdots, \bm{x}_n$ 是 $n$ 维欧氏空间 $V^n$ 的基, 对于 $V^n$ 的任意两个向量
$$
    \bm{x} = \xi_1\bm{x}_1 + \xi_2\bm{x}_2 + \cdots + \xi_n\bm{x}_n, \bm{y} = \eta_1\bm{x}_1 + \eta_2\bm{x}_2 + \cdots + \eta_n\bm{x}_n
$$
可得
$$
    (\bm{x}, \bm{y}) = \sum\limits_{i,j=1}^{n} \xi_i\eta_j(\bm{x}_i, \bm{x}_j) = \sum\limits_{i, j=1}^{n} a_{ij}\xi_i\eta_j
$$
其中 $\bm{a}_{ij} = (\bm{x}_i, \bm{x}_j)(i,j = 1,2,\cdots, n)$, 用矩阵乘法表示, 则有
$$
    (\bm{x}, \bm{y}) = (\xi_1, \xi_2, \cdots, \xi_n)\bm{A}\begin{bmatrix}
        \eta_1 \\
        \eta_2 \\
        \vdots \\
        \eta_n
    \end{bmatrix}
$$
这里
$$
    \bm{A} = (a_{ij})_{n\times n} = \begin{bmatrix}
        (\bm{x}_1, \bm{x}_1) & \cdots & (\bm{x}_1, \bm{x}_n) \\
        (\bm{x}_2, \bm{x}_1) & \cdots & (\bm{x}_2, \bm{x}_n) \\
        \vdots               &        & \vdots               \\
        (\bm{x}_n, \bm{x}_1) & \cdots & (\bm{x}_n, \bm{x}_n)
    \end{bmatrix}
$$
\par 可以看出, 只要知道其中任意两个基向量的内积, 也就知道了矩阵 $\bm{A}$, 从而也就知
道了任意两个向量的内积. 因此, 称 $\bm{A}$ 为 $V^n$ 对于基 $\bm{x}_1, \bm{x}_2, \cdots, \bm{x}_n$ 的度量矩阵, 且是对
称矩阵、正定矩阵.

\paragraph[]{正交性} \

\par 通常, 两个向量垂直的充分必要条件是它们夹角的余弦为零, 亦即它们的数量积为零. 在一般的欧氏空间中, 仍以内积定义二向量夹角的余弦.

\paragraph*{定义 2.13} 如果对于欧氏空间中的两个向量 $\bm{x}$ 与 $\bm{y}$, 有 $(\bm{x}, \bm{y}) = 0$, 则称 $\bm{x}$ 与 $\bm{y}$ 正交或垂
直, 记为 $\bm{x} \perp \bm{y}$.

\paragraph*{定义 2.14} 如果欧氏空间中一组非零向量两两正交, 则成为正交向量组.

\paragraph*{定理 2.13} 在欧氏空间中, 如果 $\bm{x}_1, \bm{x}_2, \cdots, \bm{x}_n$ 是正交向量组, 则它们必线性无关.

\paragraph*{定义 2.15} 在欧氏空间 $V^n$ 中, 由 $n$ 个非零向量组成的正交向量组称为 $V^n$ 的正交基; 由
单位向量组成的正交基称为标准正交基或法正交基.

\par 把一个正交基进行单位化, 就得到一个标准正交基.

\par 一个基为标准正交基的充要条件是它的度量矩阵为单位矩阵. 事实上, 标准正交基
$\bm{x}_1, \bm{x}_2, \cdots, \bm{x}_n$ 满足
$$
    (\bm{x_i}, \bm{x_j}) = \delta_{ij} = \begin{cases}
        1 & (i = j)    \\
        0 & (i \neq j)
    \end{cases}
$$
这里的 $\delta_{ij}$ 称为 Kronecker 记号. 于是其度量矩阵是单位矩阵. 反之, 如果单位矩阵为度
量矩阵, 则由矩阵相等可得 $(\bm{x}_i, \bm{x}_j) = \delta_{ij}$, 即 $\bm{x}_1, \bm{x}_2, \cdots, \bm{x}_n$ 为标准正交基.

\paragraph*{定理 2.14} 对于欧氏空间 $V^n$ 的任一基 $\bm{x}_1, \bm{x}_2, \cdots, \bm{x}_n$, 都可以找到一个标准正交基, 令其
为 $\bm{y}_1, \bm{y}_2, \cdots, \bm{y}_n$. 换言之, 任一非零欧氏空间都有正交基和标准正交基.

\paragraph*{证} 应用下面论述的关于向量组的 Schmidt 正交化方法, 给出定理的构造性证明. 为此取 $\bm{y}_1' = \bm{x}_1$, 作为所求正交基中的第一个向量. 再令
$$
    \bm{y}'_2 = \bm{x}_2 + k\bm{y}'_1
$$
由正交条件 $(\bm{y}_2', \bm{y}_1') = 0$ 来决定待定常数 $k$. 由
$$
    (\bm{x}_2 + k\bm{y}_1', \bm{y}_1') = (\bm{x}_2, \bm{y}_1') + k(\bm{y}_1', \bm{y}_1') = 0
$$
得
$$
    k = -\dfrac{(\bm{x}_2, \bm{y}_1')}{(\bm{y}_1', \bm{y}_1')}
$$
这样就得到两个正交的向量 $\bm{y}_1', \bm{y}_2'$, 且 $\bm{y}_2' \neq \bm{0}$. 又令
$$
    \bm{y}_3' = \bm{x}_3 + k_2\bm{y}_2' + k_1\bm{y}_1'
$$
再有正交条件 $(\bm{y}_3', \bm{y}_2') = 0$ 及 $(\bm{y}_3', \bm{y}_1') = 0$ 来决定出 $k_1$ 和 $k_2$ 为
$$
    k_2 = -\dfrac{(\bm{x}_3, \bm{y}_2')}{(\bm{y}_2', \bm{y}_2')}, \quad k_1 = -\dfrac{(\bm{x}_3, \bm{y}_1')}{\bm{y}_1', \bm{y}_1'}
$$
\par 到此, 已经做出三个两两正交的向量 $\bm{y}_1', \bm{y}_2', \bm{y}_3'$, 且 $\bm{y}_3' \neq \bm{0}$. 继续这样进行下去, 设已
做出 $m$ 个两两正交且不为零的向量 $\bm{y}_1', \bm{y}_2', \cdots, \bm{y}_m'$, 为求出第 $m + 1$ 个与之正交的向量, 令
$$
    \bm{y}_{m+1}' = \bm{x}_{m+1} + l_m\bm{y}_m' + l_{m -1}\bm{y}_{m-1}' + \cdots + l_2\bm{y}_2' + l_1\bm{y}_1'
$$
使用 $m$ 个正交条件
$$
    (\bm{y}_{m+1}', \bm{y}_i') = 0 (i = 1,2,\cdots, m)
$$
来决定 $l_m, l_{m-1}, \cdots, l_2, l_1$. 根据 $\bm{y}_1', \bm{y}_2', \cdots, \bm{y}_m'$ 两两正交的假设, 可得
$$
    (\bm{x}_{m+1},y_i') + l_i(\bm{y}_i', \bm{y}_i')
$$
故
$$
    l_i = -\dfrac{(\bm{x}_{m+1}, \bm{y}_i')}{(\bm{y}_i', \bm{y}_i')} (i = 1,2,\cdots, m)
$$
于是 $\bm{y}_{m+1}'$ 就被确定出来了.

\par 采用上述 Schmidt 正交化方法, 可由已知基构造出 $n$ 个两两正交的线性无关的非零
向量 $\bm{y}_1', \bm{y}_2', \cdots, \bm{y}_n'$, 从而形成 $V^n$ 的一组正交基. 再以 $\lvert \bm{y}_i' \rvert$ 除 $\bm{y}_i'(i = 1,2,\cdots, n)$, 就得到定
理所要求的标准正交基.
$$
    \bm{y}_i = \dfrac{1}{\lvert \bm{y}_i' \rvert} \bm{y}_i' (i = 1,2,\cdots, n)
$$

\par 上述是由基 $\bm{x}_1, \bm{x}_2, \cdots, \bm{x}_n$ 构造标准正交基 $\bm{y}_1', \bm{y}_2', \cdots, \bm{y}_n'$ 的过程, 有时也称为把基 $\bm{x}_1, \bm{x}_2, \cdots, \bm{x}_n$ 正交单位化或正交规范化.

\paragraph*{例 2.10} 试把向量组$\bm{x}_1 = (1, 1, 0, 0)$,$\bm{x}_2 = (1, 0, 1, 0),\bm{x}_3 = (-1, 0, 0, 1)$,$\bm{x}_4 =
    (1,-1, -1, 1)$正交单位化.

\paragraph*{解} 先把它们正交化,使用$l_i = -\dfrac{(\bm{x}_{m+1}, \bm{y}_i')}{\bm{y}_i', \bm{y}_i'}$,可得
\begin{align*}
    \bm{y}_1' & = \bm{x}_1 = (1,1, 0 ,0)                                                                                                                                                                                         \\
    \bm{y}_2' & = \bm{x}_2 -\dfrac{(\bm{x}_{2},\bm{y}_1')}{(\bm{y}_1', \bm{y}_1')}y_1' = \bigl(\dfrac{1}{2}, -\dfrac{1}{2}, 1, 0\bigr)                                                                                           \\
    \bm{y}_3' & = \bm{x}_3 -\dfrac{(\bm{x}_{3}, \bm{y}_2')}{(\bm{y}_2', \bm{y}_2')}y_2' -\dfrac{(\bm{x}_{3}, \bm{y}_1')}{(\bm{y}_1', \bm{y}_1')}y_1' = \bigl(-\dfrac{1}{3}, \dfrac{1}{3}, \dfrac{1}{3}, 1\bigr)                  \\
    \bm{y}_4' & = \bm{x}_4 -\dfrac{(\bm{x}_{4},\bm{y}_3')}{(\bm{y}_3', \bm{y}_3')}y_3'-\dfrac{(\bm{x}_{4}, \bm{y}_2')}{(\bm{y}_2', \bm{y}_2')}y_2' -\dfrac{(\bm{x}_{4}, \bm{y}_1')}{(\bm{y}_1', \bm{y}_1')}y_1' = (1, -1, -1, 1)
\end{align*}
再单位化,则有
\begin{align*}
    \bm{y}_1 & = \dfrac{1}{|\bm{y}_1'|}\bm{y}_1' = \bigl(\dfrac{1}{\sqrt{2}}, \dfrac{1}{\sqrt{2}}, 0, 0 \bigr)
\end{align*}
\begin{align*}
    \bm{y}_2 & = \dfrac{1}{|\bm{y}_2'|}\bm{y}_2' = \bigl(\dfrac{1}{\sqrt{6}}, \dfrac{-1}{\sqrt{6}}, \dfrac{2}{\sqrt{6}}, 0\bigr)                         \\
    \bm{y}_3 & = \dfrac{1}{|\bm{y}_3'|}\bm{y}_3' = \bigl(-\dfrac{1}{\sqrt{12}}, \dfrac{1}{\sqrt{12}}, \dfrac{1}{\sqrt{12}}, \dfrac{3}{\sqrt{12}}, \bigr) \\
    \bm{y}_4 & = \dfrac{1}{|\bm{y}_4'|}\bm{y}_4' = \bigl(\dfrac{1}{2}, -\dfrac{1}{2}, -\dfrac{1}{2}, \dfrac{1}{2}\bigr)
\end{align*}

\paragraph[]{正交变换与正交矩阵} \

\par 由解析几何知, 在旋转变化之下, 向量的长度保持不变. 在线性空间中, 能保持向量长度不变的线性变换, 在实际中应用是很广泛的.

\paragraph*{定义 2.16} 设 $\bm{V}$ 为欧氏空间, $T$ 是 $V$ 的一个线性变换, 如果 $T$ 保持 $V$ 中任意向量 $\bm{x}$ 的
长度不变, 则有
$$
    (T\bm{x}, T\bm{x}) = (\bm{x}, \bm{x})
$$
那么称 $T$ 是 $V$ 的一个正交变换.

\paragraph*{定理 2.15} 线性变换 $T$ 为正交变换的充要条件是, 对于欧氏空间 $V$ 中任意向量 $\bm{x}, \bm{y}$, 都
有 $(T\bm{x}, T\bm{y}) = (\bm{x}, \bm{y})$.

\paragraph*{定义 2.17} 如果实方阵 $\bm{Q}$ 满足 $\bm{Q}^T\bm{Q} = \bm{I}$, 则称 $\bm{Q}$ 为正交矩阵.

\par $\bm{Q}$ 是正交矩阵的充要条件是它的列向量是两两正交的单位向量, 此外, 正交矩阵还
有如下性质.
\par (1) 正交矩阵都是可逆的.
\par (2) 正交矩阵的逆矩阵仍是正交矩阵.
\par (3) 两个正交矩阵的乘积仍未正交矩阵.

\paragraph[]{对称变换与对称矩阵} \

\paragraph*{定义 2.18} 设 $T$ 是欧式空间 $V$ 的一个线性变换, 且对 $V$ 中任意两个向量 $\bm{x}, \bm{y}$, 都有
$$
    (T\bm{x}, \bm{y}) = (\bm{x}, T\bm{y})
$$
则称 $T$ 为 $V$ 中的一个对称变换.

\paragraph*{定理 2.16} 欧氏空间的线性变换是实对称变换的充要条件是, 它对于标准正交基的矩阵是实对称矩阵.

\par 实对称矩阵有如下性质.

\par (1) 实对称矩阵的特征值都是实数.
\par (2) 实对称矩阵的不同特征值所对应的特征向量是正交的.

\subsection{向量范数与矩阵范数}

在计算数学中, 特别是在数值代数中, 研究数值方法的收敛性、稳定性及误差分析
等问题时, 范数理论显得十分重要. 本节主要讨论 $n$ 维向量空间 $C^n$ 中的向量范数与矩
阵空间 $C^{m\times n}$ 中的矩阵范数的理论及其性质.

\subsubsection{向量范数介绍}

\paragraph[]{向量范数的概念及 $l_p$ 范数} \

把一个向量(或线性空间的元素)与一个非负实数相联系, 在许多场合下, 这个实数
可以作为向量大小的一种度量. 向量范数就是这样的实数, 它们在研究数值方法的收敛
性和误差分析等方面与有着重要的作用. 现定义如下.

\paragraph*{定义 2.19} 如果 $V$ 是数域 $K$ 上的线性空间, 对任意的 $\bm{x} \in V$, 定义一个实值函数 $\lVert \bm{x} \rVert$, 它
满足以下三个条件:
\par (1) 非负性: 当 $\bm{x} \neq \bm{0}$ 时, $\lVert \bm{x} \rVert > 0$; 当 $\bm{x} = \bm{0}$ 时, $\lVert \bm{x} \rVert = 0$;
\par (2) 齐次性: $\lVert a\bm{x} \rVert = \lvert a \rvert \ \lVert \bm{x} \rVert (a \in K, \bm{x} \in V)$;
\par (3) 三角不等式: $\lVert \bm{x} + \bm{y} \rVert \leqslant \lVert \bm{x} \rVert + \lVert \bm{y} \rVert (\bm{x}, \bm{y} \in V)$.
\\ 则称 $\lVert \bm{x} \rVert$ 为 $V$ 上向量 $\bm{x}$ 的范数, 简称向量范数.

\paragraph*{例 2.11} 证明 $\lVert \bm{x} \rVert = \sqrt{\lvert \xi_1 \rvert^2 + \lvert \xi_2 \rvert^2 + \cdots + \lvert \xi_n \rvert^2}$ 是 $C^n$ 上的一种范数, $\bm{x} = (\xi_1, \xi_2, \cdots, \xi_n) \in C^n$.

\paragraph*{证} 当 $\bm{x} \neq \bm{0}$ 时, 显然 $\lVert \bm{x} \rVert > 0$; 当 $\bm{x} = \bm{0}$ 时, 有 $\lVert \bm{x} \rVert = 0$.

\par 对于任意的复数 $a$, 有
$$
    \lVert a\bm{x} \rVert = \lvert a \rvert\sqrt{\lvert \xi_1 \rvert^2 + \lvert \xi_2 \rvert^2 + \cdots + \lvert \xi_n \rvert^2} = \lvert a \rvert \ \lVert \bm{x} \rVert
$$
\par 对于任意两个向量 $\bm{x}, \bm{y} \in C^n$, 有
$$
    \lVert \bm{x} + \bm{y} \rVert^2 = (\bm{x} + \bm{y}, \bm{x} + \bm{y}) = (\bm{x}, \bm{x}) + 2\mathrm{Re}(\bm{x}, \bm{y}) + (\bm{y}, \bm{y})
$$
因此,
$$
    \lVert \bm{x} + \bm{y} \rVert^2 \leqslant \lVert \bm{x} \rVert^2 + 2\lVert \bm{x} \rVert \ \lVert\bm{y} \rVert + \lVert \bm{y} \rVert^2 = (\lVert \bm{x} \rVert+ \lVert\bm{y} \rVert)^2
$$
即 $\lVert \bm{x} + \bm{y} \rVert \leqslant \lVert \bm{x} \rVert + \lVert \bm{y} \rVert$.

\par 称例2.11中的范数为向量的 $2$ -范数, 记为 $\lVert \bm{x} \rVert _2$, 即
\begin{equation}
    \lVert \bm{x} \rVert _2 = \sqrt{\lvert \xi_1 \rvert^2 + \lvert \xi_2 \rvert^2 + \cdots + \lvert \xi_n \rvert^2}
    \tag{2.2.1}
\end{equation}

\paragraph*{例 2.12} 证明 $\lVert \bm{x} \rVert = \mathop{\max}\limits_{i} \lvert \xi_i \rvert$ 是 $C^n$ 上的一种范数, 这里 $\bm{x} = (\xi_1, \xi_2, \cdots, \xi_n) \in C^n$.

\paragraph*{证} 当 $\bm{x} \neq \bm{0}$ 时, $\lVert \bm{x} \rVert = \mathop{\max}\limits_{i} \lvert \xi_i \rvert > 0$; 当 $\bm{x} = \bm{0}$ 时, 显然有 $\lVert \bm{x} \rVert = 0$.
\par 又对任意的 $a \in C$, 有
\begin{equation*}
    \lVert a\bm{x} \rVert = \mathop{\max}_{i} \lvert a\xi_i \rvert = \lvert a \rvert \mathop{\max}_{i} \lvert \xi_i\rvert
\end{equation*}
\par 对 $C^n$ 的任意两个向量 $\bm{x} = (\xi_1, \xi_2, \cdots, \xi_n), \bm{y} = (\eta_1, \eta_2, \cdots, \eta_n)$, 有
$$
    \lVert \bm{x} + \bm{y} \rVert = \mathop{\max}_{i} \lvert \xi_i + \eta_i \rvert \leqslant \mathop{\max}\limits_{i} \lvert \xi_i \rvert + \mathop{\max}\limits_{i} \lvert \eta_i \rvert = \lVert \bm{x} \rVert + \lVert \bm{y} \rVert
$$
因此, $\lVert \bm{x} \rVert = \mathop{\max}\limits_{i} \lvert \xi_i \rvert$ 是 $C^n$ 的一种范数.

\par 称例2.12中的范数为向量的 $\infty $ -范数, 记为 $\lVert \bm{x} \rVert _{\infty}$, 即
\begin{equation}
    \lVert \bm{x} \rVert _\infty = \mathop{\max}_{i} \lvert \xi_i \rvert
    \tag{2.2.2}
\end{equation}

\paragraph*{例 2.13} 证明 $\lVert \bm{x} \rVert = \sum\limits_{i = 1}^n \lvert \xi_i \rvert$ 是 $C^n$ 上的一种范数, 其中 $\bm{x} = (\xi_1, \xi_2, \cdots, \xi_n) \in C^n$.

\paragraph*{证} 当 $\bm{x} \neq \bm{0}$ 时, 显然 $\lVert \bm{x} \rVert = \sum\limits_{i=0}^n \lvert \xi_i \rvert > 0$; 当 $\bm{x} = \bm{0}$ 时, 由于 $\bm{x}$ 的每一分量都是零, 故 $\lVert \bm{x} \rVert = 0$.
\par 又对于任意 $a \in C^n$, 有
$$
    \lVert \bm{x} \rVert = \sum_{i = 1}^n \lvert a\xi_i \rvert = \lvert a \rvert \sum\limits_{i=1}^n \lvert \xi_i \rvert = \lvert a \rvert \ \lVert \bm{x} \rVert
$$
\par 对于任意两个向量 $\bm{x}, \bm{y} \in C^n$, 有
\begin{align*}
    \lVert \bm{x} + \bm{y} \rVert = & \sum\limits_{i = 1}^n \lvert \xi_i + \eta_i \rvert \leqslant \sum\limits_{i=1}^n(\lvert xi_i \rvert + \lvert \eta_i \rvert) = \\
                                    & = \sum\limits_{i=1}^n \lvert \xi_i \rvert +  \sum\limits_{i=1}^n \lvert \eta_i \rvert = \lVert \bm{x} + \bm{y} \rVert
\end{align*}
因此, $\lVert \bm{x} \rVert = \sum\limits_{i=1}^n \lvert \xi_i \rvert$ 是 $C^n$ 的一种范数.

\par 称例2.13中的范数为向量的 $1$ -范数, 记为 $\lVert \bm{x} \rVert _1$, 即
\begin{equation}
    \lVert \bm{x} \rVert _1 = \sum_{i=1}^{n} \lvert \xi_i \rvert
    \tag{2.2.3}
\end{equation}

\par 由例2.11 $\sim$ 例2.13可知, 在一个线性空间中, 可以定义多种向量范数, 实际上可以定义无限多种范数. 例如,对于不小于1的任意实数 $p$ 及 $\bm{x} = (\xi_1, \xi_2, \cdots, \xi_n) \in C^n$, 可以证明实值函数
$$
    (\sum\limits_{i=1}^n \lvert \xi_i \rvert^p)^{1/p} \quad (1 \leqslant p < +\infty)
$$
满足向量范数的三个条件. 称 $(\sum\limits_{i=1}^n \lvert \xi_i \rvert^p)^{1/p}$ 为向量 $\bm{x}$ 的 $p$ -范数或 $l_p$ 范数, 记为 $\lVert \bm{x} \rVert _p$, 即
\begin{equation}
    \lVert \bm{x} \rVert _p = (\sum{i=1}^n \lvert \xi_i \rvert^p)^{1/p}
    \tag{2.2.4}
\end{equation}

\paragraph[]{线性空间 $V^n$ 上的向量范数的等价性} \

\paragraph*{定理 2.17} 设 $\lVert \bm{x} \rVert _\alpha$ 和 $\lVert \bm{x} \rVert _\beta$ 为有限维线性空间 $V$ 上的任意两种向量范数(它们不限于 $p$ -范数), 则存在两个与向量 $\bm{x}$ 无关的正常数 $c_1$ 和 $c_2$, 使满足
$$
    c_1 \lVert \bm{x} \rVert _\beta \leqslant \lVert \bm{x} \rVert _\alpha \leqslant c_2 \lVert \bm{x} \rVert _\beta \quad (\forall \bm{x} \in V)
$$
且满足这一不等式的两种范数被称为是等价的.

\par 对于 $C^n$ 上向量 $\bm{x}$ 的 $p$ -范数, 也满足不等式
$$
    1 \lVert \bm{x} \rVert _\infty \leqslant \lVert \bm{x} \rVert _1 \leqslant n \lVert \bm{x} \rVert _\infty \qquad
    1 \lVert \bm{x} \rVert _\infty \leqslant \lVert \bm{x} \rVert _2 \leqslant \sqrt{n} \lVert \bm{x} \rVert _\infty
$$
以上两式表明, 对某一向量 $\bm{x}$ 而言, 如果它的某一种范数小(或大), 那么它的另两种范数也小(或大).

\subsubsection{矩阵范数介绍}

矩阵空间 $C^{n\times n}$ 是一个 $mn$ 维的线性空间, 将 $m\times n$ 矩阵 $\bm{A}$ 看做线性空间 $C^{m\times n}$ 中
的“向量”, 于是可以定义 $\bm{A}$ 的范数. 但是, 矩阵之间还有乘法运算, 它应该在定义矩阵范
数时予以体现.

\paragraph[]{矩阵范数的定义与性质} \

\paragraph*{定义 2.20} 设 $\bm{A} \in C^{m\times n}$, 定义一个实值函数 $\lVert \bm{A} \rVert$, 它满足以下三个条件:
\par (1) 非负性: 当 $\bm{A} \neq \bm{O}$ 时, $\lVert \bm{A} \rVert > 0$; 当 $\bm{A} = \bm{O}$ 时, $\lVert \bm{A} \rVert = 0$;
\par (2) 齐次性: $\lVert \alpha\bm{A} \rVert = \lvert \alpha \rvert \ \lVert \bm{A} \rVert (\alpha \in C)$;
\par (3) 三角不等式: $\lVert \bm{A} + \bm{B} \rVert \leqslant \lVert \bm{A} \rVert + \lVert \bm{B} \rVert (\bm{B} \in C^{m \times n})$.
则称 $\lVert \bm{A} \rVert$ 为 $\bm{A}$ 的广义矩阵范数. 若对 $C^{m\times n}, C^{n\times l}$ 及 $C^{m\times l}$ 上的同类广义矩阵范数 $\lVert \bm{\cdot} \rVert$, 还满足下面一个条件:
\par (4) 相容性:
$$
    \lVert \bm{AB} \rVert \leqslant \lVert \bm{A} \rVert \ \lVert \bm{B} \rVert \quad (\bm{B} \in C^{n\times l})
$$
则称 $\lVert \bm{A} \rVert$ 为 $\bm{A}$ 的矩阵范数.

\par 如同向量范数的情况一样, 矩阵范数也是多种多样的. 但是, 在数值方法中进行某
种估计时, 遇到的多数情况时: 矩阵范数常与向量范数混合在一起使用, 而矩阵范数经
常是作为两个线性空间上的线性映射(变换)出现的. 因此, 考虑一些矩阵范数时, 应该
使它能与向量范数联系起来. 这可由矩阵范数与向量范数相容的概念来体现.

\paragraph*{定义 2.21} 对于 $bm{C}^{m \times n}$ 上的矩阵范数 $\lVert \bm{\cdot} \rVert _M$ 和 $C^m$ 与 $C^n$ 上的同类向量范数 $\lVert \bm{\cdot} \rVert _V$, 如果
$$
    \lVert \bm{Ax} \rVert _V \leqslant \lVert \bm{A} \rVert _M \lVert \bm{x} \rVert _V \quad (\forall \bm{A} \in C^{m\times n}, \forall \bm{x} \in C^n)
$$
则称矩阵范数 $\lVert \bm{\cdot} \rVert _M$ 与 向量范数 $\lVert \bm{\cdot} \rVert _V$ 是相容的.

\paragraph*{例 2.14} 设 $\bm{A} = (a_{ij})_{m\times n} \in C^{n\times n}$, 证明函数
\begin{equation}
    \lVert \bm{A} \rVert _F = (\sum_{i=1}^m\sum_{j=1}^n \lvert a_{ij} \rvert^2)^{1/2} = (\mathrm{tr}(\bm{A}^H\bm{A}))^{1/2}
    \tag{2.2.5}
\end{equation}
是 $C^{n\times n}$ 上的矩阵范数, 且与向量范数 $\lVert \bm{\cdots} \rVert _2$ 相容.

\paragraph*{证} 显然, $\lVert \bm{A} \rVert _F$ 具有非负性和齐次性. 设 $\bm{B} \in C^{m\times n}$, 且 $\bm{A}$ 的第 $j$ 列分别为 $\bm{a}_j, b_{j}(j=
    1,2,\cdots,n)$, 则有
\begin{align*}
    \lVert \bm{A} + \bm{B} \rVert ^2_F = & \lVert \bm{a}_1 + \bm{b}_1 \rVert _2^2 + \cdots + \lVert \bm{a}_n + \bm{b}_n \rVert _2^2 \leqslant                                       \\
                                         & (\lVert \bm{a}_1 \rVert _2 + \lVert \bm{b}_1 \rVert _2)^2 + \cdots + (\lVert \bm{a}_n \rVert _2 + \lVert \bm{b}_n \rVert _2)^2 \leqslant \\
                                         & \lVert \bm{A} \rVert _F^2 + 2\lVert \bm{A} \rVert _F \lVert \bm{B} \rVert _F + \lVert \bm{B} \rVert _F^2 =                               \\
                                         & (\lVert \bm{A} \rVert _F + \lVert \bm{B} \rVert _F)^2
\end{align*}
即三角不等式成立.
\par 再设 $\bm{B} = (b_{ij})_{n\times l} \in C^{n\times l}$, 则 $\bm{AB} = (\sum\limits_{k=1}^{n}a_{ik}b_{kj})_{m\times l} \in C^{m\times l}$, 于是有
$$
    \lVert \bm{AB} \rVert _F^2 = \sum\limits_{i=1}^m\sum\limits_{j=1}^l \lvert \sum\limits_{k=1}^na_{ik}b_{kj} \rvert^2 \leqslant \sum\limits_{i=1}^m\sum\limits_{j=1}^l (\sum\limits_{k=1}^n \lvert a_{ik}\rvert \lvert b_{kj} \rvert)^2
$$
可得
$$
    \lVert \bm{AB} \rVert _F^2 \leqslant \lVert \bm{A} \rVert _F^2 \lVert \bm{B} \rVert _F^2
$$
即 $\lVert \bm{A} \rVert _F$ 是 $\bm{A}$ 的矩阵范数.

\par 取 $\bm{B} = \bm{x} \in C^{n\times l}$, 则有
$$
    \lVert \bm{Ax} \rVert _2 = \lVert \bm{AB} \rVert _F \leqslant \lVert \bm{A} \rVert _F \lVert \bm{B} \rVert _F = \lVert \bm{A} \rVert _F \lVert \bm{x} \rVert _2
$$
即矩阵范数 $\lVert \bm{\cdot} \rVert _F$ 与向量范数 $\lVert \bm{\cdot} \rVert _2$ 相容.

\par 范数(2.2.5)又称为 Frobenuis范数, 或简称为 $F$ -范数.

\paragraph[]{几种常用的矩阵范数} \

\par 现在给出一种规定矩阵范数的具体方法, 使矩阵范数与已知的向量范数相容.

\paragraph*{定理 2.18} 已知 $C^m$ 和 $C^n$ 上的同类向量范数 $\lVert \bm{\cdot} \rVert$, 设 $\bm{A} \in C^{m\times n}$, 则函数
\begin{equation}
    \lVert \bm{A} \rVert = \mathop{\max}_{\lVert \bm{x} \rVert = 1} \lVert \bm{Ax} \rVert
    \tag{2.2.6}
\end{equation}
是 $C^{m\times n}$ 上的矩阵范数, 且与已知的向量范数相容.

\par 称由式(2.2.6)给出的矩阵范数为由向量范数导出的矩阵范数, 简称为从属范数. 对于 $C^{m\times n}$ 上的任何一种从属范数, 有
$$
    \lVert \bm{I} \rVert = \mathop{\max}\limits_{\lVert \bm{x} \rVert = 1} \lVert \bm{Ix} \rVert = 1
$$
但对于一般的矩阵范数(设该矩阵范数与某向量范数相容), 由于
$$
    \lVert \bm{x} \rVert = \lVert \bm{Ix} \rVert \leqslant \lVert \bm{I} \rVert \ \lVert \bm{x} \rVert
$$
对任意的 $\bm{x} \in C^n$ 成立, 所以 $\lVert \bm{I} \rVert \geqslant 1$.

\par 上面论述表明, 矩阵范数是与向量范数密切相关的, 有什么样的向量范数就有什么
样的矩阵范数. 当在式(2.2.6)中取向量 $\bm{x}$ 的范数 $\lVert \bm{x} \rVert$ 依次为 $\lVert \bm{x} \rVert _1, \lVert \bm{x} \rVert _2, \lVert \bm{x} \rVert _\infty$ 时, 就得到
三种常用的矩阵范数, 如下.

\paragraph*{定理 2.19} 设 $\bm{A} = (a_{ij})_{m\times n} \in C^{m\times n}, \bm{x} = (\xi_1, \xi_2, \cdots, \xi_n)^T \in C^n$, 则从属于向量 $\bm{x}$ 的三种范数 $\lVert \bm{x} \rVert _1, \lVert \bm{x} \rVert _2, \lVert \bm{x} \rVert _\infty$ 的矩阵范数计算公式依次为
\par (1) $\lVert \bm{A} \rVert _1 = \mathop{\max}\limits_{j} \sum\limits_{i = 1}^m \lvert a_{ij} \rvert$;
\par (2) $\lVert \bm{A} \rVert _2 = \sqrt{\lambda_1}$, $\lambda_1$ 为 $\bm{A}^{H}\bm{A}$ 的最大特征值;
\par (3) $\lVert \bm{A} \rVert _\infty = \mathop{\max}\limits_i \sum\limits_{j = 1}^n \lvert a_{ij} \rvert$;
\par 通常称 $\lVert \bm{A} \rVert _1, \lVert \bm{A} \rVert _2, \lVert \bm{A} \rVert _\infty$ 依次为列和范数、谱范数及行和范数.

\subsubsection{矩阵可逆性条件、谱半径和条件数介绍}

本节主要介绍矩阵范数的几点应用.

\paragraph[]{矩阵的可逆性条件} \

\par 设 $\bm{A} \in C^{n\times n}$, 可以根据范数 $\lVert \bm{A} \rVert$ 的大小来判断 $\bm{I} - \bm{A}$ 是否为可逆矩阵.

\paragraph*{定理 2.20} 设 $\bm{A} \in C^{n\times n}$, 且对于 $C^{n\times n}$ 上的某种矩阵范数 $\lVert \bm{\cdot} \rVert$, 有 $\lVert \bm{A} \rVert < 1$, 则矩阵 $\bm{I} - \bm{A}$ 可逆, 且有
$$
    \lVert (\bm{I} - \bm{A})^{-1} \rVert \leqslant \dfrac{\lVert \bm{I} \rVert}{1 - \lVert \bm{A} \rVert}
$$

\paragraph*{证} 设矩阵范数 $\lVert \bm{A} \rVert$ 与向量范数 $\lVert \bm{x} \rVert _V$ 相容, 如果 $\det (\bm{I} - \bm{A}) = 0$, 则齐次线性方程组 $(\bm{I} - \bm{A})\bm{x} = \bm{0}$ 有非零解 $\bm{x}_0$, 即
$$
    (\bm{I} - \bm{A})\bm{x}_0 = \bm{0}
$$
从而有
$$
    \lVert \bm{x}_0 \rVert _V = \lVert \bm{Ax}_0 \rVert _V \leqslant \lVert \bm{A} \rVert \ \lVert \bm{x}_0 \rVert _V < \lVert \bm{x}_0 \rVert _V
$$
则是一个矛盾, 故 $\det (\bm{I} - \bm{A}) \neq 0$, 即 $\bm{I} - \bm{A}$ 可逆.
\par 再由 $(\bm{I} - \bm{A})^{-1}(\bm{I} - \bm{A}) = \bm{I}$ 可得
$$
    (\bm{I} - \bm{A})^{-1} = \bm{I} + (\bm{I} - \bm{A})^{-1}\bm{A}
$$
利用范数的三角不等式与相容性可得
$$
    \lVert (\bm{I} - \bm{A}^{-1}) \rVert \leqslant \lVert \bm{I} \rVert + \lVert (\bm{I} - \bm{A})^{-1} \rVert \ \lVert \bm{A} \rVert
$$
解此不等式可得证.

\par 现在考虑如下问题: 若矩阵 $\bm{A}$ 的范数 $\lVert \bm{A} \rVert$ 很小, 且由于 $\lVert \bm{A} \rVert$ 是它的元素的连续函
数, 所以矩阵 $\bm{A}$ 接近于零矩阵 $\bm{O}$, 而 $\bm{I} - \bm{O}$ 的逆矩阵为 $\bm{I}$, 那么, $(\bm{I} - \bm{A})^{-1}$ 与单位矩阵 $\bm{I}$
的逼近程度可由下面的定理给出.

\paragraph*{定理 2.21} 设 $\bm{A} \in C^{n\times n}$, 且对 $C^{n\times n}$ 上的某种矩阵范数 $\lVert \bm{\cdot} \rVert$, 有 $\lVert \bm{A} \rVert < 1$, 则
$$
    \lVert \bm{I} - (\bm{I} - \bm{A})^{-1} \rVert \leqslant \dfrac{\lVert \bm{A} \rVert}{1 - \lVert \bm{A} \rVert}
$$

\paragraph*{证} 因为 $\lVert \bm{A} \rVert < 1$, 所以 $(\bm{I} - \bm{A})^{-1}$ 存在, 给 $(\bm{I} - \bm{A}) - \bm{A} = -\bm{A}$ 右乘 $(\bm{I} - \bm{A})^{-1}$ 可得
$$
    \bm{I} - (\bm{I} - \bm{A})^{-1} = -\bm{A}(\bm{I} - \bm{A})^{-1}
$$
利用范数的三角不等式与相容性可得
$$
    \lVert \bm{A}(\bm{I} - \bm{A})^{-1} \rVert \leqslant \lVert \bm{A} \rVert + \lVert \bm{A} \rVert \ \lVert \bm{A}(\bm{I} - \bm{A})^{-1} \rVert
$$
即 $\lVert \bm{A}(\bm{I} - \bm{A})^{-1} \rVert \leqslant \dfrac{\lVert \bm{A} \rVert}{1 - \lVert \bm{A} \rVert}$, 故
$$
    \lVert \bm{I} - (\bm{I} - \bm{A})^{-1} \rVert = \lVert -\bm{A}(\bm{I} - \bm{A})^{-1} \rVert \leqslant \dfrac{\lVert \bm{A} \rVert}{1 - \lVert \bm{A} \rVert}
$$

\paragraph[]{矩阵的谱半径及其性质} \

\par 矩阵 $\bm{A} \in C^{n\times n}$ 的谱半径在特征值估计、广义逆矩阵、数值分析以及数值代数等理论的建树中, 都占有极其重要的地位.

\paragraph*{定义 2.22} 设 $\bm{A} \in C^{n\times n}$ 的 $n$ 个特征值为 $\lambda_1, \lambda_2, \cdots, \lambda_n$, 称
$$
    \rho(\bm{A}) = \mathop{\max}\limits_{i} \lvert \lambda_i \rvert
$$
为 $\bm{A}$ 的谱半径.

\paragraph*{定理 2.22} 设 $\bm{A} \in C^{n\times n}$, 则对 $C{n\times n}$ 上任何一种矩阵范数 $\lVert \bm{\cdot} \rVert$, 都有
$$
    \rho(\bm{A}) \leqslant \lVert \bm{A} \rVert
$$

\paragraph*{证} 设 $\bm{A}$ 的属于特征值 $\lambda$ 的特征向量为 $\bm{x}$, 取与矩阵范数 $\lVert \bm{\cdot} \rVert$ 相容的向量范数 $\lVert \bm{\cdot} \rVert _V$, 则
由 $\bm{Ax} = \lambda\bm{x}$, 可得
$$
    \lvert \lambda \rvert \ \lVert \bm{x} \rVert _V = \lVert \lambda\bm{x} \rVert _V = \lVert \bm{Ax} \rVert _V \leqslant \lVert \bm{A} \rVert \ \lVert \bm{x} \rVert _V
$$
因为 $\bm{x} \neq \bm{x}$, 所以 $\lvert \lambda \rvert \leqslant \lVert \bm{A} \rVert$, 从而 $\rho(\bm{A}) \leqslant \lVert \bm{A} \rVert$.

\paragraph*{例 2.15} 试用矩阵
$$
    \bm{A} = \begin{bmatrix}
        1 - j & 3     \\
        2     & 1 + j
    \end{bmatrix} \quad (j = \sqrt{-1})
$$
验证上述定义对三种常见范数的正确性.

\paragraph*{解} 因为 $\det(\lambda\bm{I} - \bm{A}) = (\lambda - 1)^2 - 5$, 所以 $\lambda_1(\bm{A}) = 1 + \sqrt{5}, \lambda_2(\bm{A}) = 1 -\sqrt{5}$, 从而
$$
    \rho(\bm{A}) = 1 + \sqrt{5}
$$
\par 又 $\lVert \bm{A} \rVert _1 = \lVert \bm{A} \rVert _\infty = 3 + \sqrt{2}$, 而
$$
    \bm{A}^H\bm{A} = \begin{bmatrix}
        6      & 5 + 5j \\
        5 - 5j & 11
    \end{bmatrix}, \quad \det(\lambda\bm{I} - \bm{A}^H\bm{A}) = \lambda^2 - 17\lambda + 16
$$
由此得 $\lambda_1(\bm{A}^H\bm{A}) = 16, \lambda_2(\bm{A}^H\bm{A}) = 1$, 则有
$$
    \lVert \bm{A} \rVert _2 = \sqrt{\lambda_1(\bm{A}^H\bm{A})} = 4
$$
可得
$$
    \rho(\bm{A}) < \lVert \bm{A} \rVert _1, \quad \rho(\bm{A}) < \lVert \bm{A} \rVert _2, \quad \rho(\bm{A}) < \lVert \bm{A} \rVert _\infty
$$

\paragraph*{例 2.16} 设 $\bm{A} \in C^{n\times n}$, 则 $\rho(\bm{A}^k) = [\rho(\bm{A})]^k(k = 1,2,\cdots)$.

\paragraph*{证} 设 $\bm{A}$ 的 $n$ 个特征值为 $\lambda_1, \lambda_2, \cdots, \lambda_n$, 可得, $\bm{A}^k$ 的 $n$ 个特征值为 $\lambda_1^k, \lambda_2^k, \cdots, \lambda_n^k$, 则有
$$
    \rho(\bm{A}^k) = \mathop{\max}\limits_i \lvert \lambda_i^k \rvert = (\mathop{\max}\limits_i \lvert \lambda_i \rvert)^k = [\rho(\bm{A})]^k
$$

\paragraph*{定理 2.23} 设 $\bm{A} \in C^{n\times n}$, 对任意的正数 $\epsilon$, 存在某种矩阵范数 $\lVert \bm{\cdot} \rVert _M$, 使得
$$
    \lVert \bm{A} \rVert _M \leqslant \rho(\bm{A}) + \epsilon
$$

\paragraph[]{矩阵的条件数} \

\par 设 $\bm{A} = (a_{ij})_{n\times n} \in C^{n\times n}$ 的元素 $a_{ij}$ 带有误差 $\delta a_{ij} (i, j = 1, 2, \cdots, n)$, 则准确矩阵应为
$\bm{A} + \delta\bm{A}$, 其中 $\delta\bm{A} = (\delta a_{ij})$. 若 $\bm{A}$ 为可逆矩阵, 其逆矩阵 $\bm{A}^{-1}$ 与 $(\bm{A} + \delta\bm{A})^{-1}$ 的近似程度(摄
动)如何?

\paragraph*{定理 2.24} 设 $\bm{A} \in C^{n\times n}$ 可逆, $\bm{B} \in C^{n\times n}$, 且对 $C^{n\times n}$ 上的某种矩阵范数 $\lVert \bm{\cdot} \rVert$, 有 $\lVert \bm{A}^{-1} \rVert < 1$,
则有以下结论:
\par (1) $\bm{A} + \bm{B}$ 可逆;
\par (2) 记 $\bm{F} = \bm{I} - (\bm{I} + \bm{A}^{-1}\bm{B})^{-1}$, 则 $\lVert \bm{F} \rVert \leqslant \dfrac{\lVert \bm{A}^{-1}\bm{B} \rVert}{1 - \lVert \bm{A}^{-1}\bm{B} \rVert}$;
\par (3) $\dfrac{\lVert \bm{A}^{-1} - (\bm{A} + \bm{B})^{-1} \rVert}{\lVert \bm{A}^{-1} \rVert} \leqslant \dfrac{\lVert \bm{A}^{-1}\bm{B} \rVert}{1 - \lVert \bm{A}^{-1}\bm{B} \rVert}$.

\par 在定理2.24中, 若令 $\mathrm{cond}(\bm{A}) = \lVert \bm{A} \rVert \ \lVert \bm{A}^{-1} \rVert, d_{\bm{A}} = \lVert \delta \bm{A} \rVert \ \lVert \bm{A} \rVert ^{-1}$, 则当 $\lVert \bm{A}^{-1} \rVert \ \lVert \delta \bm{A} \rVert < 1$
时, 由结论(2)与(3)可得
\begin{gather*}
    \lVert \bm{I} - (\bm{I} + \bm{A}^{-1}\delta\bm{A})^{-1} \rVert \leqslant \dfrac{d_{\bm{A}}\mathrm{cond}(\bm{A})}{1 - d_{\bm{A}}\mathrm{cond}(\bm{A})}
    \dfrac{\lVert \bm{A}^{-1} - (\bm{A} + \delta\bm{A})^{-1} \rVert}{\lVert \bm{A}^{-1} \rVert} \leqslant \dfrac{d_{\bm{A}}\mathrm{cond}(\bm{A})}{1 - d_{\bm{A}}\mathrm{cond}(\bm{A})}
\end{gather*}
称 $\mathrm{cond}(\bm{A})$ 为矩阵 $\bm{A}$ 的条件数, 它是一个求矩阵逆的摄动的一个重要量. 一般说来, 条
件数愈大, $(\bm{A} + \delta\bm{A})^{-1}$ 与 $\bm{A}^{-1}$ 的相对误差就愈大.

\subsection{矩阵函数介绍}

本节将要介绍矩阵分析理论. 矩阵分析理论的建立, 同数学分析一样, 也是以极限理
论为基础的, 其内容丰富, 是研究数值方法和其他数学分支以及许多工程问题的重要工
具. 本节首先讨论矩阵序列的极限运算, 然后介绍矩阵序列和矩阵级数的收敛定理、矩
阵幂级数和一些矩阵函数; 最后介绍矩阵的微分的概念及其性质.

\subsubsection{矩阵序列介绍}

\paragraph*{定义 2.23} 设有矩阵序列 $\{\bm{A}^{(k)}\}$, 其中 $\bm{A}^{(k)} = (a_{ij})_{m\times n} \in C^{m\times n}$, 当 $a_{ij}^{(k)} \to a_{ij} (k \to \infty)$ 时, 称 $\{\bm{A}^{(k)}\}$ 收敛, 或称矩阵 $\bm{A} = (a_{ij})_{m\times n}$ 为 $\{\bm{A}^{(k)}\}$ 的极限, 或称 $\{\bm{A}^{(k)}\}$ 收敛于 $\bm{A}$, 记为
$$
    \lim_{k\to \infty} \bm{A}^{(k)} = \bm{A} \qquad \bm{A}^{(k)} \to \bm{A}
$$
不收敛的矩阵序列称为发散.

\par 矩阵序列收敛的性质, 有许多与数列收敛的性质相类似.

\paragraph*{性质 1} 设 $\bm{A}^{(k)} \to \bm{A}_{m\times n}, \bm{B}^{(k)} \to \bm{B}_{m\times n}$, 则
$$
    \lim_{k \to \infty} (\alpha \bm{A}^{(k)} + \beta \bm{B}^{(k)}) = \alpha\bm{A} + \beta\bm{B} \ (\forall \alpha, \beta \in \mathbb{C})
$$

\paragraph*{性质 2} 设 $\bm{A}^{(k)} \to \bm{A}_{m\times n}, \bm{B}^{(k)} \to \bm{B}_{n\times l}$, 则
$$
    \lim_{k\to \infty} \bm{A}^{(k)}\bm{B}^{(k)} = \bm{AB}
$$

\paragraph*{性质 3} 设 $\bm{A}^{(k)}$ 与 $\bm{A}$ 都是可逆矩阵, $\bm{A}^{(k)} \to \bm{A}$, 则
$$
    (\bm{A}^{(k)})^{-1} \to \bm{A}^{-1}
$$

\paragraph*{定理 2.25} 设 $\bm{A}^{(k)} \in C^{m\times n}$, 则
\par (1) $\bm{A}^{(k)} \to \bm{O}$ 的充要条件是 $\lVert \bm{A}^{(k)} \rVert \to 0$;
\par (2) $\bm{A}^{(k)} \to \bm{A}$ 的充要条件是 $\lVert \bm{A}^{(k)} - \bm{A} \rVert \to 0$.
\\ 这里, $\lVert \bm{\cdot} \rVert$ 是 $C^{m\times n}$ 上的任何一种矩阵范数.

\paragraph*{证} (1) 由于 $C^{m\times n}$ 上的矩阵范数等价, 所以只要对矩阵范数 $\lVert \bm{\cdot} \rVert _{m_\infty}$ 证明结论成立即可.
已知 $\bm{A}^{(k)} \to \bm{O}$, 由定义可得
$$
    a_{ij}^{(k)} \to 0 \quad (i = 1,2,\cdots, m; j = 1, 2, \cdots,n)
$$
也就是 $\mathop{\max}\limits_{i, j} \lvert a_{ij}^{(k)} \rvert \to 0$, 即
$$
    \lVert \bm{A}^{(k)} \rVert _{m_\infty} = n \cdot \mathop{\max}\limits_{i, j} \lvert a_{ij}^{(k)} \rvert \to 0
$$
上述推导步步可逆, 所以结论 (1) 成立.
\par (2) 由于 $\bm{A}^{(k)} \to \bm{A}$ 等价于 $(\bm{A}^{(k)} - \bm{A}) \to \bm{O}$, 所以利用结论(1)即可得结论(2).

\paragraph*{定义 2.24} 矩阵序列 $\{\bm{A}^{(k)}\}$ 称为有界的, 如果存在常数 $M > 0$, 使得对一切 $k$ 都有
$$
    \lvert a_{ij}^{(k)} \rvert < M \ (i = 1, 2, \cdots, m; j = 1, 2, \cdots, n)
$$
有界的矩阵序列 $\{\bm{A}^{(k)}\}$, 有收敛的子序列 $\{\bm{A}^{(k_s)}\}$.

\par 在矩阵序列中, 最常见的是由一个方阵的幂构成的序列. 关于这样的矩阵序列, 有以下的概念和收敛定理.

\paragraph*{定义 2.25} 设 $\bm{A}$ 为方阵, 且 $\bm{A}^{(k)} \to \bm{O}(k \to \infty)$, 则称 $\bm{A}$ 为收敛矩阵.

\paragraph*{定理 2.26} $\bm{A}$ 为收敛矩阵的充要条件是 $\rho(\bm{A}) < 1$.

\paragraph*{证} 充分性. 已知 $\rho(\bm{A}) < 1$, 对于 $\epsilon = \dfrac{1}{2}[1 - \rho(\bm{A})] > 0$, 存在矩阵范数 $\lVert \bm{\cdot} \rVert _M$, 使得
$$
    \lVert \bm{A} \rVert _M \leqslant \rho(\bm{A}) + \epsilon = \dfrac{1}{2}[1 + \rho(\bm{A})] < 1
$$
于是有 $\lVert \bm{A}^k \rVert _M \leqslant \lVert \bm{A} \rVert ^k_M \to 0$, 可得 $\bm{A}^k \to \bm{O}$.

\par 必要性. 已知 $\bm{A}^k \to \bm{O}$, 设 $\lambda$ 是 $\bm{A}$ 的任一特征值, 对应的特征向量为 $\bm{x}$, 则有
$\bm{Ax} = \lambda\bm{x}(\bm{x} \neq \bm{0})$. 因为
$$
    \lambda^k\bm{x} = \bm{A}^k\bm{x} \to \bm{0}
$$
所以 $\lambda^k \to 0$, 从而 $\lvert \lambda \rvert < 1$, 故 $\rho(\bm{A}) < 1$.

\paragraph*{定理 2.27} $\bm{A}$ 为收敛矩阵的充分条件是只要有一种矩阵范数 $\lVert \bm{\cdot} \rVert$, 使得 $\lVert \bm{A} \rVert < 1$.

\paragraph*{例 2.17} 判断 $\bm{A} = \begin{bmatrix}
    0.1 & 0.3 \\
    0.7 & 0.6
\end{bmatrix}$ 是否为收敛矩阵.

\paragraph*{解} 因为 $\lVert \bm{A} \rVert _1 = 0.9 < 1$, 所以 $\bm{A}$ 是收敛矩阵.

\subsubsection{矩阵级数介绍}

在建立矩阵分析的理论时, 特别看重讨论矩阵级数, 特别是矩阵的幂级数, 因为它是
建立矩阵函数的理论基础. 在讨论矩阵级数时, 自然应该定义它的收敛、发散以及和的
概念. 这些都与数项级数的相应定义与性质完全类似.

\paragraph*{定义 2.26} 把定义2.23中的矩阵序列所形成的无穷和 $\bm{A}^{(0)} + \bm{A}^{(1)} + \bm{A}^{(2)} + \cdots + \bm{A}^{(k)} + \cdots$
称为矩阵级数, 记为 $\sum\limits_{k=0}^\infty \bm{A}^{(k)}$, 则有
\begin{equation}
    \sum\limits_{k=0}^\infty \bm{A}^{(k)} = \bm{A}^{(0)} + \bm{A}^{(1)} + \bm{A}^{(2)} + \cdots + \bm{A}^{(k)} + \cdots
    \tag{2.3.1}
\end{equation}

\paragraph*{定义 2.27} 记 $\bm{S}^{(N)} = \sum\limits_{k=0}^N \bm{A}^{(k)}$, 称其为上述矩阵级数式的部分和. 如果矩阵序列 $\{\bm{S}^{(N)}\}$ 收
敛, 且有极限 $\bm{S}$, 则有
\begin{equation}
    \lim\limits_{N \to \infty} \bm{S}^{(N)} = \bm{S}
    \tag{2.3.2}
\end{equation}
那么就称上述矩阵级数式收敛, 而且有和 $\bm{S}$, 记为
\begin{equation}
    \bm{S} = 、\sum\limits_{k=0}^\infty \bm{A}^{(k)}
    \tag{2.3.3}
\end{equation}
不收敛的矩阵级数称为是发散的.

\par 若用 $s_{ij}$ 表示 $\bm{S}$ 的第 $i$ 行第 $j$ 列的元素, 那么, 和 $\sum\limits_{k = 0}^\infty \bm{A}^{(k)} = \bm{S}$ 的意义指的是
\begin{equation}
    \sum\limits_{k = 0}^\infty a_{ij}^{(k)} = s_{ij} \ (i = 1, 2, \cdots, m; j = 1, 2, \cdots, n)
    \tag{2.3.4}
\end{equation}

\paragraph*{例 2.18} 研究矩阵级数 $\sum\limits_{k = 1}^\infty \bm{A}^{(k)}$ 的收敛性, 其中
$$
    \bm{A}^{(k)} = \begin{bmatrix}
        \dfrac{1}{2^k} & \dfrac{\pi}{3 \times 4^k} \\
        0              & \dfrac{1}{k(k + 1)}
    \end{bmatrix}
$$

\paragraph*{解} 因为
$$
    \bm{S}^{(N)} = \sum\limits_{k=1}^N \bm{A}^{(k)} = \begin{bmatrix}
        1 - (\dfrac{1}{2})^N & \dfrac{\pi}{9}[1 - (\dfrac{1}{4})^N] \\
        0                    & \dfrac{N}{N + 1}
    \end{bmatrix}
$$
所以
$$
    \bm{S} = \lim\limits_{N \to \infty} \bm{S}^{(N)} = \begin{bmatrix}
        1 & \dfrac{\pi}{9} \\
        0 & 1
    \end{bmatrix}
$$

\paragraph*{定义 2.28} 如果式(2.3.4)中左端 $mn$ 个数项级数都是绝对收敛的, 则称矩阵级数式
(2.3.1)是绝对收敛的.

\par 从绝对收敛的定义及数学分析中的相应结果, 立刻得到下面关于判别矩阵级数收敛
性的一些法则.

\paragraph*{性质 1} 若矩阵级数式(2.3.1)是绝对收敛的, 则它也一定收敛, 并且任意调换其项的顺
序所得的级数还是收敛的, 且其和不变.

\paragraph*{性质 2} 矩阵级数 $\sum\limits_{k=0}^\infty \bm{A}^{(k)}$ 为绝对收敛的充要条件是正项级数 $\sum\limits_{k=0}^\infty \lVert \bm{A}^{(k)} \rVert$ 收敛.

\paragraph*{性质 3} 如果 $\sum\limits_{k=0}^\infty \bm{A}^{(k)}$ 是收敛(或绝对收敛)的, 那么 $\sum\limits_{k=0}^{N} \bm{PA}^{(k)}\bm{Q}$ 也是收敛(或绝对收敛)的, 并且有
$$
    \sum\limits_{k=0}^{N} \bm{PA}^{(k)}\bm{Q} = \bm{P}(\sum\limits_{k=0}^\infty \bm{A}^{(k)}) \bm{Q}
$$

\paragraph*{性质 4} 设 $C^{m\times n}$ 中的两个矩阵级数
\begin{gather*}
    \bm{S}_1: \bm{A}^{(1)} + \bm{A}^{(2)} + \cdots + \bm{A}^{(k)} + \cdots \\
    \bm{S}_2: \bm{B}^{(1)} + \bm{B}^{(2)} + \cdots + \bm{B}^{(k)} + \cdots
\end{gather*}
都绝对收敛, 其和分别为 $\bm{A}$ 与 $\bm{B}$, 则级数 $\bm{S}_1$ 与 $\bm{S}_2$ 按项相乘所得的矩阵级数
$$
    \bm{S}_3: = \sum\limits_{k=1}^\infty (\sum\limits_{i=1}^k \bm{A}^{(i)}\bm{B}^{(k + 1 - i)})
$$
绝对收敛, 且有和 $\bm{AB}$.


\par 下面讨论矩阵幂级数, 首先从一个比较简单的方阵幂级数谈起.

\paragraph*{定理 2.28} 方阵 $\bm{A}$ 的幂级数(Neumann 级数)
$$
    \sum\limits_{k=0}^\infty \bm{A}^k = \bm{I} + \bm{A} + \bm{A}^2 + \cdots + \bm{A}^k + \cdots
$$
收敛的充要条件是 $\bm{A}$ 为收敛矩阵, 并且咋收敛时, 其和为 $(\bm{I} - \bm{A})^{-1}$.

\paragraph*{定理 2.29} 设方阵 $\bm{A}$ 对某一矩阵范数 $\lVert \bm{\cdot} \rVert$ 有 $\lVert \bm{A} \rVert < 1$, 则对任何非负整数 $N$, 以 $(\bm{I} - \bm{A})^{-1}$
为部分和 $\bm{I} + \bm{A} + \bm{A}^2 + \cdots + \bm{A}^N$ 的近似矩阵时, 其误差为
$$
    \lVert (\bm{I} - \bm{A})^{-1} - (\bm{I} + \bm{A} + \bm{A}^2 + \cdots + \bm{A}^N) \rVert \leqslant \dfrac{\lVert \bm{A} \rVert ^{N + 1}}{1 - \lVert \bm{A} \rVert}
$$

\paragraph*{定理 2.30} 设幂级数
$$
    f(z) = \sum\limits_{k = 0}^\infty c_k z^k
$$
的收敛半径为 $r$, 如果方阵 $\bm{A}$ 满足 $\rho(\bm{A}) < r$, 则矩阵幂级数
$$
    \sum\limits_{k = 0}^\infty c_k \bm{A}^k
$$
是绝对收敛的; 如果 $\rho(\bm{A}) > r$, 则矩阵幂级数是发散的.

\paragraph*{证} (1) 当 $\rho(\bm{A}) < r$ 时, 选定正数 $\epsilon$, 使满足 $\rho(\bm{A}) + \epsilon < r$. 则存在矩阵范数 $\lVert \bm{\cdot} \rVert$, 使得
$\lVert \bm{A} \rVert \leqslant \rho(\bm{A}) + \epsilon$. 从而有
$$
    \lVert c_k \bm{A}^k \rVert \leqslant \lvert c_k \rvert \ \lVert \bm{A} \rVert ^k \leqslant \lvert c_k \rvert (\rho(\bm{A}) + \epsilon)^k
$$
因为 $\rho(\bm{A}) + \epsilon < r$, 所以 $\sum\limits_{k=0}^\infty c_k(\rho(\bm{A}) + \epsilon)^k$ 绝对收敛, 从而 $\sum\limits_{k=0}^\infty \lVert c_k\bm{A}^k \rVert$ 收敛. 从而 $\sum\limits_{k = 0}^\infty c_k \bm{A}^k$ 绝
对收敛.

\par (2) 当 $\rho(\bm{A}) > r$ 时, 设 $\bm{A}$ 的 $n$ 个特征值为 $\lambda_1, \lambda_2, \cdots, \lambda_n$, 则有某个 $\lambda_l$ 满足 $\lvert \lambda_l \rvert > r$. 则存在可逆矩阵 $\bm{P}$, 使得
$$
    \bm{P}^{-1}\bm{AP} = \begin{bmatrix}
        \lambda_1 & *         & \cdots & *         \\
                  & \lambda_2 & \ddots & \vdots    \\
                  &           & \ddots & *         \\
                  &           &        & \lambda_n
    \end{bmatrix}
$$
而 $\sum\limits_{k=0}^\infty c_k \bm{B}^k$ 的对角线元素为 $\sum\limits_{k=0}^\infty c_k \lambda_i^k (i = 1, 2, \cdots, n)$. 因为 $\lvert \lambda_l \rvert > r$, 所以 $\sum\limits c_k \lambda_l^k$ 发散, 从而 $\sum\limits_{k=0}^\infty c_k \bm{B}^k$ 发散. 从而
$\sum\limits_{k = 0}^\infty c_k\bm{A}^k = \sum\limits_{k = 0}^\infty c_k\bm{PB}^k\bm{P}^{-1}$ 也发散.

\subsubsection{矩阵函数介绍}

矩阵函数的概念与通常的函数概念一样, 它是以 $n$ 阶矩阵为自变量和函数值(因
变量)的一种函数. 本节给出矩阵函数的定义及其性质.

\paragraph*{定义 2.29} 设一元函数 $f(z)$ 能够展开为 $z$ 的幂级数
$$
    f(z) = \sum\limits_{k=0}^\infty c_k z^k \ (\lvert z \rvert < r)
$$
其中 $r > 0$ 表示该幂级数的收敛半径. 当 $n$ 阶矩阵 $\bm{A}$ 的谱半径 $\rho(\bm{A}) < r$ 时, 把收敛的矩阵幂级数 $\sum\limits_{k = 0}^{\infty} c_k \bm{A}^k$ 的和称为矩阵函数, 记为 $f(\bm{A})$, 即
$$
    f(\bm{A}) = \sum\limits_{k=0}^{\infty} c_k \bm{A}^k
$$

\paragraph*{定理 2.31} 如果 $\bm{AB} = \bm{BA}$, 则 $e^{\bm{A}}e^{\bm{B}} = e^{\bm{B}}e^{\bm{A}} = e^{\bm{A} + \bm{B}}$.

\paragraph*{证} 因为矩阵加法满足交换律, 所以只需证明 $e^{\bm{A}}e^{\bm{B}} = e^{\bm{A} + B}$ 就行了.
\begin{align*}
    e^{\bm{A}}e^{\bm{B}} = & (\bm{I} + \bm{A} + \dfrac{1}{2!}\bm{A}^2 + \cdots)(\bm{I} + \bm{B} + \dfrac{1}{2!}\bm{B}^2 + \cdots) =                          \\
                           & \bm{I} + (\bm{A} + \bm{B}) + \dfrac{1}{2!}(\bm{A}^2 + \bm{AB} + \bm{BA} + \bm{B}^2) +                                           \\
                           & \dfrac{1}{3!}(\bm{A}^3 + 3\bm{A}^2\bm{B} + 3\bm{AB}^2 + \bm{B}^3) + \cdots =                                                    \\
                           & \bm{I} + (\bm{A} + \bm{B}) + \dfrac{1}{2!}(\bm{A} + \bm{B})^2 + \dfrac{1}{3!}(\bm{A} + \bm{B})^3 + \cdots = e^{\bm{A} + \bm{B}}
\end{align*}

\paragraph*{例 2.19} 设函数 $f(z) = \dfrac{1}{1 - z}(\lvert z \rvert < 1)$, 求矩阵函数 $f(\bm{A})$.

\paragraph*{解} 因为
$$
    f(z) = \dfrac{1}{1 - z} = \sum\limits_{k=0}^\infty z^k \ (\lvert z \rvert < 1)
$$
当方阵 $\bm{A}$ 的谱半径 $\rho(\bm{A}) < 1$ 时, 有
$$
    f(\bm{A}) = \sum\limits_{k = 0}^\infty \bm{A}^k
$$
从而 $f(\bm{A}) = (\bm{I} - \bm{A})^{-1}$.

\subsubsection{函数矩阵对矩阵的导数}

\paragraph*{定义 2.30} 设 $\bm{X} = (\xi_{ij})_{m\times n}$, $mn$ 元函数 $f(\bm{X}) = f(\xi_{11}, \xi_{12}, \cdots, \xi_{1n}, \xi_{21}, \cdots, \xi_{mn})$, 定义 $f(\bm{X})$ 对矩阵 $\bm{X}$ 的导数为
$$
    \dfrac{df}{d\bm{X}} = (\dfrac{\partial f}{\partial \xi_{ij}})_{m\times n} = \begin{bmatrix}
        \dfrac{\partial f}{\partial\xi_{11}} & \cdots & \dfrac{\partial f}{\partial \xi_{1n}} \\
        \vdots & & \vdots \\
        \dfrac{\partial f}{\partial\xi_{m1}} & \cdots & \dfrac{\partial f}{\partial \xi_{mn}} \\
    \end{bmatrix}
$$

\paragraph*{例 2.20} 设 $\bm{x} = (\xi_1, \xi_2, \cdots, \xi_n)^T$, $n$ 元函数 $f(\bm{x}) = f(\xi_1, \xi_2, \cdots, \xi_n)$, 求 $\dfrac{df}{d\bm{x}}$ 与 $\dfrac{df}{d\bm{x}^T}$.

\paragraph*{解} 根据定义, 有
$$
    \dfrac{df}{d\bm{x}} = (\dfrac{\partial f}{\partial \xi_1}, \dfrac{\partial f}{\partial \xi_2}, \cdots, \dfrac{\partial f}{\partial \xi_n})^T, \quad \dfrac{df}{d\bm{x}^T} = (\dfrac{\partial f}{\partial \xi_1}, \dfrac{\partial f}{\partial \xi_2}, \cdots, \dfrac{\partial f}{\partial \xi_n})
$$

\paragraph*{定义 2.31} 设 $\bm{X} = (\xi_{ij})_{m\times n}$, $mn$ 元函数 $f_{ij}(\bm{X}) = f_{ij}(\xi_{11}, \xi_{12}, \cdots, \xi_{1n}, \xi_{21}, \cdots, \xi_{mn})$, 其中$i$ 和
$j$ 有如下范围 $(i = 1,2,\cdots, r; j = 1,2,\cdots,s)$. 定义函数矩阵
$$
    \bm{F}(\bm{X}) = \begin{bmatrix}
        f_{11}(\bm{X}) & \cdots & f_{1s}(\bm{X}) \\
        \vdots & & \vdots \\
        f_{r1}(\bm{X}) & \cdots & f_{rs}(\bm{X})
    \end{bmatrix}
$$
对矩阵 $\bm{X}$ 的导数为
$$
\dfrac{d \bm{F}}{d \bm{X}} = \begin{bmatrix}
    \dfrac{\partial \bm{F}}{\partial xi_{11}} & \dfrac{\partial \bm{F}}{\partial xi_{12}} & \cdots & \dfrac{\partial \bm{F}}{\partial xi_{1n}} \\
    \dfrac{\partial \bm{F}}{\partial xi_{21}} & \dfrac{\partial \bm{F}}{\partial xi_{22}} & \cdots & \dfrac{\partial \bm{F}}{\partial xi_{2n}} \\
    \vdots                                    & \vdots                                    &        & \vdots
    \dfrac{\partial \bm{F}}{\partial xi_{m1}} & \dfrac{\partial \bm{F}}{\partial xi_{m2}} & \cdots & \dfrac{\partial \bm{F}}{\partial xi_{mn}}
\end{bmatrix}
$$
其中
$$
    \dfrac{\partial \bm{F}}{\partial \xi_{ij}} = \begin{bmatrix}
        \dfrac{\partial f_{11}}{\partial\xi_{ij}} & \dfrac{\partial f_{12}}{\partial \xi_{ij}} & \cdots & \dfrac{\partial f_{1s}}{\partial\xi_{ij}} \\
        \dfrac{\partial f_{21}}{\partial\xi_{ij}} & \dfrac{\partial f_{22}}{\partial \xi_{ij}} & \cdots & \dfrac{\partial f_{2s}}{\partial\xi_{ij}} \\
        \vdots                                    & \vdots                                     &        & \vdots                                    \\
        \dfrac{\partial f_{r1}}{\partial\xi_{ij}} & \dfrac{\partial f_{r2}}{\partial \xi_{ij}} & \cdots & \dfrac{\partial f_{rs}}{\partial\xi_{ij}}
    \end{bmatrix}
$$

\paragraph*{例 2.21} 设 $\bm{x} = (\xi_1, \xi_2, \cdots, \xi_n)^T$, $n$ 元函数 $f(\bm{x}) = f(\xi_1, \xi_2, \cdots, \xi_n)$, 求 $\dfrac{d}{d\bm{x}^T}(\dfrac{df}{d\bm{x}})$.

\paragraph*{解} 由例2.20知
$$
\dfrac{df}{d\bm{x}} = (\dfrac{\partial f}{\partial \xi_1}, \dfrac{\partial f}{\partial \xi_2}, \cdots, \dfrac{\partial f}{\partial \xi_n})^T
$$
可得
$$
    \dfrac{d}{d\bm{x}^T}(\dfrac{df}{d\bm{x}}) = \begin{bmatrix}
        \dfrac{\partial^2 f}{\partial \xi_1^2}            & \dfrac{\partial^2 f}{\partial \xi_1\partial\xi_2} & \cdots & \dfrac{\partial f}{\partial \xi_1\partial\xi_n} \\
        \dfrac{\partial^2 f}{\partial \xi_2\partial\xi_1} & \dfrac{\partial^2 f}{\partial \xi_2^2}            & \cdots & \dfrac{\partial f}{\partial \xi_2\partial\xi_n} \\
        \vdots                                            & \vdots                                            &        & \vdots                                          \\
        \dfrac{\partial^2 f}{\partial \xi_n\partial\xi_1} & \dfrac{\partial^2 f}{\partial \xi_n\partial\xi_2} & \cdots & \dfrac{\partial f}{\partial \xi_n^2}
    \end{bmatrix}
$$