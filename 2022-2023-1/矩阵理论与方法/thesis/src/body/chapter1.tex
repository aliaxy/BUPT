\pagenumbering{arabic}
\section{引言}
\subsection{背景介绍}

\subsubsection{矩阵理论与方法介绍}

\par 矩阵理论是一门研究矩阵在数学上应用的科目,其是学习数值分析、最优化理论、
概率统计、运筹学、控制理论、力学、电学、信息科学、管理科学与工程的基础, 尤其
是计算机的广泛应用,为矩阵理论的应用开辟了广阔的前景.

\par 根据世界数学发展史记载, 矩阵概念产生于19世纪50年代, 是为了解线性方程组
的需要而产生的. 然而, 在公元前我国就已经有了矩阵的萌芽. 在我国的《九章算术》一
书中已经有所描述, 只是没有将它作为一个独立的概念加以研究, 而仅用它解决实际问
题, 所以没能形成独立的矩阵理论.

\par 1850年, 英国数学家西尔维斯特 (SylveSter, 1814--1897)在研究方程的个数与未知
量的个数不相同的线性方程组时, 由于无法使用行列式, 所以引入了矩阵的概念.

\par 1855年, 英国数学家凯莱 (Caylag, 1821--1895)在研究线性变换下的不变量时, 为
了简洁、方便, 引入了矩阵的概念. 1858年, 凯莱在《矩阵论的研究报告》中, 定义了
两个矩阵相等、相加以及数与矩阵的数乘等运算和算律, 同时, 定义了零矩阵、单位
阵等特殊矩阵, 更重要的是在该文中他给出了矩阵相乘、矩阵可逆等概念, 以及利用伴随
阵求逆阵的方法, 证明了有关的算律, 如矩阵乘法有结合律, 没有交换律, 两个非零阵
乘积可以为零矩阵等结论, 定义了转置阵、对称阵、反对称阵等概念.

\par 1878年, 德国数学家弗罗伯纽斯 (Frobeniws, 1849--1917)在他的论文中引入了
$\lambda$ 矩阵的行列式因子、不变因子和初等因子等概念, 证明了两个 $\lambda$ 矩阵等价当且仅当它
们有相同的不变因子和初等因子, 同时给出了正交矩阵的定义,1879年, 他又在自己的
论文中引进矩阵秩的概念.

\par 矩阵的理论发展非常迅速, 到19世纪末, 矩阵理论体系已基本形成. 到20
世纪, 矩阵理论得到了进一步的发展. 目前, 它己经发展成为在物理、控制论、
机器人学、生物学、经济学等学科有大量应用的数学分支.


\subsubsection{函数矩阵和矩阵函数介绍}

\par 矩阵函数是以矩阵为自变量且取值为矩阵的一类函数, 它是对一元函数概念的推
广. 起先, 矩阵函数是一个收敛的矩阵幂级数的和来定义. 之后, 根据计算矩阵函数值的
Jordan 标准形方法, 对矩阵函数的概念进行了拓宽. 因此矩阵函数的基础是矩阵序列与
矩阵级数.

\par 函数矩阵是以函数为元素的矩阵. 函数矩阵的导数与积分是将通常函数的导数与积
分等概念形式上推广到矩阵的情形. 当一个矩阵的元素都是变量 $t$ 的函数时, 可以建立
矩阵对变量 $t$ 的导数与积分概念; 当一个多元函数的自变量都是矩阵 $\bm{X}$ 的元素时, 可以
建立起多元函数对矩阵 $\bm{X}$ 的导数概念; 当一个矩阵的元素都是矩阵 $\bm{X}$ 的元素的多元函
数时, 可以建立矩阵对矩阵 $\bm{X}$ 的导数概念.

\par 函数矩阵和矩阵函数是两个不同的概念. 但是, 在一些情形下, 矩阵函数在其定义
内的值可以看做函数矩阵. 比如, 矩阵 $\bm{A}$ 的指数还能输 $e^{\bm{A}t}$ 可以看作变量 $t$ 的函数矩
阵. 借助于矩阵的指数函数, 可以给出某些线性微分方程组和线性矩阵方程一般解的解
析表达式.

\subsubsection{线性代数方程组求解介绍}

线性方程组的求解问题在科学技术与经济管理领域有着广泛的应用. 线性规划问
题, 某些工程问题, 经济问题等都可以转化成线性方程组求解问题.
\par 一般 $n$ 元线性方程组是指形如
$$
    \begin{cases}
        a_{11}x_1 + a_{12}x_2 + \cdots + a_{1n}x_n = b_1 \\
        a_{21}x_1 + a_{22}x_2 + \cdots + a_{2n}x_n = b_2 \\
        \cdots
        a_{m1}x_1 + a_{m2}x_2 + \cdots + a_{mn}x_n = b_m
    \end{cases}
$$
的方程组, 其中 $x_1, x_2, \cdots, x_n$ 为未知量, $a_{ij}(i = 1,2,\cdots,m;j=1,2,\cdots,n)$ 为方程组的系
数, $m$ 为方程个数, $b_j(j=1,2,\cdots,m)$ 为方程组的常数项. 若常数项全为零, 则称其为 $n$
元齐次线性方程组; 若常数项不全为零, 则称其为 $n$ 元非齐次先线性方程组.

\par 易知齐次线性方程组一定有解, 即零解必为它的一个解, 所以求解齐次线性方程组
实际上是探究其是否有非零解. 而非齐次线性方程组的求解则更加复杂, 因为一个线性
方程组的解可能是无解, 唯一解, 无穷多个解, 所以在解非齐次线性方程组时需先对其解
进行判定再探究其解的结构.

\subsection{问题介绍}

\subsubsection{矩阵函数的求法问题介绍}

\par 矩阵函数理论时矩阵理论的一个重要组成部分. 矩阵函数把对矩阵的研究带入分析
领域. 同时也解决了数学领域及工程技术等其它领域的计算难题.

\par 矩阵函数定义的引出把矩阵理论延伸到分析, 从而使矩阵的研究又提升到一个新的
层次, 增加了新的手段, 同时也使矩阵理论在数学、物理等许多领域有了新的应用.

\par 然而, 矩阵函数的形式复杂多样, 而且计算方法和理论也不尽相同, 因此需要根据实
际问题选取不同的解法, 以达到事半功倍的效果.

\subsubsection{矩阵分解的方法问题介绍}

\par 矩阵分解对矩阵理论及近代计算数学的发展起了关键的作用. 寻求矩阵在各种意义
下的分解形式, 是对矩阵有关的数值计算和理论都有着极为重要的意义. 因为这些分解
式的特殊形式, 一是能明显的反映出原矩阵的某些特征; 二是分解的方法与过程提供了
某些有效的数值计算方法和理论分析依据. 这些分解在数值代数和最优化问题的解决中
都有着十分重要的角色以及在其它领域方面起着必不可少的作用.

\subsection{上述问题国内外研究成果介绍}

\subsubsection{矩阵函数的求法研究现状}

\par 待定系数法是以 Hamilton-Cayley 定理为基础的一种求矩阵函数的方法. 借助于 Hamilton-Cayley 定理, 可以将矩阵函数的求值问题(即矩阵幂级数的求和问题)转化为矩阵多项式的计算问题.

\par 数项级数求和法是根据最小多项式导出的矩阵递推关系来求解矩阵函数的方法.

\par 对角形法是线性代数课程中已经介绍过的求矩阵函数的方法. 其是矩阵函数中较简单的一种, 这取决于可对角化矩阵的优良性质.

\par Jordan 标准型法在形式上类似于对角形法. 借助于矩阵的 Jordan 标准形理论, 可以将矩阵函数的求值问题转化为矩阵的乘法运算问题.

\subsubsection{矩阵分解方法研究现状}

\par 对于一般的 $n$ 阶方阵而言, 前 $n-1$ 个顺序主子式不等于零是可进行三角分解的充要条件. 在此条件下, $n$ 阶方阵的 LDU 分解、Crout 分解以及 Doolittle 分解都是唯一的.

\par 矩阵的 LDU 分解可以通过它的 Crout 分解或者 Doolittle 分解构造出来, 矩阵的 Doolittle 分解也可以通过它的 Crout 分解构造出来, 反之亦然.

\par 借助于矩阵的三角分解, 可以将一般方阵的求逆计算转换为上三角矩阵和下三角矩阵的求逆计算, 也可以将一般线性代数方程组的求解问题转化为两个三角方程组的求解问题, 也就是求解线性代数方程组的追赶法.

\par 任何矩阵都可以进行 QR 分解. 方阵可以分解为正交矩阵与上三角矩阵的乘积; 当 $m > n$ 时, $m\times n$ 矩阵可以分解为列向量组标准正交的矩阵与上三角矩阵的乘积; 当 $m < n$ 时, $m\times n$ 矩阵可以分解为行向量组标准正交的矩阵与上三角矩阵的乘积.

\par 借助于矩阵的 QR 分解, 可以将一般方阵的求逆计算转化为三角矩阵的求逆计算, 也可以将一般线性代数方程组的求解问题转化为三角方程组的求解问题.

\par 任何矩阵都可以进行满秩分解和奇异值分解. 相对而言, 求矩阵的满秩分解要比求矩阵的奇异值分解容易一些, 特别是采用 Hermite 标准形方法求矩阵的满秩分解更为简单.

\par 矩阵的奇异值概念是对矩阵特征值概念的推广, 矩阵的奇异值分解是对矩阵的正交相似对角化问题的推广. 矩阵的奇异值可以给出该矩阵的值域和零空间的基, 也可以给出以该矩阵为系数矩阵的线性代数方程组的最小二乘解和具有最小2-范数的最小二乘解.

\subsection{本论文工作简述}

\subsubsection{本论文对上述问题研究简述}

\par 本论文总结了函数矩阵的各种求法以及矩阵分解的各类方法, 并举例具体说明每种
方法的优点与其局限性, 针对不同的矩阵来选择不同的方法, 以此提高效率、优化解答、
过程, 更好的应用于实际当中.

\subsubsection{本论文创新点或特点简述}

\par 本论文在分析研究学者理论的基础上, 从系统化、全局化的角
度入手结合了多个角度,全面的对矩阵函数的求解和矩阵的分解进行了分析.

\par 经过查阅大量资料, 对于该课题的研究大多数并没有给出理论依据, 仅提供了方法,
本论文更加注重方法的理论建设, 知其然并知其所以然, 更好地能够得到应用于实践中
的结果.


\subsubsection{本论文撰写结构简述}

\par 本论文分为四部分:
\par 第一部分主要介绍研究上述问题的基石. 主要包括线性空间与线性变换、向量范数与
矩阵范数以及矩阵函数的介绍;
\par 第二部分是对矩阵函数求法展开的研究. 探讨了待定系数法、数项级数求和法、对
角型法和若尔当标准型法求解矩阵函数的步骤推导, 并具体展示了具体求解过程;
\par 第三部分是对矩阵分解方法展开的研究. 探讨了 LU 分解、QR 分解、满秩分解和
奇异值分解的步骤推导, 并具体展示了具体求解过程;
\par 第四部分是对本论文工作的总结.
























