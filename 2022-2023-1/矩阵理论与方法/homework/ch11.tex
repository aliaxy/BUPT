\documentclass[12pt, a4paper, oneside, fontset=none]{ctexart}
\usepackage{amsmath, amsthm, amssymb, graphicx, color, fontspec, float, pgfplots}
\usepackage{bm}
\usepackage[bookmarks=true, colorlinks, citecolor=blue, linkcolor=black]{hyperref}
\pgfplotsset{compat=1.16}
\usepackage{xeCJK, CJKnumb}
\xeCJKsetup{CJKmath=true,CheckSingle=true}
\setCJKmainfont[ItalicFont=KaiTi]{微软雅黑}
\usepackage{geometry}
\geometry{left=1.8cm, right=1.8cm, top=2.18cm, bottom=2.18cm}
\author{}
\date{}
\linespread{1.25}
\title{\vspace{-3em}\textbf{矩阵论 \quad 第十一次作业}\vspace{-3em}}

\begin{document}

\maketitle

\section*{第3章 \quad 矩阵分析及其应用}

\subsection*{3.1 \quad 矩阵序列}

\centerline{\large{\textbf{定义}}} \ \par

\paragraph*{定义 3.1} 设有矩阵序列$\{\bm{A}^{(k)}\}$,其中$\bm{A}^{(k)} = (a_{ij}^(k))_{m\times n} \in \bm{C}^{m\times n}$,当$a_{ij}^{(k)} \to a_{ij}(k\to \infty)$时,
称$\{\bm{A}^(k)\}$收敛,或称矩阵$\bm{A} = (a_{ij})_{m\times n}$为$\{\bm{A}^(k)\}$的极限,或称$\{\bm{A}^(k)\}$收敛于$\bm{A}$,记为
\[
    \lim_{k\to \infty}\bm{A}^{(k)} = \bm{A} \quad 或 \quad \bm{A}^{(k)} \to \bm{A}
\]
不收敛的矩阵序列称为\textbf{发散}.

\paragraph*{定义 3.2} 矩阵序列$\{\bm{A}^{(k)}\}$称为\textbf{有界}的,如果存在常数$M > 0$,使得对一切$k$都有
\[
    |a_{ij}^{(k)}| < M\quad (i = 1,2,\cdots,m;j = 1,2,\cdots,n)
\]

\paragraph*{定义 3.3} 设$\bm{A}$为方阵,且$\bm{A}^k \to \bm{O}(k \to \infty)$,则称$\bm{A}$为\textbf{收敛矩阵}.

\par \ \par

\centerline{\large{\textbf{定理}}} \ \par

\paragraph*{定理 3.1} 设$\bm{A}^{(k)} \in \bm{C}^{m\times n}$,则
\begin{enumerate}
    \item[(1)] $\bm{A}^{(k)} \to \bm{O}$的充要条件是$\left\lVert \bm{A}^{(k)} \right\rVert \to 0$.
    \item[(2)] $\bm{A}^{(k)} \to \bm{A}$的充要条件是$\left\lVert \bm{A}^{(k)} - \bm{A} \right\rVert \to 0$.
\end{enumerate}

\paragraph*{定理 3.2} $\bm{A}$为收敛矩阵的充要条件是$\rho(\bm{A}) < 1$.

\paragraph*{定理 3.3} $\bm{A}$为收敛矩阵的充分条件是只要有一种矩阵范数$\left\lVert \bm{\cdot}\right\rVert$,使得$\left\lVert \bm{A}\right\rVert < 1$.

\par \ \par

\centerline{\large{\textbf{例题}}} \ \par

\paragraph*{例 3.1} 判断$\bm{A} = \begin{bmatrix}
        0.1 & 0.3 \\
        0.7 & 0.6
    \end{bmatrix} 是否为收敛矩阵$.

\paragraph*{解} 因为$\left\lVert \bm{A}\right\rVert _1 = 0.9 < 1$,所以$\bm{A}$是收敛矩阵.

\par \ \par

\centerline{\large{\textbf{习题}}} \ \par

\paragraph*{习题 3.1.2} 设$\bm{A} = \begin{bmatrix}
        0 & c & c \\
        c & 0 & c \\
        c & c & 0
    \end{bmatrix}\ (c \in R)$,讨论$c$取何值时$\bm{A}$为收敛矩阵.

\paragraph*{解}
\[
    |\lambda E - \bm{A}| = \begin{bmatrix}
        \lambda & -c      & -c      \\
        -c      & \lambda & -c      \\
        -c      & -c      & \lambda
    \end{bmatrix} = (\lambda + c)^2(\lambda - 2c)
\]
所以$\bm{A}$的特征值为$\lambda_1 = 2c,\lambda_2 = \lambda_3 = -c$,于是$r(\bm{A}) = 2|c|$\\
根据$r(\bm{A}) < 1$得到,$-\dfrac{1}{2} < c < \dfrac{1}{2}$.

\subsection*{3.2 \quad 范数的一些应用}

\centerline{\large{\textbf{定义}}} \ \par

\paragraph*{定义 3.4} 把定义3.1中的矩阵序列所形成的无穷和$\bm{A}^{(0)} + \bm{A}^{(1)} + \bm{A}^{(2)} + \cdots + \bm{A}^{(k)} + \cdots$称为
\textbf{矩阵级数},记为$\sum_{k=0}^\infty \bm{A}^{(k)}$,则有
\[
    \sum_{k=0}^\infty \bm{A}^{(k)} = \bm{A}^{(0)} + \bm{A}^{(1)} + \bm{A}^{(2)} + \cdots + \bm{A}^{(k)} + \cdots
\]

\paragraph*{定义 3.5} 记$\bm{S}^{(N)} = \sum_{k = 0}^N \bm{A}^{(k)}$,称其为矩阵级数式的\textbf{部分和}.如果矩阵序列$\{\bm{S}^{(N)}\}$收敛,且有极限$\bm{S}$,则有
\[
    \lim_{N\to \infty} \bm{S}^{(N)} = \bm{S}
\]
那么就成矩阵级数式\textbf{收敛},而且有\textbf{和}$\bm{S}$,记为
\[
    \bm{S} = \sum_{k = 0}^\infty \bm{A}^{(k)}
\]
不收敛的矩阵级数称为是\textbf{发散}的. \\
若用$s_{ij}$表示$\bm{S}$的第$i$行第$j$列的元素,那么,和$\sum_{k=0}^\infty \bm{A}^{(k)} = \bm{S}$的意义指的是
\[
    \sum_{k=0}^\infty a_{ij}^{(k)} = s_{ij} \quad (i = 1,2,\cdots,m;j = 1,2,\cdots,n)
\]

\paragraph*{定义 3.6} 如果左端$mn$个数项数都是绝对收敛的,则称矩阵级数式是\textbf{绝对收敛}的.

\par \ \par

\centerline{\large{\textbf{定理}}} \ \par

\paragraph*{定理 3.5} 设方阵$\bm{A}$对某一矩阵范数$\left\lVert \bm{\cdot}\right\rVert$有$\left\lVert \bm{A}\right\rVert < 1$,则对任何非负整数$N$,以$(\bm{I} - \bm{A})^{-1}$为
部分和的$\bm{I} + \bm{A} + \bm{A}^2 + \cdot + \bm{A}^N$的近似矩阵时,其误差为
\[
    \left\lVert (\bm{I} - \bm{A})^{-1} - (\bm{I} + \bm{A} + \bm{A}^2 + \cdot + \bm{A}^N) \right\rVert \leqslant \frac{\left\lVert \bm{A}\right\rVert ^{N + 1}}{1 - \left\lVert \bm{A}\right\rVert}
\]

\paragraph*{定理 3.6} 设幂级数
\[
    f(z) = \sum_{k=0}^\infty c_kz^k
\]
的收敛半径为$r$,如果方阵$\bm{A}$满足$\rho(\bm{A}) < r$,则矩阵幂级数
\[
    \sum_{k = 0}^\infty c_k \bm{A}^k
\]

\par \ \par

\centerline{\large{\textbf{例题}}} \ \par

\paragraph*{例 3.2} 研究矩阵级数$\sum_{k = 1}^\infty \bm{A}^{(k)}$的收敛性,其中
\[
    \bm{A}^{(k)} = \begin{bmatrix}
        \frac{1}{2^k} & \frac{\pi}{3 \times 4^k} \\
        0             & \frac{1}{k(k + 1)}
    \end{bmatrix}\quad (k = 1,2,\cdots)
\]

\paragraph*{解} 因为
\[
    \bm{S}^{(N)} = \sum_{k = 1}^N \bm{A}^{(k)} = \begin{bmatrix}
        \sum_{k=1}^N \frac{1}{2^k} & \sum_{k=1}^N\frac{\pi}{3 \times 4^k} \\
        0                          & \sum_{k=1}^N\frac{1}{k(k + 1)}
    \end{bmatrix} = \begin{bmatrix}
        1 - (\frac{1}{2})^N & \frac{\pi}{9}[1 - (\frac{1}{4})^N] \\
        0                   & \frac{N}{N + 1}
    \end{bmatrix}
\]
所以
\[
    \bm{S} = \lim_{N\to \infty} \bm{S}^{(N)} = \begin{bmatrix}
        1 & \frac{\pi}{9} \\
        0 & 1
    \end{bmatrix}
\]
因此,所给级数收敛.

\par \ \par

\centerline{\large{\textbf{习题}}} \ \par

\paragraph*{习题 3.2.1} 问矩阵幂级数$\sum_{k = 1}^\infty \bm{A}^k$收敛还是发散,其原因是什么?其中
\[
    \bm{A} = \begin{bmatrix}
        -1 & 0 & 1 \\
        1  & 1 & 0 \\
        -4 & 0 & 3
    \end{bmatrix}
\]

\paragraph*{解}
\[
    |\lambda \bm{E} - \bm{A}| = \begin{bmatrix}
        \lambda + 1 & 0           & -1          \\
        -1          & \lambda - 1 & 0           \\
        4           & 0           & \lambda - 3
    \end{bmatrix} = (\lambda - 1)^3
\]
所以,$\rho(\bm{A}) = 1$,因此,发散.

\paragraph*{习题 3.2.2} 设幂级数$\sum_{k=1}^\infty c_kz^k$的收敛半径是3,3阶方阵$\bm{A}$的谱半径也是3,问矩阵幂级数$\sum_{k = 1}^\infty c_k \bm{A}^k$是否有可能收敛.

\paragraph*{解} 有可能收敛,但不是绝对收敛.

\paragraph*{习题 3.2.4} 设$\bm{A}^{(k)} \in \bm{C}^{m \times n}$,且矩阵级数$\sum_{k=0}^\infty \bm{A}^{(k)}$收敛,证明$\lim_{k\to \infty} = \bm{O}$.

\paragraph*{解} 记$\bm{S}^{(N)} = \sum_{k=0}^N \bm{A}^{(k)}$,已知$\sum_{k=0}^\infty \bm{A}^{(k)}$收敛,可设
\[
    \lim_{N \to \infty} \bm{S}^{(N)} = \bm{S}
\]
于是有,$\lim_{N\to \infty} \bm{A}^{(N)} = \lim_{N\to \infty} (\bm{S}^{(N)} - \bm{S}^{(N - 1)}) = \bm{O}$


\end{document}