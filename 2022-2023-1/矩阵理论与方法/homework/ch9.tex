\documentclass[12pt, a4paper, oneside, fontset=none]{ctexart}
\usepackage{amsmath, amsthm, amssymb, graphicx, color, fontspec, float, pgfplots}
\usepackage{bm}
\usepackage[bookmarks=true, colorlinks, citecolor=blue, linkcolor=black]{hyperref}
\pgfplotsset{compat=1.16}
\usepackage{xeCJK, CJKnumb}
\xeCJKsetup{CJKmath=true,CheckSingle=true}
\setCJKmainfont[ItalicFont=KaiTi]{微软雅黑}
\usepackage{geometry}
\geometry{left=1.8cm, right=1.8cm, top=2.18cm, bottom=2.18cm}
\author{}
\date{}
\linespread{1.25}
\title{\vspace{-3em}\textbf{矩阵论 \quad 第九次作业}\vspace{-3em}}

\begin{document}

\maketitle

\section*{第2章 \quad 范数理论及其应用}

\subsection*{2.1 \quad 向量范数及其性质}

\centerline{\large{\textbf{定义}}} \ \par

\paragraph*{定义 2.1} 如果$V$是数域$K$上的线性空间,对任意的$\bm{x}\in V$,定义一个实值函数$\left\lVert \bm{x}\right\rVert $,它满足
以下三个条件:\par
(1) 非负性:当$\bm{x} \neq \bm{0}$时,$\left\lVert \bm{x} \right\rVert > 0$;当$\bm{x} = \bm{0}$时,$\left\lVert \bm{x} \right\rVert = 0$;\par
(2) 齐次性:$\left\lVert a\bm{x} \right\rVert = \left\lvert a\right\rvert \left\lVert \bm{x} \right\rVert\ (a\in K,\, \bm{x} \in V)$ \par
(3) 三角不等式:$\left\lVert \bm{x} + \bm{y} \right\rVert \leqslant \left\lVert \bm{x} \right\rVert + \left\lVert \bm{y} \right\rVert\ (\bm{x},\bm{y} \in V)$ \\
则称$\left\lVert \bm{x} \right\rVert$为$V$上向量$\bm{x}$的范数,简称\textbf{向量范数}.

\paragraph*{定义 2.2} 满足
\[
    c_1\left\lVert \bm{x} \right\rVert _\beta \leqslant \left\lVert \bm{x} \right\rVert _\alpha \leqslant c_2 \left\lVert \bm{x} \right\rVert _\beta \qquad (\forall \bm{x} \in V)
\]
的范数是等价的.即有限维空间上的不同范数是等价的.

\par \ \par

\centerline{\large{\textbf{定理}}} \ \par

\paragraph*{定理 2.1} 设$\left\lVert \bm{x}\right\rVert _\alpha$和$\left\lVert \bm{x} \right\rVert _\beta$为有限维线性空间$V$上的任意两种向量范数(他们不限于$p$-范数),
则存在两个与向量$\bm{x}$无关的正常数$c_1$和$c_2$,使满足
\[
    c_1\left\lVert \bm{x} \right\rVert _\beta \leqslant \left\lVert \bm{x} \right\rVert _\alpha \leqslant c_2 \left\lVert \bm{x} \right\rVert _\beta \qquad (\forall \bm{x} \in V)
\]

\paragraph*{定理 2.2} $C^n$中的向量序列
\[
    \bm{x}^{(k)} = (\xi_1^{(k)},\xi_2^{(k)},\cdots,\xi_n^{(k)}) \quad (k=1,2,3,\cdots)
\]
收敛到向量$\bm{x} = (\xi_1,\xi_2,\cdots,\xi_n)$的充要条件是对任意一种向量范数$\left\lVert \bm{\cdot} \right\rVert$,数列$\{\left\lVert \bm{x}^{(k)} - \bm{x} \right\rVert\}$ 收
敛于零.

\centerline{\large{\textbf{例题}}} \ \par

\paragraph*{例 2.1} 在$n$维酉空间$C^n$上,复向量$\bm{x} = (\xi_1,\xi_2,\cdots,\xi_n)$的长度
\[
    \left\lVert \bm{x} \right\rVert = \sqrt{\left\lvert \xi_1 \right\rvert^2 + \left\lvert \xi_2 \right\rvert^2 + \cdots +\left\lvert \xi_n \right\rvert^2}
\]
就是一种范数. \par

\paragraph*{解} 为了说明这里的$\left\lvert \bm{x} \right\rvert$时范数,只需验证它满足范数的三个条件
\begin{enumerate}
    \item[(1)]
        当$\bm{x} \neq \bm{0}$时,显然$\left\lVert \bm{x} \right\rVert > 0$;当$\bm{x} = \bm{0}$时,有$\left\lVert \bm{x} \right\rVert = 0$.
    \item[(2)] 对任意的复数$a$,因为
        \[
            a\bm{x} = (a\xi_1, a\xi_2, \cdots, a\xi_n)
        \]
        所以
        \[
            \left\lVert a\bm{x} \right\rVert = \left\lvert a \right\rvert \sqrt{\left\lvert \xi_1 \right\rvert^2 + \left\lvert \xi_2 \right\rvert^2 + \cdots +\left\lvert \xi_n \right\rvert^2} =
            \left\lvert a\right\rvert \left\lVert \bm{x} \right\rVert
        \]
    \item[(3)] 对于任意两个复向量$\bm{x} = (\xi_1,\xi_2,\cdots,\xi_n),\ \bm{y} = (\eta_1, \eta_2, \cdots, \eta_n)$,有
        \[
            \bm{x} + \bm{y} = (\xi_1 + \eta_1, \xi_2 + \eta_2,\cdots,\xi_n+\eta_n)
        \]
        可得
        \begin{gather*}
            \left\lVert \bm{x} + \bm{y} \right\rVert\ = \sqrt{\left\lvert \xi_1 + \eta_1 \right\rvert^2 + \left\lvert \xi_2 + \eta_2 \right\rvert^2 + \cdots +\left\lvert \xi_n + \eta_n \right\rvert^2} \\
            \left\lVert \bm{x} + \bm{y} \right\rVert^2 = (\bm{x} + \bm{y}, \bm{x} + \bm{y}) = (\bm{x}, \bm{x}) + 2Re(\bm{x},\bm{y}) + (\bm{y}, \bm{y})
        \end{gather*}
        因为
        \[
            Re(\bm{x}, \bm{y}) \leqslant \left\lvert (\bm{x}, \bm{y}) \right\rvert \leqslant = \sqrt{(\bm{x},\bm{x}), (\bm{y}, \bm{y})} = \left\lVert \bm{x} \right\rVert \left\lVert \bm{y} \right\rVert\
        \]
        所以
        \[
            \left\lVert \bm{x} + \bm{y} \right\rVert \leqslant \left\lVert \bm{x} \right\rVert + 2\left\lVert \bm{x}\right\rVert\left\lVert \bm{y} \right\rVert = (\left\lVert \bm{x} + \bm{y} \right\rVert)^2
        \]
        即 $\left\lVert \bm{x} + \bm{y} \right\rVert\leqslant \left\lVert \bm{x}\right\rVert\left\lVert \bm{y} \right\rVert$.
\end{enumerate}

\paragraph*{例 2.2} 证明$\left\lVert \bm{x} + \bm{y} \right\rVert = \max\left\lVert \xi_i \right\rvert$是$C^n$上的一种范数,这里$\bm{x} = (\xi_1,\xi_2,\cdots,\xi_n) \in C^n$.

\paragraph*{证} 当$\bm{x} \neq 0$时,有 $\left\lVert \bm{x} \right\rVert = \max_i\left\lvert \xi_i \right\rvert > 0$;当$\bm{x} = \bm{0}$时,显然有$\left\lVert \bm{x} \right\rVert = 0$. \\
又对任意的$a\in C$,有
\[
    \left\lVert a\bm{x} \right\rVert = \max_i\left\lvert a\xi_i \right\rvert = \left\lvert a \right\rvert \max_i \left\lvert \xi_i \right\rvert = \left\lvert a \right\rvert \left\lVert \bm{x} \right\rVert
\]
对$C^n$的任意两个向量$\bm{x} = (\xi_1,\xi_2,\cdots,\xi_n),\ \bm{y} = (\eta_1,\eta_2,\cdots,\eta_n)$,有
\begin{align*}
    \left\lVert \bm{x} + \bm{y} \right\rVert = & \max_i \left\lvert \xi_i + \eta_i \right\rvert \leqslant                                                                                          \\
                                               & \max_i \left\lvert \xi_i \right\rvert + \max_i\left\lvert \eta_i \right\rvert = \left\lVert \bm{x} \right\rVert + \left\lVert \bm{y} \right\rVert
\end{align*}
因此,$\left\lVert \bm{x}\right\rVert = \max_i \left\lvert \xi_i \right\rvert $是$C^n$上的一种范数.

\paragraph*{例 2.3} 证明$\left\lVert \bm{x}\right\rVert = \sum_{i=1}^n\left\lvert \xi_i \right\rvert $是$C^n$上的一种范数,其中$\bm{x} = (\xi_1,\xi_2,\cdots,\xi_n) \in C^n$.

\paragraph*{证} 当$\bm{x} \neq \bm{0}$时,显然$\left\lVert \bm{x} \right\rVert = \sum_{i=1}^n \left\lvert \xi_i \right\rvert > 0$;当$\bm{x} = \bm{0}$时,有$\left\lVert \bm{x} \right\rVert = 0$.\par
又对于任意$a\in C$,有
\[
    \left\lVert a\bm{x} \right\rVert = \left\lvert a \right\rvert \sum_{i=1}^n\left\lvert \xi_i \right\rvert = \left\lvert a \right\rvert \left\lVert \bm{x}\right\rVert
\]
对于任意两个向量$\bm{x},\bm{y}\in C^n$,有
\begin{align*}
    \left\lVert \bm{x} + \bm{y} \right\rVert = & \sum_{i=1}^n\left\lvert \xi_i + \eta_i \right\rvert \leqslant \sum_{i=1}^n(\left\lvert \xi_i \right\rvert + \left\lvert \eta \right\rvert ) = \\
                                               & \sum_{i=1}^n\left\lvert \xi_i \right\rvert + \sum_{i=1}^n\left\lvert \eta_i \right\rvert = \left\lVert \bm{x} + \bm{y} \right\rVert
\end{align*}
于是$\left\lVert \bm{x} \right\rVert = \sum_{i=1}^n \left\lvert \xi_i \right\rvert $是$C^n$上的一种范数.

\par \ \par

\centerline{\large{\textbf{习题}}} \ \par

\paragraph*{习题 2.1.1} 求向量$\bm{e} = (1,1,\cdots,1)$的$l_1,l_2$及$l_\infty$范数.

\paragraph*{解}
\begin{enumerate}
    \item[(1)] $\left\lVert \bm{e} \right\rVert _1 = \sum_{i=1}^n\left\lvert 1 \right\rvert = n$;
    \item[(2)] $\left\lVert \bm{e} \right\rVert _2 = \sqrt{\left\lvert 1 \right\rvert^2 + \left\lvert 1 \right\rvert^2 + \cdots + \left\lvert 1 \right\rvert^2} = \sqrt{n}$;
    \item[(3)] $\left\lVert \bm{e} \right\rVert _\infty = \max_i \left\lvert 1 \right\rvert = 1$.
\end{enumerate}

\subsection*{2.1 \quad 矩阵的范数}

\centerline{\large{\textbf{定义}}} \ \par

\paragraph*{定义 2.3} 设$\bm{A} = \bm{C}^{m\times n}$,定义一个实值函数$\left\lVert \bm{A} \right\rVert$,它满足以下三个条件:\par
(1) 非负性:当$\bm{A} \neq \bm{O}$时,$\left\lVert \bm{A} \right\rVert > 0$;当$\bm{A} = \bm{O}$时,$\left\lVert \bm{A} \right\rVert = 0$; \par
(2) 齐次性:$\left\lVert a\bm{A} \right\rVert = \left\lvert a \right\rvert \left\lVert \bm{A} \right\rVert \quad (a \in C)$;\par
(3) 三角不等式:$\left\lVert \bm{A} + \bm{B} \right\rVert \leqslant \left\lVert \bm{A} \right\rVert + \left\lVert \bm{B} \right\rVert \quad (\bm{B} \in \bm{C}^{m\times n})$
则称$\left\lVert \bm{A} \right\rVert$为$\bm{A}$的\textbf{广义矩阵范数}.若对$\bm{C}^{m\times n},\bm{C}^{n\times l}$及$\bm{C}^{m\times l}$
上的同类广义矩阵范数$\left\lVert \bm{\cdot} \right\rVert$,还满足下面一个条件:\par
(4) 相容性:$\left\lVert \bm{AB} \right\rVert \leqslant \left\lVert \bm{A} \right\rVert \left\lVert \bm{B} \right\rVert \qquad (\bm{B} \in \bm{C}^{n\times l})$
则称$\left\lVert \bm{A} \right\rVert$为$\bm{A}$的\textbf{矩阵范数}.

\paragraph*{定义 2.4} 对于$C^{m\times n}$上的矩阵范数$\left\lVert \bm{\cdot} \right\rVert _M$和$\bm{C}^m$与$\bm{C}^n$上的同类向量范数$\left\lVert \bm{\cdot} \right\rVert _V$,如果
\[
    \left\lVert \bm{Ax} \right\rVert _V \leqslant \left\lVert \bm{A} \right\rVert _M \left\lVert \bm{x} \right\rVert _V \qquad (\forall \bm{A} \in C^{m\times n},\ \forall bm{x} \in C^n)
\]
则称矩阵范数$\left\lVert \bm{\cdot} \right\rVert _M$与向量范数$\left\lVert \bm{\cdot} \right\rVert _V$是相容的.

\par \ \par

\centerline{\large{\textbf{定理}}} \ \par

\paragraph*{定理 2.4} 已知$C^m$和$C^n$上的同类向量范数$\left\lVert \bm{\cdot} \right\rVert$,设 $\bm{A} \in \bm{C}^{m\times n}$,则函数
\[
    \left\lVert \bm{A} \right\rVert = \max_{\left\lVert \bm{x} \right\rVert = 1} \left\lVert \bm{Ax} \right\rVert
\]
是$\bm{C}^{m\times n}$上的矩阵范数,且与已知的向量范数相容.

\paragraph*{定理 2.5} 设 $\bm{A} = (a_{ij})_{m\times n} \in \bm{C}^{m\times n},\ \bm{x} = (\xi_1,\xi_2,\cdots,\xi_n)^T \in \bm{C}^n$,则从属于向量$\bm{x}$的三种范数
$\left\lVert \bm{x} \right\rVert _1, \left\lVert \bm{x} \right\rVert _2, \left\lVert \bm{x} \right\rVert _\infty$的矩阵范数计算公式依次为 \par
(1) $\left\lVert \bm{A} \right\rVert _1 = \max_j \sum_{i=1}^m \left\lvert a_{ij} \right\rvert $; \par
(2) $\left\lVert \bm{A} \right\rVert _2 = \sqrt{\lambda_1}$,$\lambda_1$为$\bm{A}^H\bm{A}$的最大特征值;\par
(3) $\left\lVert \bm{A} \right\rVert _\infty = \max_i \sum_{j=1}^n \left\lvert a_{ij} \right\rvert $ \par
通常称$\left\lVert \bm{A} \right\rVert _1, \left\lVert \bm{A} \right\rVert _2$及$\left\lVert \bm{A} \right\rVert _\infty$ 依次为\textbf{列和范数、谱范数}及\textbf{行和范数}.

\par \ \par

\centerline{\large{\textbf{例题}}} \ \par

\paragraph*{例 2.8} 设$\bm{A} = (a_{ij})_{m\times n} \in C^{m\times n}$,证明函数
\[
    \left\lVert \bm{A} \right\rVert _F = (\sum_{i=1}^m \sum_{j=1}^n \left\lvert a_{ij} \right\rvert)^\frac{1}{2} = (tr(\bm{A}^H\bm{A}))^\frac{1}{2}
\]
是$C^{m\times n}$上的矩阵范数,且与向量范数$\left\lVert \bm{\cdot} \right\rVert _2$相容.

\paragraph*{证} 显然$\left\lVert \bm{A} \right\rVert _F$具有非负性与齐次性.设$\bm{B} \in \bm{C}^{m\times n}$,且$\bm{A}$的第$j$列分别为$\bm{a}_j,\bm{b}_j(j =
    1,2,\cdots,n)$,则有
\begin{align*}
    \left\lVert \bm{A} + \bm{B} \right\rVert ^2_F = & \left\lVert \bm{a}_1 + \bm{b}_1 \right\rVert ^2_2 + \cdots + \left\lVert \bm{a}_n + \bm{b}_n \right\rVert ^2_2 \leqslant                                                   \\
                                                    & (\left\lVert \bm{a_1} \right\rVert _2 + \left\lVert \bm{b}_1 \right\rVert _2) + \cdots + (\left\lVert \bm{a_n} \right\rVert _2 + \left\lVert \bm{b}_n \right\rVert _2)^2 = \\
                                                    & (\left\lVert \bm{a_1} \right\rVert ^2_2 + \cdots + \left\lVert \bm{a}_n \right\rVert ^2_2) +                                                                               \\
                                                    & 2(\left\lVert \bm{a_1} \right\rVert _2\left\lVert \bm{b_1} \right\rVert _2 + \cdots + \left\lVert \bm{a_n} \right\rVert _2 \left\lVert \bm{b_n} \right\rVert _2)           \\
                                                    & (\left\lVert \bm{b_1} \right\rVert ^2_2 + \cdots + \left\lVert \bm{b}_n \right\rVert ^2_2)
\end{align*}
可得
\[
    \left\lVert \bm{A} + \bm{B} \right\rVert _F^2 \leqslant \left\lVert \bm{A} \right\rVert ^2_F + 2\left\lVert \bm{A} \right\rVert _F \left\lVert \bm{B} \right\rVert _F + \left\lVert \bm{B} \right\rVert ^2_F = (\left\lVert \bm{A} \right\rVert _F + \left\lVert \bm{B} \right\rVert _F)^2
\]
即三角不等式成立.\par
再设$\bm{B} = (b_{ij}){n\times l} \in \bm{C}^{n\times l}$,则$\bm{AB} = (\sum_{k=1}^na_{ik}b_{kj})_{m\times l} \in \bm{C}^{m \times l}$,于是有
\[
    \left\lVert \bm{AB} \right\rVert ^2_F = \sum_{i=1}^m\sum_{j=1}^l \left\lvert \sum_{k=1}^n a_{ik}b_{kj} \right\rvert ^2 \leqslant \sum_{i=1}^m\sum_{j=1}^l(\sum_{k=1}^n \left\lvert a_{ik} \right\rvert\left\lvert b_{kj} \right\rvert)^2
\]
可得
\begin{align*}
    \left\lVert \bm{AB} \right\rVert ^2_F \leqslant & = \sum_{i=1}^m\sum_{j=1}^l [(\sum_{k=1}^n \left\lvert a_{ik} \right\rvert ^2)(\sum_{k=1}^n \left\lvert b_{kj} \right\rvert ^2)] =                                                                    \\
                                                    & (\sum_{i=1}^m\sum_{k=1}^n\left\lvert a_{ik} \right\rvert ^2)(\sum_{j=1}^l\sum_{k=1}^n \left\lvert b_{kj} \right\rvert ^2) = \left\lVert \bm{A} \right\rVert ^2_F\left\lVert \bm{B} \right\rVert ^2_F
\end{align*}
即$\left\lVert \bm{A} \right\rVert _F$是$\bm{A}$的矩阵范数.\par
取$\bm{B} = \bm{x} \in \bm{C}^{n\times l}$,则有
\[
    \left\lVert \bm{Ax} \right\rVert _2 = \left\lVert \bm{AB} \right\rVert _F \leqslant \left\lVert \bm{A} \right\rVert _F \left\lVert \bm{B} \right\rVert _F =
    \left\lVert \bm{A} \right\rVert _F \left\lVert \bm{x} \right\rVert _2
\]
即矩阵范数$\left\lVert \bm{\cdot} \right\rVert _F$ 与向量范数$\left\lVert \bm{\cdot} \right\rVert _2$相容.

\paragraph*{例 2.9} 设$\left\lVert \bm{\cdot} \right\rVert _M$ 是矩阵范数,任取$C^n$中的非零列向量$\bm{y}$,则函数
\[
    \left\lVert \bm{x} \right\rVert _V = \left\lVert \bm{xy}^H \right\rVert _M \qquad (\forall \bm{x} \in \bm{C}^n)
\]
是$\bm{C}^n$上的向量范数,且矩阵范数$\left\lVert \bm{\cdot} \right\rVert _M$ 与向量范数$\left\lVert \bm{\cdot} \right\rVert _V$相容.

\paragraph*{证}
非负性.当$\bm{x} \neq \bm{0}$时,$\bm{xy}^H \neq \bm{O}$,从而$\left\lVert \bm{x} \right\rVert _V > 0$;当$\bm{x} = \bm{0}$时,$\bm{xy}^H = \bm{O}$,从而$\left\lVert \bm{x} \right\rVert _V = 0$.\par
齐次性.对任意$\bm{x}_1,\bm{x}_2 \in \bm{C}^n$,有
\[
    \left\lVert k\bm{x} \right\rVert _V = \left\lVert k\bm{xy}^H \right\rVert _M = \left\lvert k \right\rvert \left\lVert \bm{xy}^H \right\rVert _M = \left\lVert k\bm{xy}^H \right\rVert _M = \left\lvert k \right\rvert \left\lVert \bm{x} \right\rVert _V
\] \par
三角不等式.对任意$\bm{x}_1,\bm{x}_2 \in \bm{C}^n$,有
\begin{align*}
    \left\lVert \bm{x}_1 + \bm{x}_2 \right\rVert _V  = & \left\lVert (\bm{x}_1 + \bm{x}_2)\bm{y}^H \right\rVert _M = \left\lVert \bm{x}_1\bm{y}^H + \bm{x}_2\bm{y}^H \right\rVert _M \leqslant                                     \\
                                                       & \left\lVert \bm{x}_1\bm{y}^H \right\rVert _M + \left\lVert \bm{x}_2\bm{y}^H \right\rVert _M = \left\lVert \bm{x}_1 \right\rVert _V + \left\lVert \bm{x}_2 \right\rVert _V
\end{align*}
因此,$\left\lVert \bm{x} \right\rVert _V$是$\bm{C}^n$上的向量范数.当$\bm{A} \in \bm{C}^{n\times n},\ \bm{x} \in \bm{C}^n$时,有
\begin{align*}
    \left\lVert \bm{Ax} \right\rVert _V = & \left\lVert (\bm{Ax})\bm{y}^H \right\rVert _M = \left\lVert \bm{x}_1\bm{y}^H + \bm{x}_2\bm{y}^H \right\rVert _M \leqslant                   \\
                                          & \left\lVert A \right\rVert _M \left\lVert \bm{xy}^H \right\rVert _M = \left\lVert \bm{A} \right\rVert _M \left\lVert \bm{x} \right\rVert _V
\end{align*}
即矩阵范数$\left\lVert \bm{\cdot} \right\rVert _M$与向量范数$\left\lVert \bm{\cdot} \right\rVert _V$相容.

\par \ \par

\centerline{\large{\textbf{习题}}} \ \par

\paragraph*{习题 2.2.1} 求矩阵
$\bm{A} = \begin{bmatrix}
        -1 & 2 & 1
    \end{bmatrix}$和
$\bm{B} = \begin{bmatrix}
        -j & 2 & 2 \\
        1  & 0 & j
    \end{bmatrix}$
的$\left\lVert \bm{\cdot} \right\rVert _ 1,\ \left\lVert \bm{\cdot} \right\rVert _\infty$及$\left\lVert \bm{\cdot} \right\rVert _2$.

\paragraph*{解}
\begin{enumerate}
    \item[(1)]
        $\left\lVert \bm{A} \right\rVert _1 = \max_j\sum_{i=1}^m \left\lvert a_{ij} \right\rvert = 2$; \\
        $\left\lVert \bm{B} \right\rVert _1 = \max_j\sum_{i=1}^m \left\lvert b_{ij} \right\rvert = 4$;
    \item[(2)]
        $\left\lVert \bm{A} \right\rVert _\infty = \max_i\sum_{i=j}^m \left\lvert a_{ij} \right\rvert = 4$; \\
        $\left\lVert \bm{B} \right\rVert _\infty = \max_i\sum_{i=j}^m \left\lvert b_{ij} \right\rvert = 6$;
    \item[(3)]
        $\left\lVert \bm{A} \right\rVert _2 = \sqrt{\lambda_1} = \sqrt{6}$; \\
        $\left\lVert \bm{B} \right\rVert _2 = \sqrt{\lambda_1} = \sqrt{8 + 2\sqrt{13}}$;
\end{enumerate}

\end{document}