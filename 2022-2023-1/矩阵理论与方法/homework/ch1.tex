\documentclass[12pt, a4paper, oneside, fontset=none]{ctexart}
\usepackage{amsmath, amsthm, amssymb, graphicx, color, fontspec, float, pgfplots}
\usepackage{bm}
\usepackage[bookmarks=true, colorlinks, citecolor=blue, linkcolor=black]{hyperref}
\pgfplotsset{compat=1.16}
\usepackage{xeCJK, CJKnumb}
\xeCJKsetup{CJKmath=true,CheckSingle=true}
\setCJKmainfont[ItalicFont=KaiTi]{微软雅黑}
\usepackage{geometry}
\geometry{left=1.8cm, right=1.8cm, top=2.18cm, bottom=2.18cm}
\author{}
\date{}
\linespread{1.25}
\title{\vspace{-3em}\textbf{矩阵论 \quad 第一次作业}\vspace{-3em}}
% \title{\textbf{矩阵论 \quad 第一次作业}}

\begin{document}

\maketitle

\section*{第1章 \quad 线性空间和线性变换}

\subsection*{1.1 \quad 线性空间}

\centerline{\large{\textbf{定义}}} \ \par

\textbf{定义 1.1} \quad 设 $V$ 是一个非空集合,它的元素用 \bm{$x,y,z$} 等表示,并称之为向量;$K$是一个
数域,它的元素用 $k,l,m$ 等表示.如果 $V$ 满足条件:\par
(1)在 $V$ 中定义一个加法运算,即当 \bm{$x,y$}$\in V$ 时,有唯一的和 \bm{$x + y$}$\in V$ 且加法
运算满足以下性质:\par
1) 结合律 \bm{$x + (y + z) = (x + y) + z$}; \par
2) 交换律 \bm{$x + y = y + x$}; \par
3) 存在\textbf{零元素 0},使 \bm{$x + 0 = x$}; \par
4) 存在\textbf{负元素}, 即对任一向量 \bm{$x$}$\in V$,存在向量$ \bm{y} \in V$,使 $\bm{x + y = 0}$,则称$\bm{y}$ 为 $\bm{x}$ 的负
元素,记为$\bm{-x}$ ,于是有 $\bm{x + (- x) = 0}$. \par
(2) 在 $V$ 中定义数乘(数与向量的乘法)运算,即当 \bm{$x$}$\in V$,$k \in K$ 时,有唯一的乘积
$k\bm{x} \in V$,且数乘运算满足以下性质: \par
5) 数因子分配律 $k(\bm{x + y} = k \bm{x} + k \bm{y})$; \par
6) 分配律 $(k + l) \bm{x} = k \bm{x} + l \bm{x}$; \par
7) 结合律 $k(l \bm{x}) = (kl) \bm{x}$; \par
8) $1 \cdot \bm{x} = \bm{x}$ \\
则称 $V$ 为数域 $K$ 上的\textbf{线性空间}或\textbf{向量空间}. \par

$V$中所定义的加法及数乘运算统称为 $V$ 的线性运算.在不致产生混淆时,将数域 $K$ 上的
线性空间简称为线性空间.数 $k$ 与向量 $\bm{x}$ 的乘积$k\bm{x}$也可写成$\bm{x}k$. \par

需要指出,不管$V$的元素如何,当$K$为实数域$\mathbb{R}$时,则称$V$为实线性空间;当$K$为复
数域$\mathbb{C}$时,就称$V$为复线性空间.

\textbf{定义 1.3} \quad 设 $V$ 是数域$K$上的线性空间,$\bm{x}_1\bm{,x}_2\bm{, \cdots, x}_r(r \geqslant 1)$是属于$V$的任意$r$个向
量,如果它满足 \\
(1)$\bm{x}_1\bm{,x}_2\bm{, \cdots, x}_r$ 线性无关; \par
(2)$V$ 中任一向量 $\bm{x}$ 都是$\bm{x}_1\bm{,x}_2\bm{, \cdots, x}_r$的线性组合. \\
则称$\bm{x_1,x_2, \cdots, x_r}$ 为 $V$ 的一个\textbf{基}或\textbf{基底},并称$\bm{x}_i(i = 1,2,\cdots,r)$为\textbf{基向量}.

\textbf{定义 1.4} \quad 称线性空间 $V^n$ 的一个基$\bm{x}_1\bm{,x}_2\bm{, \cdots, x}_n$为$V^n$的一个\textbf{坐标系}.设向量 $\bm{x} \in V^n$,
它在该基下的线性表示式为 \\
\centerline{$\bm{x = \xi_1x_1 + \xi_2x_2 + \cdots + \xi_nx_n}$}
则称$\bm{\xi}_1\bm{,\xi}_2\bm{, \cdots, \xi}_n$ 为 $\bm{x}$在该坐标系中的\textbf{坐标}或\textbf{分量},记为 \\
\centerline{$(\xi_1, \xi_2 ,\bm{\cdots},\xi_n)^T$} \par \ \par

\centerline{\large{\textbf{例题}}} \ \par

\textbf{例 1.3} \quad 在集合$P_n$中,按照通常意义定义多项式加法及实数与多项式乘法,则$P_n$对
这两种运算是封闭的,因为,如果$f(t)\in P_n$,$g(t) \in P_n$,则$f(t) + g(t) \in P_n$;若$k \in \mathbb{R}$,则
$kf(t)\in P_n$,易验证对$P_n$的这两种运算,也满足交换律与结合律.

\textbf{例 1.4} \quad 在所有$n$阶实矩阵的集合$\mathbf{R}^{n\times n}$(或复矩阵的集合$\mathbf{C}^{n\times n}$中,如果 $\mathbf{A} + \mathbf{B} \in \mathbf{R}^{n \times n}$
(或$\mathbf{C}^{n \times n}$),则$\mathbf{A + B} \in \mathbf{R}^{n \times n}$(或$\mathbf{C}^{n \times n}$);如果 $k \in \mathbb{R}$ 或 $\mathbb{C}$,则$k \mathbf{A} \in \mathbf{R}^{n \times n}$ (或 $\mathbf{C}^{n \times n}$).即集
合对于这两种运算是封闭的.加法与数乘矩阵也都满足诸算律.

\textbf{例 1.5} \quad 设$\mathbb{R^{+}}$为所有正实数组成的数集,其加法与数乘运算分别定义为 \\
\centerline{$m \oplus n = mn$,\quad $k \circ m = m^k $}
证明$\mathbb{R^{+}}$是$\mathbb{R}$是上的线性空间. \par

\textbf{证} \quad 设$m,n\in \mathbb{R^+}, \quad k \in R$,则有 \\
\centerline{$m \oplus n = mn \in \mathbb{R^+}$,\quad $k \circ m = m^k \in \mathbb{R^+}$}
即$\mathbb{R^+}$对所定义的加法运算“$\oplus$”与数乘运算“$\circ$”是封闭的,且有 \par

(1)$(m \oplus n) \oplus p = (mn) \oplus p = mnp = m \oplus (np) = $ \\
\centerline{$m \oplus (n \oplus p)$} \par
(2)$m \oplus n = mn = nm = n \oplus m$ \par
(3)1是零元素,因为$m \oplus 1 = m \times 1 = m$ \par \ \par
(4)$m$的负元素是$\dfrac{1}{m}$,因为$m \oplus \dfrac{1}{m} = m \oplus \dfrac{1}{m} = 1$ \par \ \par
(5)$k \circ (m \oplus n) = k \circ (mn) = (mn)^k = $ \\
\centerline{$m^kn^k = (k \circ m) \oplus (k \circ n)$} \par
(6)$(k + l) \circ m = m^{k + l} = m^km^l = (k \circ m) \oplus (l \circ m)$ \par
(7)$k \circ (l \circ m) = k \circ m^l = m^{lk} = m^{kl} = (kl) \circ m$ \par
(8)$1 \circ m = m^1 = m$ \\
成立,故$\mathbb{R^+}$是实线性空间. \par \ \par

\centerline{\large{\textbf{习题}}} \ \par

\textbf{习题 1.1.6} \quad 求$P_2$中向量$1+t+t^2$对基:$1, t - 1, (t - 2)(t - 1)$的坐标.

\textbf{解}
\begin{align*}
    1 + t + t^2 & = a \cdot 1 + b(t - 1) + c(t - 1)(t - 2) \\
    1 + t + t^2 & = (a - b + 2c) + (b - 3c)t + ct^2        \\
                & \begin{cases}
                      a - b + 2c = 1 \\
                      b - 3c = 1     \\
                      c = 1
                  \end{cases} \Rightarrow
    \begin{cases}
        a = 3 \\
        b = 4 \\
        c = 1
    \end{cases}
\end{align*}
因此,向量$1+t+t^2$对基:$1, t - 1, (t - 2)(t - 1)$的坐标为$(3, 4, 1)^T$.

\textbf{习题 1.1.11} \quad 求$\mathbf{R}^4$的两个子空间
$$
    \begin{cases}
        V_1 = \{(\xi_1, \xi_2, \xi_3, \xi_4) \ | \ \xi_1 - \xi_2 + \xi_3 - \xi_4 = 0 \} \\
        V_2 = \{(\xi_1, \xi_2, \xi_3, \xi_4) \ | \ \xi_1 + \xi_2 + \xi_3 + \xi_4 = 0 \} \\
    \end{cases}
$$
的交$V_1 \cap V_2$的基.

\textbf{解}
\begin{align*}
    A = \begin{bmatrix}
            1 & -1 & 1 & -1 \\
            1 & 1  & 1 & 1
        \end{bmatrix} \rightarrow
    \begin{bmatrix}
        1 & 0 & 1 & 0 \\
        0 & 1 & 0 & 1
    \end{bmatrix} \quad
    r(A) = 2,\ n - r(A) = 2
\end{align*}
因此,$V_1 \cap V_2$基的维数是2.
$$
    \begin{cases}
        \xi_1 + \xi_3 = 0 \\
        \xi_2 + \xi_4 = 0
    \end{cases} \rightarrow
    (1, 0, -1, 0)^T ,\quad  (0, 1, 0, -1)^T
$$
所以,$V_1 \cap V_2$的基为$\{(1, 0, -1, 0)^T ,\quad  (0, 1, 0, -1)^T\}$


\end{document}