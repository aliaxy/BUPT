\documentclass[12pt, a4paper, oneside, fontset=none]{ctexart}
\usepackage{amsmath, amsthm, amssymb, graphicx, color, fontspec, float, pgfplots}
\usepackage{bm}
\usepackage[bookmarks=true, colorlinks, citecolor=blue, linkcolor=black]{hyperref}
\pgfplotsset{compat=1.16}
\usepackage{xeCJK, CJKnumb}
\xeCJKsetup{CJKmath=true,CheckSingle=true}
\setCJKmainfont[ItalicFont=KaiTi]{微软雅黑}
\usepackage{geometry}
\geometry{left=1.8cm, right=1.8cm, top=2.18cm, bottom=2.18cm}
\author{}
\date{}
\linespread{1.25}
\title{\vspace{-3em}\textbf{矩阵论 \quad 第二次作业}\vspace{-3em}}

\begin{document}

\maketitle

\section*{第1章 \quad 线性空间和线性变换}

\subsection*{1.2 \quad 线性变换及其矩阵}

\centerline{\large{\textbf{定义}}} \ \par

\textbf{定义 1.10} \quad 设$V$是数域$K$上的线性空间,$T$是$V$到自身的一个映射,使对于任意向量$\bm{x} \in V$,
$V$中都有唯一的向量$\bm{y}$与之对应,则称$T$是$V$的一个\textbf{变换}或\textbf{算子},记为$T\bm{x = y}$,称
$\bm{y}$为$\bm{x}$在$T$下的\textbf{象},而$\bm{x}$是$\bm{y}$的\textbf{原象}(或\textbf{象源}).

\textbf{定义 1.11} \quad 如果数域$K$上的线性空间$V$的一个变换$T$具有以下性质:
$$
    T(k\bm{x} + l\bm{y}) = k(T\bm{x}) + l(Y\bm{y})
$$
其中$\bm{x,y} \in V$,$k, l \in K$,则称$T$为$V$的一个\textbf{线性变换}或\textbf{线性算子}. \par\ \par

\centerline{\large{\textbf{例题}}} \ \par

\textbf{例 1.10} \quad 把线性空间$R^2$的所有向量均绕原点依顺(或逆)时针方向旋转$\theta$角的变换,就
是一个线性变换.这是象$(\eta_1,\ \eta_2)$与原象$(\xi_1 ,\ \xi_2)$之间的关系为
$$
    \begin{bmatrix}
        \eta_1 \\
        \eta_2
    \end{bmatrix} = \begin{bmatrix}
        \cos\theta  & \sin\theta \\
        -\sin\theta & \cos\theta
    \end{bmatrix} \begin{bmatrix}
        \xi_1 \\
        \xi_2
    \end{bmatrix}
$$

\textbf{例 1.11} \quad 在线性空间$P_n$中,求微分是其一个线性变换,这里用$D$表示,即
$$
    Df(t) = f^{'}t \qquad (\forall f(t) \in P_n)
$$ \par
事实上,对任意的$f(t), g(t) \in P_n$及$k,l \in R$,有
$$
    D(kf(t) + lg(t)) = (kf(t) + lg(t))^{'} =
    kf^{'}(t) + lg^{'}(t) = k(Df(t)) + l(Dg(t))
$$

\textbf{例 1.12} \quad 定义在闭区间$[a,\ b]$上的所有实连续函数的集合$(C(a, \ b))$构成$R$上的一个线性
空间,在$C(a,\ b)$上定义变换$J$,即
$$
    J(f(t)) = \int_a^t f(t)\, du \qquad (\forall f(t) \in C(a, \ b))
$$
则$J$是$C(a, \ b)$的一个线性变换. \par \ \par

\centerline{\large{\textbf{习题}}} \ \par

\textbf{习题 1.1.10} \quad 假定$\bm{x}_1, \bm{x}_2, \bm{x}_3$是$R^3$的一个基,试求由 \\
\centerline{$\bm{y}_1 = \bm{x}_1 - 2\bm{x}_2 + 3\bm{x}_3,\quad \bm{y}_2 = 2\bm{x}_1 - 3\bm{x}_2 + 2\bm{x}_3,\quad \bm{y}_3 = 4\bm{x}_1 + 13\bm{x}_2$}
生成的子空间$L(\bm{y}_1, \bm{y}_2, \bm{y}_3)$的基.

\textbf{解} \quad
\begin{align*}
    A = \begin{bmatrix}
            1  & -2 & 3 \\
            2  & 3  & 2 \\
            -4 & 13 & 0
        \end{bmatrix} = \begin{bmatrix}
                            1 & -2 & 3 \\
                            2 & 3  & 2 \\
                            0 & 0  & 0
                        \end{bmatrix}
    r(A) = 2
\end{align*}
所以,基的维数是2,且$\bm{y}_1$与$\bm{y}_2$线性无关,\\
故生成子空间$L(\bm{y}_1, \bm{y}_2, \bm{y}_3)$的基为$\{\bm{y}_1, \bm{y}_2\}$.

\textbf{习题 1.1.12} \quad 给定$R^{2\times 2} = \{A = (a_{ij})_{2\times 2} \ | \ a_{ij} \in R\}$(数域$R$上的2阶实方阵按通常矩阵的
加法与数乘矩阵构成的线性空间)的子集 \par
\centerline{$V = \{A = (a_{ij})_{2\times 2} \ | \ a_{ij} \in R$且$a_{11}+a_{22}=0\}$}
(1)证明$V$是$R^{2\times 2}$的子空间; \par
(2)求$V$的维数和一个基.

\textbf{解} \par
(1)设$A = (a_{ij})_{2 \times 2} \in V$,$B = (b_{ij})_{2 \times 2} \in V$,则有 \\
\centerline{$a_{11} + a_{22} = 0$,$b_{11}+b_{22} = 0$}
所以
\begin{align*}
    A + B & = (a_{ij})_{2\times 2} + (b_{ij})_{2\times 2} \\
          & = (a_{ij} + b_{ij})_{2\times 2}
\end{align*}
即
\begin{align*}
    a_{11} + a_{22} + b_{11} + b_{22} = 0 \\
    \Rightarrow (a_{11} + b_{11}) + (a_{22} + b_{22}) = 0
\end{align*}
又
\begin{align*}
    kA = k(a_{ij})_{2\times 2} & = (ka_{ij})_{2 \times 2} \\
                               & = ka_{11} + ka_{22}      \\
                               & = k(a_{11} + a_{22}) = 0
\end{align*}
$$\Rightarrow A + B \in V, \ kA \in V$$
因此,$V$是$R^{2\times 2}$的子空间. \par
(2)设$V$中,
$$
    A_1 = \begin{bmatrix}
        1 & 0  \\
        0 & -1
    \end{bmatrix} \qquad A_2 = \begin{bmatrix}
        0 & 1 \\
        0 & 0
    \end{bmatrix} \qquad A_3 = \begin{bmatrix}
        0 & 0 \\
        1 & 0
    \end{bmatrix}
$$
线性无关.对于任意$A = (a_{ij})_{2 \times 2} \in A$,\\
有$a_{11} + a_{22} = 0$,即$a_{22} = -a_{11}$, \\
所以 $A = a_{11}A_1 + a_{12}A_2 + a_{21}A_3$ \\
因此,$V$的维数是3,一组基为$\{A_1, A_2, A_3\}$ \par \ \par

\textbf{习题 1.2.1} \quad 判别下列哪些是线性变换:\par
(1)在$R^3$中,设$\bm{x} = (\bm{\xi}_1, \bm{\xi}_2, \bm{\xi}_3)$,$T\bm{x} = (\bm{\xi}_1^2, \bm{xi}_1 + \bm{\xi}_2, \bm{\xi}_3)$; \par
(2)在矩阵空间$R^{n \times n}$中,$T\mathbf{X} = \mathbf{BXC}$,这里$\mathbf{B,C}$是给定矩阵;\par
(3)在线性空间$P_n$中,$Tf(t) = f(t + 1)$.

\textbf{解} \quad
(1)不是
(2)是
(3)是



\end{document}