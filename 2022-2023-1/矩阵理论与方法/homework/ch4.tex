\documentclass[12pt, a4paper, oneside, fontset=none]{ctexart}
\usepackage{amsmath, amsthm, amssymb, graphicx, color, fontspec, float, pgfplots}
\usepackage{bm}
\usepackage[bookmarks=true, colorlinks, citecolor=blue, linkcolor=black]{hyperref}
\pgfplotsset{compat=1.16}
\usepackage{xeCJK, CJKnumb}
\xeCJKsetup{CJKmath=true,CheckSingle=true}
\setCJKmainfont[ItalicFont=KaiTi]{微软雅黑}
\usepackage{geometry}
\geometry{left=1.8cm, right=1.8cm, top=2.18cm, bottom=2.18cm}
\author{}
\date{}
\linespread{1.25}
\title{\vspace{-3em}\textbf{矩阵论\quad 第四次作业}\vspace{-3em}}

\begin{document}

\maketitle

\section*{第1章 \quad 线性空间和线性变换}

\subsection*{1.1 \quad 线性空间}

\centerline{\large{\textbf{习题}}} \ \par

\paragraph*{1.1.8} 设$\mathbf{R}^4$中两个基为 \\
(\uppercase\expandafter{\romannumeral1}): $\bm{x}_1 = \bm{e}_1 \bm{,}\bm{x}_2 = \bm{e}_2\bm{,}\bm{x}_3 = \bm{e}_3\bm{,}\bm{x}_4 = \bm{e}_4\bm{;}$ \\
(\uppercase\expandafter{\romannumeral2}): $\bm{y}_1 = (2,1,-1,1) \bm{,}\bm{y}_2 = (0,3,1,0)\bm{,}\bm{y}_3 = (5,3,2,1)\bm{,}\bm{y}_4 = (6,6,1,3)\bm{;}$
\begin{enumerate}
    \item[(1)] 求由基(\uppercase\expandafter{\romannumeral1})改变为(\uppercase\expandafter{\romannumeral2})的过渡矩阵;
    \item[(2)] 求向量$\bm{x} = (\xi_1, \xi_2, \xi_3, \xi_4)$对基(\uppercase\expandafter{\romannumeral2})的坐标;
    \item[(3)] 求对两个基由相同坐标的非零向量.
\end{enumerate}

\paragraph*{解}
\begin{enumerate}
    \item[(1)] 设过渡矩阵为$T$,则
        \[
            Y = XT \rightarrow T = X^{-1}Y
        \]
        由题意,有
        \[
            X = X^{-1} = \begin{bmatrix}
                1 & 0 & 0 & 0 \\
                0 & 1 & 0 & 0 \\
                0 & 0 & 1 & 0 \\
                0 & 0 & 0 & 1
            \end{bmatrix}, \quad Y = \begin{bmatrix}
                2  & 0 & 5 & 3 \\
                1  & 3 & 3 & 6 \\
                -1 & 1 & 2 & 1 \\
                1  & 0 & 1 & 3
            \end{bmatrix}
        \]
        得
        \[
            T = X^{-1}Y = \begin{bmatrix}
                2  & 0 & 5 & 6 \\
                1  & 3 & 3 & 6 \\
                -1 & 1 & 2 & 1 \\
                1  & 0 & 1 & 3
            \end{bmatrix}
        \]
    \item[(2)]
        \[
            \bm{x} = X(\xi_1, \xi_2, \xi_3,\xi_4)^T = YT^{-1}(\xi_1, \xi_2, \xi_3,\xi_4)^T
        \]
        又
        \[
            T^{-1} = \begin{bmatrix}
                \frac{4}{9}   & \frac{1}{3}  & -1           & -\frac{11}{9}  \\
                \frac{1}{27}  & \frac{4}{9}  & -\frac{1}{3} & -\frac{23}{27} \\
                \frac{1}{3}   & 0            & 0            & -\frac{2}{3}   \\
                -\frac{7}{27} & -\frac{1}{9} & \frac{1}{3}  & \frac{26}{27}
            \end{bmatrix}
        \]
        所以
        \[
            X = \begin{bmatrix}
                \dfrac{4}{9}x_1 + \dfrac{1}{3}x_2 -x_3 - \dfrac{11}{9}x_4          \\
                \dfrac{1}{27}x_1 + \dfrac{4}{9}x_2 - \dfrac{1}{3} - \dfrac{23}{27} \\
                \dfrac{1}{3}x_1 - \dfrac{2}{3}_4                                   \\
                -\dfrac{7}{27}x_1 - \dfrac{1}{9}x_2 + \dfrac{1}{3}x_3 + \dfrac{26}{27}x_4
            \end{bmatrix}^T
        \]
    \item[(3)] 设$\eta = (k_1, k_2, k_3, k_4)$
        \[
            \eta = X \begin{bmatrix}
                k_1 \\
                k_2 \\
                k_3 \\
                k_4
            \end{bmatrix} = Y \begin{bmatrix}
                k_1 \\
                k_2 \\
                k_3 \\
                k_4
            \end{bmatrix} = XT \begin{bmatrix}
                k_1 \\
                k_2 \\
                k_3 \\
                k_4
            \end{bmatrix}
        \]
        因此,
        \[
            (T - E) \begin{bmatrix}
                k_1 \\
                k_2 \\
                k_3 \\
                k_4
            \end{bmatrix} = \bm{0}, \quad T - E = \begin{bmatrix}
                1  & 0 & 5 & 6 \\
                1  & 2 & 3 & 6 \\
                -1 & 1 & 1 & 1 \\
                1  & 0 & 1 & 2
            \end{bmatrix}
        \]
        得
        \[
            \begin{cases}
                k_1 + 5k_3 + 6k_4 = 0        \\
                k_1 + 2k_2 + 3k_3 + 6k_4 = 0 \\
                k_1 - k_2 - k_3 - k_4 = 0    \\
                k_1 + k_3 + 2k_4 = 0
            \end{cases}
        \]
        解得$\eta = k(1,1,1,-1)$,$k$为任意实数.
\end{enumerate}

\paragraph*{1.1.9} 设线性空间$V$中的向量组$\bm{x}_1,\bm{x}_2,\bm{\cdots},\bm{x}_m$与向量组$\bm{y}_1,\bm{y}_2,\bm{\cdots},\bm{y}_m$满足关系式
\[
    (\bm{y}_1,\bm{y}_2,\bm{\cdots},\bm{y}_m) = (\bm{x}_1,\bm{x}_2,\bm{\cdots},\bm{x}_m) \bm{P}
\]
其中$\bm{P}$是$m$阶矩阵,证明:若以下三个条件
\begin{enumerate}
    \item[(a)] 向量组$\bm{x}_1,\bm{x}_2,\bm{\cdots},\bm{x}_m$线性无关;
    \item[(b)] 向量组$\bm{y}_1,\bm{y}_2,\bm{\cdots},\bm{y}_m$线性无关;
    \item[(c)] 矩阵$\bm{P}$可逆.
\end{enumerate}
中的任意两个成立时,其余的一个也成立.

\paragraph*{证}
\begin{enumerate}
    \item[(1)] 若(a)(b)成立\\
        由于$\bm{y}_1,\bm{y}_2,\bm{\cdots},\bm{y}_m$线性无关,则
        \[
            r(P) = r(\bm{x}_1,\bm{x}_2,\bm{\cdots},\bm{x}_m)
        \]
        又因为$\bm{x}_1,\bm{x}_2,\bm{\cdots},\bm{x}_m$线性无关,所以
        \[
            r(P) = r(\bm{x}_1,\bm{x}_2,\bm{\cdots},\bm{x}_m) = n
        \]
        因此,矩阵$P$可逆.
    \item[(2)] 若(a)(c)成立,有
        \[
            \begin{vmatrix}
                P
            \end{vmatrix} \neq 0, \quad \begin{vmatrix}
                (\bm{x}_1,\bm{x}_2,\bm{\cdots},\bm{x}_m)
            \end{vmatrix} \neq 0
        \]
        则
        \[
            \begin{vmatrix}
                (\bm{y}_1,\bm{y}_2,\bm{\cdots},\bm{y}_m)
            \end{vmatrix} = \begin{vmatrix}
                P
            \end{vmatrix} \begin{vmatrix}
                (\bm{x}_1,\bm{x}_2,\bm{\cdots},\bm{x}_m)
            \end{vmatrix} \neq 0
        \]
        因此,向量组$\bm{y}_1,\bm{y}_2,\bm{\cdots},\bm{y}_m$线性无关.
    \item[(3)] 同理可证,若(b)(c)成立,题意也成立.
\end{enumerate}

\subsection*{1.2 \quad 线性变换及其矩阵}

\centerline{\large{\textbf{定义}}} \ \par

\paragraph*{1.16} 设$T$是数域$K$上的线性空间$V^n$的线性变换,且对$K$中某一数$\lambda_0$,存在非零向量
$\bm{x} \in V^n$,使得
\[
    T\bm{x} = \lambda_0 \bm{x} \tag{1.2.18} \label{1}
\]
成立,则称$\lambda_0$为$T$的\textbf{特征值},$\bm{x}$为$T$的属于$\lambda_0$的\textbf{特征向量}.

\[
    (\lambda_0 \bm{I} - \bm{A})\begin{bmatrix}
        \xi_1  \\
        \xi_2  \\
        \vdots \\
        \xi_n
    \end{bmatrix} = \bm{0} \tag{1.2.20}
\]

\paragraph*{1.17} 设$\bm{A} = (a_{ij}{m\times n})$是数域$K$上的$n$阶矩阵,$\lambda$是参数,$\bm{A}$的\textbf{特征矩阵}$\lambda \bm{I} - \bm{A}$的行列式
\[
    \det(\lambda \bm{I} - \bm{A}) = \begin{vmatrix}
        \lambda - a_{11} & -a_{12}          & \cdots & -a_{1n}          \\
        -a_{21}          & \lambda - a_{22} & \cdot  & -a_{2n}          \\
        \vdots           & \vdots           &        & \vdots           \\
        -a_{n1}          & -a_{n2}          & \cdots & \lambda - a_{nn}
    \end{vmatrix}
    \tag{1.2.22} \label{2}
\]
称为矩阵$\bm{A}$的\textbf{特征多项式},它是$K$上的一个$n$次多项式,记为$\varphi(\lambda)$,$\varphi(\lambda)$的根(或零点)$\lambda_0$
称为$\bm{A}$\textbf{特征值(根)};而相应于方程组\eqref{2}的非零解向量$(\xi_1, \xi_2, \cdots , \xi_n)^T$称为$\bm{A}$\textbf{的属于特征值$\lambda_0$的特征向量}.

\centerline{\large{\textbf{例题}}} \ \par

\paragraph*{1.7} 在$\mathbf{R}^n$中,已知向量$\bm{x}$在基$\bm{e}_1,\bm{e}_2, \cdots ,\bm{e}_n$下的坐标为$(\xi_1, \xi_2, \cdots , \xi_n)^T$,求当该基改变为
\[
    \left.
    \begin{matrix}
        \bm{y}_1 = (1,1,\cdots,1,1) \\
        \bm{y}_2 = (0,1,\cdots,1,1) \\
        \cdots \cdots               \\
        \bm{y}_n = (0,0,\cdots,0,1)
    \end{matrix}
    \right\}
\]
时,向量$\bm{x}$在新基下的坐标$(\eta_1, \eta_2, \cdots , \eta_n)^T$.

\paragraph*{解} 由题意,得
\[
    (\bm{y}_1, \bm{y}_2, \cdots, \bm{y}_n) = (\bm{e}_1, \bm{e}_2, \cdots, \bm{e}_n) \begin{bmatrix}
        1      \quad      & 0           & \bm{\cdots} & 0           \\
        1      \quad      & 1           & \bm{\cdots} & 0           \\
        \bm{\vdots} \quad & \bm{\vdots} &             & \bm{\vdots} \\
        1      \quad      & 1           & \bm{\cdots} & 1           \\
    \end{bmatrix}
\]
于是过渡矩阵为
\[
    \bm{C} = \begin{bmatrix}
        1           & 0           & \bm{\cdots} & 0           \\
        1           & 1           & \bm{\cdots} & 0           \\
        \bm{\vdots} & \bm{\vdots} &             & \bm{\vdots} \\
        1           & 1           & \bm{\cdots} & 1
    \end{bmatrix}
\]
不难求得
\[
    \bm{C}^{-1} = \begin{bmatrix}
        1           & 0           & 0           & \bm{\cdots} & 0           & 0           \\
        -1          & 1           & 0           & \bm{\cdots} & 0           & 0           \\
        0           & -1          & 1           & \bm{\cdots} & 0           & 0           \\
        \bm{\vdots} & \bm{\vdots} & \bm{\vdots} &             & \bm{\vdots} & \bm{\vdots} \\
        0           & 0           & 0           & \bm{\cdots} & -1          & 1
    \end{bmatrix}
\]
由$\beta = \bm{C}^{-1}\alpha$得$\bm{x}$在新基$\bm{y}_1,\bm{y}_2,\cdots,\bm{y}_n$下的坐标为
\[
    \begin{bmatrix}
        \eta_1      \\
        \eta_2      \\
        \bm{\vdots} \\
        \eta_n
    \end{bmatrix} = \begin{bmatrix}
        1           & 0           & 0           & \bm{\cdots} & 0           & 0           \\
        -1          & 1           & 0           & \bm{\cdots} & 0           & 0           \\
        0           & -1          & 1           & \bm{\cdots} & 0           & 0           \\
        \bm{\vdots} & \bm{\vdots} & \bm{\vdots} &             & \bm{\vdots} & \bm{\vdots} \\
        0           & 0           & 0           & \bm{\cdots} & -1          & 1
    \end{bmatrix} \begin{bmatrix}
        \xi_1       \\
        \xi_2       \\
        \bm{\vdots} \\
        \xi_n
    \end{bmatrix}
\]
也就是
\[
    \begin{cases}
        \eta_1 = \xi_1 \\
        \eta_i = \xi_i - \xi_{i - 1} \quad (i = 2,3,\bm{\cdots}, n)
    \end{cases}
\]

\paragraph*{1.8} 已知矩阵$\mathbf{R}^{2\times 2} $的两个基 \par
(\uppercase\expandafter{\romannumeral1})
\[
    \bm{A}_1 = \begin{bmatrix}
        1 & 0 \\
        0 & 1
    \end{bmatrix}, \quad \bm{A}_2 = \begin{bmatrix}
        1 & 0  \\
        0 & -1
    \end{bmatrix}, \quad \bm{A}_3 = \begin{bmatrix}
        0 & 1 \\
        1 & 0
    \end{bmatrix}, \quad \bm{A}_4 = \begin{bmatrix}
        0  & 1 \\
        -1 & 0
    \end{bmatrix}
\]
(\uppercase\expandafter{\romannumeral2})
\[
    \bm{B}_1 = \begin{bmatrix}
        1 & 1 \\
        1 & 1
    \end{bmatrix}, \quad \bm{B}_2 = \begin{bmatrix}
        1 & 1 \\
        1 & 0
    \end{bmatrix}, \quad \bm{B}_3 = \begin{bmatrix}
        1 & 1 \\
        0 & 0
    \end{bmatrix}, \quad \bm{B}_4 = \begin{bmatrix}
        1 & 0 \\
        0 & 0
    \end{bmatrix}
\]
求由基(\uppercase\expandafter{\romannumeral1})改变为(\uppercase\expandafter{\romannumeral2})的过渡矩阵.

\paragraph*{解} 为了计算简单,采用中介基方法,引进$\mathbf{R}^{2\times 2}$的简单基\par
(\uppercase\expandafter{\romannumeral3})
\[
    \bm{E}_{11} = \begin{bmatrix}
        1 & 0 \\
        0 & 0
    \end{bmatrix}, \quad \bm{E}_{12} = \begin{bmatrix}
        0 & 1 \\
        0 & 0
    \end{bmatrix}, \quad \bm{E}_{21} = \begin{bmatrix}
        0 & 0 \\
        1 & 0
    \end{bmatrix}, \quad \bm{E}_{22} = \begin{bmatrix}
        0 & 0 \\
        0 & 1
    \end{bmatrix}
\]
直接写出由基(\uppercase\expandafter{\romannumeral3})改变为基(\uppercase\expandafter{\romannumeral1})的过渡矩阵为
\[
    \bm{C}_1 = \begin{bmatrix}
        1 & 1  & 0 & 0  \\
        0 & 0  & 1 & 1  \\
        0 & 0  & 1 & -1 \\
        1 & -1 & 0 & 0
    \end{bmatrix}
\]
即
\[
    (\bm{A}_1, \bm{A}_2, \bm{A}_3, \bm{A}_4) = (\bm{E}_1, \bm{E}_2, \bm{E}_3, \bm{E}_4)\bm{C}_1
\]
再写出由基(\uppercase\expandafter{\romannumeral3})改变为基(\uppercase\expandafter{\romannumeral2})的过渡矩阵为
\[
    \bm{C}_2 = \begin{bmatrix}
        1 & 1 & 1 & 1 \\
        1 & 1 & 1 & 0 \\
        1 & 1 & 0 & 0 \\
        1 & 0 & 0 & 0
    \end{bmatrix}
\]
即
\[
    (\bm{B}_1, \bm{B}_2, \bm{B}_3, \bm{B}_4) = (\bm{E}_1, \bm{E}_2, \bm{E}_3, \bm{E}_4)\bm{C}_2
\]
所以有
\[
    (\bm{B}_1, \bm{B}_2, \bm{B}_3, \bm{B}_4) = (\bm{A}_1, \bm{A}_2, \bm{A}_3, \bm{A}_4)\bm{C}^{-1}\bm{C}_2
\]
于是得由基(\uppercase\expandafter{\romannumeral1})改变为(\uppercase\expandafter{\romannumeral2})的过渡矩阵为
\[
    \bm{C} = \bm{C_1^{-1}C_2} = \dfrac{1}{2} \begin{bmatrix}
        1 & 0 & 0  & 1  \\
        1 & 0 & 0  & -1 \\
        0 & 1 & 1  & 0  \\
        0 & 1 & -1 & 0
    \end{bmatrix} \begin{bmatrix}
        1 & 1 & 1 & 1 \\
        1 & 1 & 1 & 0 \\
        1 & 1 & 0 & 0 \\
        1 & 0 & 0 & 0
    \end{bmatrix} = \dfrac{1}{2} \begin{bmatrix}
        2 & 1 & 1 & 1 \\
        0 & 1 & 1 & 1 \\
        2 & 2 & 1 & 0 \\
        0 & 0 & 1 & 0
    \end{bmatrix}
\]

\paragraph*{1.18} 设
\[
    B = \begin{bmatrix}
        1 & 1 \\
        0 & 1
    \end{bmatrix}
\]
线性空间
\[
    V = \{X = (\bm{X}_{ij})_{2\times 2} | x_{11} + x_{22} = 0, \, x_{ij} \in R\}
\]
中的线性变换为$T(\bm{X}) = \bm{B}^T\bm{X} - \bm{X}^T\bm{B}\, (\forall \bm{X} \in \bm{V})$,求$T$的特征值与特征向量.

\paragraph*{解} 设
\[
    \bm{X} = \begin{bmatrix}
        x_{11} & x_{12} \\
        x_{21} & x_{22}
    \end{bmatrix} \in V
\]
则有
\begin{align*}
    \bm{X} & = \begin{bmatrix}
                   x_{11} & x_{12}  \\
                   x_{21} & -x_{11}
               \end{bmatrix} = \begin{bmatrix}
                                   x_{11} & 0       \\
                                   0      & -x_{11}
                               \end{bmatrix} + \begin{bmatrix}
                                                   0 & x_{12} \\
                                                   0 & 0
                                               \end{bmatrix} + \begin{bmatrix}
                                                                   0      & 0 \\
                                                                   x_{21} & 0
                                                               \end{bmatrix}      \\
           & = x_{11} \begin{bmatrix}
                          1 & 0  \\
                          0 & -1
                      \end{bmatrix} + x_{12} \begin{bmatrix}
                                                 0 & 1 \\
                                                 0 & 0
                                             \end{bmatrix} + x_{21} \begin{bmatrix}
                                                                        0 & 0 \\
                                                                        1 & 0
                                                                    \end{bmatrix}
\end{align*}
这表示$\bm{X} \in V$可由
\[
    \bm{X}_1 \begin{bmatrix}
        1 & 0  \\
        0 & -1
    \end{bmatrix}, \quad \bm{X}_2 = \begin{bmatrix}
        0 & 1 \\
        0 & 0
    \end{bmatrix}, \quad \bm{X}_3 = \begin{bmatrix}
        0 & 0 \\
        1 & 0
    \end{bmatrix}
\]
线性表示.容易验证$\bm{X}_1, \bm{X}_2, \bm{X}_3$线性无关,故$\bm{X}_1, \bm{X}_2, \bm{X}_3$构成$V$的一个基,且$X$在该基下的
坐标为$(x_{11},x_{12},x_{21})^T$.由线性变换公式求得
\begin{gather*}
    T(\bm{X}_1) = \begin{bmatrix}
        0 & -1 \\
        1 & 0
    \end{bmatrix} = 0\bm{X}_1 - 1\bm{X}_2 + 1\bm{X}_3 \\
    T(\bm{X}_2) = \begin{bmatrix}
        0  & 1 \\
        -1 & 0
    \end{bmatrix} = 0\bm{X}_1 + 1\bm{X}_2 - 1\bm{X}_3 \\
    T(\bm{X}_3) = \begin{bmatrix}
        0 & -1 \\
        1 & 0
    \end{bmatrix} = 0\bm{X}_1 - 1\bm{X}_2 + 1\bm{X}_3
\end{gather*}
故$T$在该基下的矩阵为
\[
    \bm{A} = \begin{bmatrix}
        0  & 0  & 0  \\
        -1 & 1  & -1 \\
        1  & -1 & 1
    \end{bmatrix}
\]
解得
\[
    \lambda_1 = \lambda_2 = 0, \quad \alpha_1 = \begin{bmatrix}
        1 \\
        1 \\
        0
    \end{bmatrix}, \quad \alpha_2 = \begin{bmatrix}
        0 \\
        1 \\
        1
    \end{bmatrix}; \quad \lambda_3 = 2, \quad \alpha_3 = \begin{bmatrix}
        0 \\
        1 \\
        -1
    \end{bmatrix}
\]
那么,$T$的特征值$\lambda_1 = \lambda_2 = 0$对应的线性无关的特征向量为
\[
    \bm{Y}_1 = (\bm{X}_1, \bm{X}_2, \bm{X}_3) \alpha_1 = \begin{bmatrix}
        1 & 1  \\
        0 & -1
    \end{bmatrix}, \quad \bm{Y}_2  = (\bm{X}_1, \bm{X}_2, \bm{X}_3) \alpha_2 = \begin{bmatrix}
        0 & 1 \\
        1 & 0
    \end{bmatrix}
\]
全体特征向量为$k_1\bm{Y}_1 + k_2\bm{Y_2}(k_1, k_2 \in R$ 不同时为零$)$;$T$的特征值$\lambda_3 = 2$的线性无关的特征
向量为
\[
    \bm{Y}_3 = (\bm{X}_1, \bm{X}_2, \bm{X}_3) \alpha_3 = \begin{bmatrix}
        0  & 1 \\
        -1 & 0
    \end{bmatrix}
\]
全体特征向量为$k_3\bm{Y}_3(0\neq k_3 \in R).$

\centerline{\large{\textbf{习题}}} \ \par

\paragraph*{1.2.7} 已知$R^3$的线性变换$T$在基$\bm{x}_1 = (-1,1,1),\bm{x}_2 = (1,0,-1), \bm{x}_3 = (0,1,1)$下的矩阵是
\[
    \begin{bmatrix}
        1  & 0 & 1 \\
        1  & 1 & 0 \\
        -1 & 2 & 1
    \end{bmatrix}
\]
求$T$在基$\bm{e}_1 = (1,0,0),\bm{e}_2 = (0,1,0),\bm{e}_3 = (0,0,1)$下的矩阵.

\paragraph*{解} 由题意,可得
\[
    \begin{bmatrix}
        1 & 0 & 0 \\
        0 & 1 & 0 \\
        0 & 0 & 1
    \end{bmatrix} = \begin{bmatrix}
        -1 & 1  & 0 \\
        1  & 0  & 1 \\
        1  & -1 & 1
    \end{bmatrix} \bm{C}
\]
解得
\[
    \bm{C}^{-1} = \begin{bmatrix}
        -1 & 1  & 0 \\
        1  & 0  & 1 \\
        1  & -1 & 1
    \end{bmatrix}, \quad \bm{C}^ = \begin{bmatrix}
        -1 & 1 & -1 \\
        0  & 1 & -1 \\
        1  & 0 & 1
    \end{bmatrix}
\]
由$\bm{B} = \bm{C}^{-1}\bm{AC}$得,
\begin{align*}
    \bm{B} & = \begin{bmatrix}
                   -1 & 1  & 0 \\
                   1  & 0  & 1 \\
                   1  & -1 & 1
               \end{bmatrix}\begin{bmatrix}
                                1  & 0 & 1 \\
                                1  & 1 & 0 \\
                                -1 & 2 & 1
                            \end{bmatrix}\begin{bmatrix}
                                             -1 & 1 & -1 \\
                                             0  & 1 & -1 \\
                                             1  & 0 & 1
                                         \end{bmatrix} \\
           & = \begin{bmatrix}
                   0  & 1 & -1 \\
                   0  & 2 & 2  \\
                   -1 & 1 & 2
               \end{bmatrix} \begin{bmatrix}
                                 -1 & 1  & 0 \\
                                 1  & 0  & 1 \\
                                 1  & -1 & 1
                             \end{bmatrix}             \\
           & = \begin{bmatrix}
                   -1 & 1 & -2 \\
                   2  & 2 & 0  \\
                   3  & 0 & 2
               \end{bmatrix}
\end{align*}

\paragraph*{1.2.11} 给定$R^3$得两个基
\begin{gather*}
    \bm{x}_1 = (1,0,1), \quad \bm{x}_2 = (2,1,0), \quad \bm{x}_3 = (1,1,1) \\
    \bm{y}_1 = (1,2,-1), \quad \bm{y}_2 = (2,2,-1), \quad \bm{y}_3 = (2,-1,-1)
\end{gather*}
定义线性变换
\[
    T\bm{x}_i = y_i \quad (i = 1,2,3)
\]
\begin{enumerate}
    \item[(1)] 写出由基$\bm{x}_1, \bm{x}_2, \bm{x}_3$到基$\bm{y}_1, \bm{y}_2, \bm{y}_3$得过渡矩阵.
    \item[(2)] 写出$T$在基$\bm{x}_1, \bm{x}_2, \bm{x}_3$下得矩阵.
    \item[(3)] 写出$T$在基$\bm{y}_1, \bm{y}_2, \bm{y}_3$下得矩阵.
\end{enumerate}

\paragraph*{解}
\begin{enumerate}
    \item[(1)] 由$(\bm{y}_1, \bm{y}_2, \bm{y}_3) = (\bm{x}_1, \bm{x}_2, \bm{x}_3) \bm{C}$得,
        \begin{gather*}
            \begin{bmatrix}
                1  & 2  & 2  \\
                2  & 2  & -1 \\
                -1 & -1 & -1
            \end{bmatrix} = \begin{bmatrix}
                1 & 2 & 1 \\
                0 & 1 & 1 \\
                1 & 0 & 1
            \end{bmatrix} \bm{C}
        \end{gather*}
        解得
        \[
            \bm{C} = \dfrac{1}{2} \begin{bmatrix}
                -4 & -3 & 3  \\
                2  & 3  & 3  \\
                2  & 1  & -5
            \end{bmatrix}
        \]
    \item[(2)] 由于$T(\bm{x}_1, \bm{x}_2, \bm{x}_3) = ((\bm{y}_1, \bm{y}_2, \bm{y}_3)) = (\bm{x}_1, \bm{x}_2, \bm{x}_3) \bm{C}$ \\
        那么假设$T$在该基下的矩阵为$\bm{A}$,有$\bm{A} = \bm{C}$
    \item[(3)] 那么假设$T$在该基下的矩阵为$\bm{B}$,有
        \begin{align*}
            \bm{B} & = \bm{C}^{-1} A \bm{C} \\
                   & = \bm{C}^{-1}\bm{CC}   \\
                   & = \bm{C}
        \end{align*}
\end{enumerate}

\paragraph*{1.2.12} 设$T$是数域$\mathbb{C}$上线性空间$V^3$的线性变换,已知$T$在$V^3$的基$\bm{x}_1, \bm{x}_2, \bm{x}_3$下的矩阵
\[
    \bm{A} = \begin{bmatrix}
        3  & 1  & 0  \\
        -4 & -1 & 0  \\
        4  & -8 & -2
    \end{bmatrix}
\]
求$T$的特征值与特征向量.

\paragraph*{解} 由题意
\[
    |\lambda E - A| = \begin{vmatrix}
        \lambda - 3 & -1          & 0           \\
        4           & \lambda + 1 & 0           \\
        -4          & 8           & \lambda + 2
    \end{vmatrix} = (\lambda + 2)(\lambda - 1)^2
\]
当$\lambda = -2$时,
\[
    -2E - A = \begin{bmatrix}
        -5 & -1 & 0 \\
        4  & -1 & 0 \\
        -4 & 8  & 0
    \end{bmatrix} \to \begin{bmatrix}
        1 & 0 & 0 \\
        0 & 1 & 0 \\
        0 & 0 & 0
    \end{bmatrix}
\]
解得特征向量为$\alpha_1 = (0,0,1)^T$. \\
当$\lambda = 1$时,有
\[
    E - A = \begin{bmatrix}
        -2 & -1 & 0 \\
        4  & 2  & 0 \\
        -4 & 8  & 3
    \end{bmatrix} \rightarrow \begin{bmatrix}
        2 & 1  & 0 \\
        0 & 10 & 3 \\
        0 & 0  & 0
    \end{bmatrix}
\]
解得特征向量为$\alpha_2 = (3,6,-20)^T$. \\
那么,$T$得特征值$\lambda = -2$时对应得线性无关的特征向量为
\[
    \bm{Y}_1 = \bm{x}_3
\]
全体特征向量为$k\bm{Y}_1,k\neq 0$;$T$的特征值$\lambda_2 = \lambda_3 = 1$对应的线性无关的特征向量为
\[
    \bm{Y}_2 = 3\bm{x}_1 - 6\bm{x}_2 + 20\bm{x}_3
\]
全体特征向量为$k\bm{Y}_2,k\neq 0.$

\end{document}