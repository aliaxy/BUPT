\documentclass[12pt, a4paper, oneside, fontset=none]{ctexart}
\usepackage{amsmath, amsthm, amssymb, graphicx, color, fontspec, float, pgfplots}
\usepackage{bm}
\usepackage[bookmarks=true, colorlinks, citecolor=blue, linkcolor=black]{hyperref}
\pgfplotsset{compat=1.16}
\usepackage{xeCJK, CJKnumb}
\xeCJKsetup{CJKmath=true,CheckSingle=true}
\setCJKmainfont[ItalicFont=KaiTi]{微软雅黑}
\usepackage{geometry}
\geometry{left=1.8cm, right=1.8cm, top=2.18cm, bottom=2.18cm}
\author{}
\date{}
\linespread{1.25}
\title{\vspace{-3em}\textbf{矩阵论 \quad 第十四次作业}\vspace{-3em}}

\begin{document}

\maketitle

\section*{第4章 \quad 矩阵分解}

\subsection*{4.3 \quad 矩阵的满秩分解}

\paragraph*{定义 4.8} 设 $\bm{A} \in \bm{C}^{m\times n}_r (r > 0)$, 如果存在矩阵 $\bm{F} \in \bm{C}^{m\times n}_r$ 和 $\bm{G} \in \bm{C}^{m\times n}_r$, 使得
$$
    \bm{A} = \bm{FG}
$$
则称其为矩阵 $\bm{A}$ 的\textbf{满秩分解}.
\par 当 $\bm{A}$ 是满秩(列满秩或行满秩)矩阵时, $\bm{A}$ 可分解为一个因子是单位矩阵, 另一个因子是 $\bm{A}$
本身, 称此满秩分解为\textbf{平凡分解}.

\paragraph*{定理 4.13} 设 $\bm{A} \in \bm{C}^{m\times n}_r(r > 0)$, 则 $\bm{A}$ 有满秩分解 $\bm{A} = \bm{FG}$.

\paragraph*{例 4.10} 求矩阵 $\bm{A}$ 的满秩分解, 其中
$$
    \bm{A} = \begin{bmatrix}
        -1 & 0 & 1  & 2  \\
        1  & 2 & -1 & 1  \\
        2  & 2 & -2 & -1
    \end{bmatrix}
$$

\paragraph*{解}
\begin{align*}
    [\bm{A} \vdots \bm{I}] & = \begin{bmatrix}
                                   -1 & 0 & 1  & 2  & \vdots & 1 & 0 & 0 \\
                                   1  & 2 & -1 & 1  & \vdots & 0 & 1 & 0 \\
                                   2  & 2 & -2 & -1 & \vdots & 0 & 0 & 1
                               \end{bmatrix} \\
                           & \to \begin{bmatrix}
                                     -1 & 0 & 1 & 2 & \vdots & 1 & 0 & 0 \\
                                     0  & 2 & 0 & 3 & \vdots & 1 & 1 & 0 \\
                                     0  & 0 & 0 & 0 & \vdots & 0 & 0 & 1
                                 \end{bmatrix}
\end{align*}
则有
$$
    \bm{B} = \begin{bmatrix}
        -1 & 0 & 1 & 2 \\
        0  & 2 & 0 & 3 \\
        0  & 0 & 0 & 0
    \end{bmatrix}, \quad
    \bm{P} = \begin{bmatrix}
        1 & 0  & 0 \\
        1 & 1  & 0 \\
        1 & -1 & 1
    \end{bmatrix}
$$
可求得
$$
    \bm{P}^{-1} = \begin{bmatrix}
        1  & 0 & 0 \\
        -1 & 1 & 0 \\
        -2 & 1 & 1
    \end{bmatrix}
$$
于是有
$$
    \bm{A} = \begin{bmatrix}
        1  & 0 \\
        -1 & 1 \\
        -2 & 1
    \end{bmatrix} \begin{bmatrix}
        -1 & 0 & 1 & 2 \\
        0  & 2 & 0 & 3
    \end{bmatrix}
$$

\paragraph*{习题 4.3.1} 求下列各矩阵的满秩分解.
$$
    (1) \begin{bmatrix}
        1 & 2 & 3 & 0  \\
        0 & 2 & 1 & -1 \\
        1 & 0 & 2 & 1
    \end{bmatrix};\quad (2) \begin{bmatrix}
        1  & -1 & 1  & 1  \\
        -1 & 1  & -1 & -1 \\
        -1 & -1 & 1  & 1  \\
        1  & 1  & -1 & -1
    \end{bmatrix}
$$

\paragraph*{解}

\begin{enumerate}
    \item[(1)]
        \begin{align*}
            [\bm{A} \vdots \bm{I}] & = \begin{bmatrix}
                                           1 & 2 & 3 & 0  & \vdots & 1 & 0 & 0 \\
                                           0 & 2 & 1 & -1 & \vdots & 0 & 1 & 0 \\
                                           1 & 0 & 2 & 1  & \vdots & 0 & 0 & 1
                                       \end{bmatrix}    \\
                                   & \to \begin{bmatrix}
                                             1 & 2 & 3 & 0  & \vdots & 1  & 0 & 0 \\
                                             0 & 2 & 1 & -1 & \vdots & 0  & 1 & 0 \\
                                             0 & 0 & 0 & 0  & \vdots & -1 & 1 & 1
                                         \end{bmatrix}
        \end{align*}
        则有
        $$
            \bm{B} = \begin{bmatrix}
                1 & 2 & 3 & 0  \\
                0 & 2 & 1 & -1 \\
                0 & 0 & 0 & 0
            \end{bmatrix},\quad
            \bm{P} = \begin{bmatrix}
                1  & 0 & 0 \\
                0  & 1 & 0 \\
                -1 & 1 & 1
            \end{bmatrix}
        $$
        可求得
        $$
            \bm{P}^{-1} = \begin{bmatrix}
                1 & 0  & 0 \\
                0 & 1  & 0 \\
                1 & -1 & 1
            \end{bmatrix}
        $$
        于是有
        $$
            \bm{A} = \begin{bmatrix}
                1 & 0  \\
                0 & 1  \\
                1 & -1
            \end{bmatrix} \begin{bmatrix}
                1 & 2 & 3 & 0  \\
                0 & 2 & 1 & -1
            \end{bmatrix}
        $$
    \item[(2)]
        \begin{align*}
            [\bm{A} \vdots \bm{I}] & = \begin{bmatrix}
                                           1  & -1 & 1  & 1  & \vdots & 1 & 0 & 0 & 0 \\
                                           -1 & 1  & -1 & -1 & \vdots & 0 & 1 & 0 & 0 \\
                                           -1 & -1 & 1  & 1  & \vdots & 0 & 0 & 1 & 0 \\
                                           1  & 1  & -1 & -1 & \vdots & 0 & 0 & 0 & 1
                                       \end{bmatrix} \\
                                   & \to \begin{bmatrix}
                                             1 & -1 & 1 & 1 & \vdots & 1 & 0 & 0 & 0 \\
                                             0 & -2 & 2 & 2 & \vdots & 1 & 0 & 1 & 0 \\
                                             0 & 0  & 0 & 0 & \vdots & 1 & 1 & 0 & 0 \\
                                             0 & 0  & 0 & 0 & \vdots & 0 & 0 & 1 & 1
                                         \end{bmatrix}
        \end{align*}
        则有
        $$
            \bm{B} = \begin{bmatrix}
                1 & -1 & 1 & 1 \\
                0 & -2 & 2 & 2 \\
                0 & 0  & 0 & 0 \\
                0 & 0  & 0 & 0
            \end{bmatrix}, \quad
            \bm{P} = \begin{bmatrix}
                1 & 0 & 0 & 0 \\
                1 & 0 & 1 & 0 \\
                1 & 1 & 0 & 0 \\
                0 & 0 & 1 & 1
            \end{bmatrix}
        $$
        可求得
        $$
            \bm{P}^{-1} = \begin{bmatrix}
                1  & 0  & 0 & 0 \\
                -1 & 0  & 1 & 0 \\
                -1 & 1  & 0 & 0 \\
                1  & -1 & 0 & 1
            \end{bmatrix}
        $$
        于是有
        $$
            \bm{A} = \begin{bmatrix}
                1  & 0  \\
                -1 & 0  \\
                -1 & 1  \\
                1  & -1
            \end{bmatrix} \begin{bmatrix}
                1 & -1 & 1 & 1 \\
                0 & -2 & 2 & 2
            \end{bmatrix}
        $$
\end{enumerate}

\subsection*{4.4 \quad 矩阵的奇异值分解}

\paragraph*{定义 4.11} 设 $\bm{A} \in \bm{C}^{m\times n}_r(r > 0)$, $\bm{A}^{H}\bm{A}$ 的特征值为
$$
    \lambda_1 \geqslant \lambda_2 \geqslant \cdots \geqslant \lambda_r > \lambda_{r + 1} = \cdots = \lambda_n = 0
$$
则称 $\sigma_i = \sqrt{\lambda_i} (i = 1,2, \cdots, n)$ 为 $\bm{A}$ 的\textbf{奇异值}; 当 $\bm{A}$ 为零矩阵时, 它的奇异值都是0.

\paragraph*{定理 4.15} 设 $\bm{A} \in R^{n\times n}$ 可逆, 则存在正交矩阵 $\bm{P}$ 和 $\bm{Q}$, 使得
$$
    \bm{Q}^T(\bm{A}^T\bm{A})\bm{Q} = \mathrm{diag}(\lambda_1,\lambda_2,\cdots,\lambda_n)
$$
其中 $\lambda_i > 0 (i = 1, 2, \cdots, n)$ 为 $\bm{A}^T\bm{A}$ 的特征值.

\paragraph*{定理 4.16} 设 $\bm{A} \in \bm{C}^{m\times n}_r(r > 0)$, 则存在 $m$ 阶酉矩阵 $\bm{U}$ 和 $n$ 阶酉矩阵 $\bm{V}$, 使得
$$
    \bm{U}^H\bm{AV} = \begin{bmatrix}
        \bm{\Sigma} & \bm{O} \\
        \bm{O}      & \bm{O}
    \end{bmatrix}
$$
其中 $\Sigma = \mathrm{diag}(\sigma_1, \sigma_2, \cdots, \sigma_r)$, 而 $\sigma_i(i = 1,2,\cdots,r)$ 为矩阵 $\bm{A}$ 的全部非零奇异值.

\paragraph*{例 4.14} 求矩阵 $\bm{A} = \begin{bmatrix}
        1 & 0 & 1 \\
        0 & 1 & 1 \\
        0 & 0 & 0
    \end{bmatrix}$ 的奇异值分解.

\paragraph*{解} 计算
$$
    \bm{B} = \bm{A}^T\bm{A} = \begin{bmatrix}
        1 & 0 & 1 \\
        0 & 1 & 1 \\
        1 & 1 & 2
    \end{bmatrix}
$$
求得 $\bm{B}$ 的特征值为 $\lambda_1 = 3, \lambda_2 = 1, \lambda_3 = 0$, 对应的特征向量依次是
$$
    \bm{\xi}_1 = \begin{bmatrix}
        1 \\
        1 \\
        2
    \end{bmatrix}, \quad
    \bm{\xi}_2 = \begin{bmatrix}
        1  \\
        -1 \\
        0
    \end{bmatrix}, \quad
    \bm{\xi}_3 = \begin{bmatrix}
        1 \\
        1 \\
        -1
    \end{bmatrix}
$$
可得
$$
    \mathrm{rank}\bm{A} = 2, \bm{\Sigma} = \begin{bmatrix}
        \sqrt{3} & 0 \\
        0        & 1
    \end{bmatrix}
$$
且正交矩阵为
$$
    \bm{V} = \begin{bmatrix}
        \dfrac{1}{\sqrt{6}} & \dfrac{1}{\sqrt{2}}  & \dfrac{1}{\sqrt{3}}  \\
        \dfrac{1}{\sqrt{6}} & -\dfrac{1}{\sqrt{2}} & \dfrac{1}{\sqrt{3}}  \\
        \dfrac{2}{\sqrt{6}} & 0                    & -\dfrac{1}{\sqrt{3}}
    \end{bmatrix}
$$
计算
$$
    \bm{U}_1 = \bm{AV}_1\bm{\Sigma}^{-1} = \begin{bmatrix}
        \dfrac{1}{\sqrt{2}} & \dfrac{1}{\sqrt{2}}  \\
        \dfrac{1}{\sqrt{2}} & -\dfrac{1}{\sqrt{2}} \\
        0                   & 0
    \end{bmatrix}
$$
构造
$$
    \bm{U}_2 = \begin{bmatrix}
        0 \\
        0 \\
        1
    \end{bmatrix}, \quad
    \bm{U} = \begin{bmatrix}
        \dfrac{1}{\sqrt{2}} & \dfrac{1}{\sqrt{2}}  & 0 \\
        \dfrac{1}{\sqrt{2}} & -\dfrac{1}{\sqrt{2}} & 0 \\
        0                   & 0                    & 1
    \end{bmatrix}
$$
则 $\bm{A}$ 的奇异值分解为
$$
    \bm{A} = \bm{U}\begin{bmatrix}
        \sqrt{3} & 0 & 0 \\
        0        & 1 & 0 \\
        0        & 0 & 0
    \end{bmatrix}\bm{V}^T
$$

\paragraph*{例 4.15} 设矩阵 $\bm{A}$ 的奇异值分解为
$$
    \bm{A} = \bm{U}\begin{bmatrix}
        \bm{\Sigma} & \bm{O} \\
        \bm{O}      & \bm{O}
    \end{bmatrix}\bm{V}^H
$$
证明: $\bm{U}$ 的列向量是 $\bm{AA}^H$ 的特征向量, $\bm{V}$ 的列向量是$\bm{AA}^H$ 的特征向量.

\paragraph*{证} 可求得
$$
    \bm{AA}^H = \bm{U}\begin{bmatrix}
        \bm{\Sigma}^2 & \bm{O} \\
        \bm{O}        & \bm{O}
    \end{bmatrix}\bm{U}^H
$$
即
$$
    (\bm{AA}^H)\bm{U} = \bm{U} \cdot \mathrm{diag}(\lambda_1, \lambda_2, \cdots, \lambda_r, 0, \cdots, 0)
$$
记 $\bm{U} = (\bm{u}_1, \bm{u}_2, \cdots, \bm{u}_m)$, 则有
$$
    (\bm{AA}^H)\bm{u}_i = \lambda_i\bm{u}_i (i = 1,2,\cdots,m)
$$
这表明 $\bm{u}_i$ 是 $\bm{AA}^H$ 的属于特征值 $\lambda_i$ 的特征向量. 同理可证另一结论.

\paragraph*{习题 4.4.2} 给出应用奇异值分解式求解齐次线性方程组 $\bm{Ax} = \bm{0}$ 的方法.

\paragraph*{解} 当 $\bm{Ax} = \bm{0}$ 时, $\bm{A}^T\bm{Ax} = \bm{0}$, 所以 $\bm{Ax} = \bm{0}$ 的解是$\bm{A}^T\bm{Ax} = \bm{0}$ 的解.
\\ 当 $\bm{A}^T\bm{Ax} = \bm{0}$ 时, 等式两边同时乘以 $\bm{x}^T$, 得 $\bm{x}^T\bm{A}^T\bm{Ax} = \bm{0}$, 也就是 $(\bm{Ax})^T\bm{Ax} = \bm{0}$.
\\ 而 $(\bm{Ax})^T\bm{Ax} = ||\bm{Ax}||$, 称为 $\bm{Ax}$ 的范数, 它的取值大于等于0, 当且仅当 $\bm{Ax} = 0$ 时, $||\bm{Ax}|| = 0$.
\\ 所以 $\bm{A}^T\bm{Ax} = \bm{0}$ 的解是 $\bm{Ax} = \bm{0}$ 的解.

\begin{gather*}
    \bm{AA}^T = \bm{E\Sigma V}^T\bm{V\Sigma}^T\bm{U}^T = \bm{U\Sigma\Sigma}^T\bm{U}^T \\
    \bm{A}^T\bm{A} = \bm{V\Sigma}^T\bm{U}^T\bm{U\Sigma V}^T = \bm{V\Sigma}^T\bm{\Sigma V}^T \\
    \bm{\Sigma\Sigma}^T = \begin{bmatrix}
        \sigma_1^2 & 0          & \cdots & 0          \\
        0          & \sigma_2^2 & \cdots & 0          \\
        \vdots     & \vdots     &        & \vdots     \\
        0          & 0          & \cdots & \sigma_m^2
    \end{bmatrix}, \quad \bm{\Sigma^T}\bm{\Sigma}= \begin{bmatrix}
        \sigma_1^2 & 0          & \cdots & 0          \\
        0          & \sigma_2^2 & \cdots & 0          \\
        \vdots     & \vdots     &        & \vdots     \\
        0          & 0          & \cdots & \sigma_n^2
    \end{bmatrix}
\end{gather*}

\paragraph*{习题 4.4.4} 求 $\bm{A} = \begin{bmatrix}
        1 & 0 \\
        0 & 1 \\
        1 & 1
    \end{bmatrix}$ 的奇异值分解.

\paragraph*{解} 计算
$$
    \bm{B} = \bm{A}^T\bm{A} = \begin{bmatrix}
        2 & 1 \\
        1 & 2
    \end{bmatrix}
$$
求得 $\bm{B}$ 的特征值为 $\lambda_1 = 3, \lambda_2 = 1$, 对应的特征向量为
$$
    \bm{\xi}_1 = \begin{bmatrix}
        1 \\
        1
    \end{bmatrix}, \quad \bm{\xi}_2 = \begin{bmatrix}
        1 \\
        -1
    \end{bmatrix}
$$
于是可得
$$
    \mathrm{rank}\bm{A} = 2, \bm{\Sigma} = \begin{bmatrix}
        \sqrt{3} & 0 \\
        0        & 1
    \end{bmatrix}
$$
且正交矩阵为
$$
    \bm{V} = \begin{bmatrix}
        \dfrac{1}{\sqrt{2}} & \dfrac{1}{\sqrt{2}}  \\
        \dfrac{1}{\sqrt{2}} & -\dfrac{1}{\sqrt{2}}
    \end{bmatrix}
$$
构造
$$
    \bm{U} = \begin{bmatrix}
        \dfrac{1}{\sqrt{6}} & \dfrac{1}{\sqrt{2}}  & \dfrac{1}{\sqrt{3}}  \\
        \dfrac{1}{\sqrt{6}} & -\dfrac{1}{\sqrt{2}} & \dfrac{1}{\sqrt{3}}  \\
        \dfrac{2}{\sqrt{6}} & 0                    & -\dfrac{1}{\sqrt{3}}
    \end{bmatrix}
$$
因此
$$
    \bm{A} = \begin{bmatrix}
        \dfrac{1}{\sqrt{6}} & \dfrac{1}{\sqrt{2}}  & \dfrac{1}{\sqrt{3}}  \\
        \dfrac{1}{\sqrt{6}} & -\dfrac{1}{\sqrt{2}} & \dfrac{1}{\sqrt{3}}  \\
        \dfrac{2}{\sqrt{6}} & 0                    & -\dfrac{1}{\sqrt{3}}
    \end{bmatrix}\begin{bmatrix}
        \sqrt{3} & 0 \\
        0        & 1 \\
        0        & 0
    \end{bmatrix}\begin{bmatrix}
        \dfrac{1}{\sqrt{2}} & \dfrac{1}{\sqrt{2}}  \\
        \dfrac{1}{\sqrt{2}} & -\dfrac{1}{\sqrt{2}}
    \end{bmatrix}
$$

\end{document}