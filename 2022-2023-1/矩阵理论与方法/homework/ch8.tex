\documentclass[12pt, a4paper, oneside, fontset=none]{ctexart}
\usepackage{amsmath, amsthm, amssymb, graphicx, color, fontspec, float, pgfplots}
\usepackage{bm}
\usepackage[bookmarks=true, colorlinks, citecolor=blue, linkcolor=black]{hyperref}
\pgfplotsset{compat=1.16}
\usepackage{xeCJK, CJKnumb}
\xeCJKsetup{CJKmath=true,CheckSingle=true}
\setCJKmainfont[ItalicFont=KaiTi]{微软雅黑}
\usepackage{geometry}
\geometry{left=1.8cm, right=1.8cm, top=2.18cm, bottom=2.18cm}
\author{}
\date{}
\linespread{1.25}
\title{\vspace{-3em}\textbf{矩阵论 \quad 第八次作业}\vspace{-3em}}

\begin{document}

\maketitle

\section*{第1章 \quad 线性空间和线性变换}

\subsection*{1.3 \quad 两个特殊的线性空间}

\centerline{\large{\textbf{ppt例题}}} \ \par

\paragraph*{P8} 设矩阵空间 $R^{2\times 2} $的子空间为
\[
    V = \{X = (x_{ij})_{2\times 2}\ |\ x_{11} + x_{12} + x_{21} = 0,\ x_{ij} \in R \}
\]
$V$中的线性变换为 $T(X) = X + X^T$.\\
求$(T^3)(X),\ X = \begin{bmatrix}
        4 & -4 \\
        0 & -3
    \end{bmatrix} \in V$.\\
求$(T^k)(X),\ \forall X \in V$.

\paragraph*{解} 任意找一组基,利用Schmidt正交化方法得到$V$的一组标准正交基$e_1,\cdots,e_n$,$x = k_1e_1+
    \cdots+k_ne_n$,其中 $k_i = (x,\ e_i)$.\\
令 $x_{11} = -x_{12} - x_{21}$
\[
    X = \begin{bmatrix}
        -x_{12} - x_{21} & x_{12} \\
        x_{21}           & x_{22}
    \end{bmatrix},\quad Y = \begin{bmatrix}
        -y_{12} - y_{21} & y_{12} \\
        y_{21}           & y_{22}
    \end{bmatrix}
\]
定义 $V$ 的内积为 $(X,\ Y) = tr(XY^T) = (x_{12} + x_{21})(y_{12} + y_{21}) + x_{12}y_{12} + x_{21}y_{21} + x_{22}y_{22}$. \\
任意找一组基
\[
    X = \begin{bmatrix}
        -x_{12} - x_{21} & x_{12} \\
        x_{21}           & x_{22}
    \end{bmatrix} = x_{12}\begin{bmatrix}
        -1 & 1 \\
        0  & 0
    \end{bmatrix} + x_{21}\begin{bmatrix}
        -1 & 0 \\
        1  & 0
    \end{bmatrix} + x_{22}\begin{bmatrix}
        0 & 0 \\
        0 & 1
    \end{bmatrix} = x_{12}X_1 + x_{21}X_2 + x_{22}X_3
\]
则,
\begin{align*}
    Y_1' & = X_1 = \begin{bmatrix}
                       -1 & 1 \\
                       0  & 0
                   \end{bmatrix}                                                              \\
    Y_2' & = X_2 - \dfrac{(X_2, Y_1')}{(Y_1',Y_1')}Y_1'                                        \\
         & = \begin{bmatrix}
                 -1 & 0 \\
                 1  & 0
             \end{bmatrix} - \dfrac{1}{2}\begin{bmatrix}
                                             -1 & 1 \\
                                             0  & 0
                                         \end{bmatrix}
    = \dfrac{1}{2}\begin{bmatrix}
                      -1 & -1 \\
                      2  & 0
                  \end{bmatrix}                                                               \\
    Y_3' & = X_3 - \dfrac{(X_3,Y_2')}{(Y_2', Y_2')}Y_2' - \dfrac{(X_3, Y_1')}{(Y_1',Y_1')}Y_1' \\
         & = \begin{bmatrix}
                 0 & 0 \\
                 0 & 1
             \end{bmatrix} - \dfrac{3}{2}\begin{bmatrix}
                                             -\frac{1}{2} & -\frac{1}{2} \\
                                             1            & 0
                                         \end{bmatrix} - 0
    = \begin{bmatrix}
          0 & 0 \\
          0 & 1
      \end{bmatrix}
\end{align*}
单位化,
\begin{align*}
    e_1 & = \dfrac{1}{|Y_1'|}Y_1' = \dfrac{1}{\sqrt{2}}\begin{bmatrix}
                                                           -1 & 1 \\
                                                           0  & 0
                                                       \end{bmatrix} \\
    e_2 & = \dfrac{1}{|Y_2'|}Y_2' = \dfrac{1}{\sqrt{6}}\begin{bmatrix}
                                                           -1 & -1 \\
                                                           2  & 0
                                                       \end{bmatrix} \\
    e_3 & = \dfrac{1}{|Y_3'|}Y_3' = \begin{bmatrix}
                                        0 & 0 \\
                                        0 & 1
                                    \end{bmatrix}
\end{align*}
则,
\[
    x = \begin{bmatrix}
        4 & -4 \\
        0 & -3
    \end{bmatrix} = (e_1,e_2,e_3)\begin{bmatrix}
        k_1 \\
        k_2 \\
        k_3
    \end{bmatrix}
    \begin{cases}
        k_1 = (x,e_1) = -4\sqrt{2} \\
        k_2 = (x,e_2) = 0          \\
        k_3 = (x,e_3) = -3
    \end{cases}
\]
\[
    Te_1 = \dfrac{1}{\sqrt{2}}\begin{bmatrix}
        -3 & 1 \\
        2  & 0
    \end{bmatrix},\quad Te_2 = \dfrac{1}{\sqrt{6}}\begin{bmatrix}
        -3 & 3 \\
        0  & 0
    \end{bmatrix},\quad Te_3 = \begin{bmatrix}
        0 & 0 \\
        0 & 3
    \end{bmatrix}
\]
解得,
\begin{align*}
    Te_1 & = (e_1,e_2,e_3)\begin{bmatrix}
                              2        \\
                              \sqrt{3} \\
                              0
                          \end{bmatrix} \\
    Te_2 & = (e_1,e_2,e_3)\begin{bmatrix}
                              \sqrt{3} \\
                              0        \\
                              0
                          \end{bmatrix}
    Te_3 & = (e_1,e_2,e_3)\begin{bmatrix}
                              0 \\
                              0 \\
                              3
                          \end{bmatrix}
\end{align*}
得到
\[
    T(e_1\cdots e_n) = (e_1\cdots e_n)\begin{bmatrix}
        2        & \sqrt{3} & 0 \\
        \sqrt{3} & 0        & 0 \\
        0        & 0        & 3
    \end{bmatrix} = (e_1\cdots e_n)A_0
\]
其中$A_0 = PJP^{-1}$,$J$是Jordan标准型$\Rightarrow T(e_1\cdots e_n) = (e_1\cdots e_n)PJP^{-1}.$
\begin{align*}
    \lambda I - A_0 & = \begin{bmatrix}
                            \lambda - 2 & -\sqrt{3} & 0           \\
                            -\sqrt{3}   & \lambda   & 0           \\
                            0           & 0         & \lambda - 3
                        \end{bmatrix}                                                 \\
                    & \to \begin{bmatrix}
                              -\sqrt{3} & \lambda - 2 & 0           \\
                              \lambda   & -\sqrt{3}   & 0           \\
                              0         & 0           & \lambda - 3
                          \end{bmatrix}
    \to \begin{bmatrix}
            -\sqrt{3} & 0                                          & 0           \\
            \lambda   & \frac{\lambda-2}{\sqrt{3}}\lambda-\sqrt{3} & 0           \\
            0         & 0                                          & \lambda - 3
        \end{bmatrix}                    \\
                    & \to \begin{bmatrix}
                              -\sqrt{3} & 0                                           & 0           \\
                              \lambda   & \frac{1}{\sqrt{3}}(\lambda + 1)(\lambda -3) & 0           \\
                              0         & 0                                           & \lambda - 3
                          \end{bmatrix}
    \to \begin{bmatrix}
            1 & 0                          & 0           \\
            0 & 0                          & \lambda - 3 \\
            0 & (\lambda + 1)(\lambda - 3) & 0
        \end{bmatrix}                                                          \\
                    & \to \begin{bmatrix}
                              1 & 0           & 0                          \\
                              0 & \lambda - 3 & 0                          \\
                              0 & 0           & (\lambda + 1)(\lambda - 3)
                          \end{bmatrix}
\end{align*}
不变因子:$d_1(\lambda) = 1,\ d_2(\lambda) = \lambda - 3,\ d_3(\lambda) = (\lambda + 1)(\lambda -3)$ \\
初等因子:$(\lambda - 3);\ (\lambda + 1),\ (\lambda - 3)$ \\
初等因子组:$(\lambda - 3),\ (\lambda + 1),\ (\lambda - 3)$ \\
Jordan块:$J_1(\lambda_1) = (3),\ J_2(\lambda_2) = (-1),\ J_3(\lambda_3) = (3)$
Jordan标准型:$J = \begin{bmatrix}
        3 &    &   \\
          & -1 &   \\
          &    & 3
    \end{bmatrix}$
则$P = (x_1,x_2,x_3),\ PJ = A_0P \Rightarrow (3x_1,-x_2,3x_3) = (A_0x_1,A_1x_2,A_0x_3).$
\begin{align*}
    (3I - A_0)x_1 & = \begin{bmatrix}
                          1         & -\sqrt{3} &   \\
                          -\sqrt{3} & 3         &   \\
                                    &           & 0
                      \end{bmatrix}x_1 = 0  \\
    (-I - A_0)x_2 & = \begin{bmatrix}
                          -3        & -\sqrt{3} &    \\
                          -\sqrt{3} & -1        &    \\
                                    &           & -4
                      \end{bmatrix}x_2 = 0 \\
    (3I - A_0)x_3 & = \begin{bmatrix}
                          1         & -\sqrt{3} &   \\
                          -\sqrt{3} & 3         &   \\
                                    &           & 0
                      \end{bmatrix}x_3 = 0  \\
\end{align*}
解得
\[
    x_1 = \begin{bmatrix}
        \sqrt{3} \\
        1        \\
        0
    \end{bmatrix},\quad x_2 = \begin{bmatrix}
        -1       \\
        \sqrt{3} \\
        0
    \end{bmatrix},\quad x_3 = \begin{bmatrix}
        0 \\
        0 \\
        1
    \end{bmatrix}
\]
因此
\[
    P = (x_1,x_2,x_3) = \begin{bmatrix}
        \sqrt{3} & -1       & 0 \\
        1        & \sqrt{3} & 0 \\
        0        & 0        & 1
    \end{bmatrix}\quad P^{-1} = \begin{bmatrix}
        \frac{\sqrt{3}}{4} & \frac{1}{4}        & 0 \\
        -\frac{1}{4}       & \frac{\sqrt{3}}{4} & 0 \\
        0                  & 0                  & 1
    \end{bmatrix}
\]
有
\begin{align*}
    E_1 & = (e_1,e_2,e_3)\begin{bmatrix}
                             \sqrt{3} \\
                             1        \\
                             0
                         \end{bmatrix}
    = \dfrac{2}{\sqrt{6}}\begin{bmatrix}
                             -2 & 1 \\
                             1  & 0
                         \end{bmatrix} \\
    E_2 & = (e_1,e_2,e_3)\begin{bmatrix}
                             -1       \\
                             \sqrt{3} \\
                             0
                         \end{bmatrix}
    = \sqrt{2}\begin{bmatrix}
                  0 & -1 \\
                  1 & 0
              \end{bmatrix}            \\
    E_3 & =(e_1,e_2,e_3)\begin{bmatrix}
                            0 \\
                            0 \\
                            1
                        \end{bmatrix}
    = \begin{bmatrix}
          0 & 0 \\
          0 & 1
      \end{bmatrix}
\end{align*}
且 $T(E_1,E_2,E_3) = (E_1,E_2,E_3)J$,
通过坐标变换得到
\[
    x = (E_1,\cdots,E_n)P^{-1}\begin{bmatrix}
        k_1    \\
        \vdots \\
        k_n
    \end{bmatrix} = (E_1,\cdots,E_n)\begin{bmatrix}
        l_1    \\
        \vdots \\
        l_n
    \end{bmatrix}
\]
\begin{align*}
    x & = \begin{bmatrix}
              4 & -4 \\
              0 & -3
          \end{bmatrix}
    = (e_1,e_2,e_3)\begin{bmatrix}
                       -4\sqrt{2} \\
                       0          \\
                       -3
                   \end{bmatrix}           \\
      & = (E_1,E_2,E_3)P^{-1}\begin{bmatrix}
                                 -4\sqrt{2} \\
                                 0          \\
                                 -3
                             \end{bmatrix}
    = (E_1,E_2,E_3)\begin{bmatrix}
                       -\sqrt{6} \\
                       \sqrt{2}  \\
                       -3
                   \end{bmatrix}
\end{align*}
于是
\begin{align*}
    (T^3)(x) & = (E_1,E_2,E_3)
    \begin{bmatrix}
        27 &    &    \\
           & -1 &    \\
           &    & 27
    \end{bmatrix}
    \begin{bmatrix}
        -\sqrt{6} \\
        \sqrt{2}  \\
        -3
    \end{bmatrix}
    = \begin{bmatrix}
          108 & -52 \\
          -56 & -81
      \end{bmatrix}           \\
    (T^k)(x) & = (E_1,E_2,E_3)
    \begin{bmatrix}
        3^k &        &     \\
            & (-1)^k &     \\
            &        & 3^k
    \end{bmatrix}
    \begin{bmatrix}
        \frac{\sqrt{3}}{4} & \frac{1}{4}        & 0 \\
        -\frac{1}{4}       & \frac{\sqrt{3}}{4} & 0 \\
        0                  & 0                  & 1
    \end{bmatrix}
    \begin{bmatrix}
        (x,e_1) \\
        (x,e_2) \\
        (x,e_3)
    \end{bmatrix}
\end{align*}
\par \ \par

\centerline{\large{\textbf{例题}}} \ \par

\paragraph*{例 1.36} 在欧式空间 $R^{2\times 2}$中,矩阵 $\bm{A}$和$\bm{B}$的内积定义为$(\bm{A},\bm{B}) = tr(\bm{A}^T\bm{B})$,子空间
\[
    V = \biggl\{\bm{X} = \begin{bmatrix}
        x_1 & x_2 \\
        x_3 & x_4
    \end{bmatrix}\ \bigg| \ x_3 - x_4 = 0\biggr\}
\]
$V$中的线性变换为
\[
    T(\bm{X}) = \bm{XB} + \bm{X}^T \quad (\forall \bm{X} \in V), \quad \bm{P} = \begin{bmatrix}
        1 & 2 \\
        1 & 1
    \end{bmatrix}
\]
\begin{enumerate}
    \item[(1)] 求$V$的一个标准正交基;
    \item[(2)] 验证$T$是 $V$中的对称变换;
    \item[(3)] 求$V$的一个标准正交基,使$T$在该基下的矩阵为对角矩阵.
\end{enumerate}

\paragraph*{解}
\begin{enumerate}
    \item[(1)]
        设$\bm{X} \in V$,则
        \[
            \bm{X} = \begin{bmatrix}
                x_1 & x_2 \\
                x_3 & x_3
            \end{bmatrix} = x_1\begin{bmatrix}
                1 & 0 \\
                0 & 0
            \end{bmatrix} + x_2\begin{bmatrix}
                0 & 1 \\
                0 & 0
            \end{bmatrix} + x_3\begin{bmatrix}
                0 & 0 \\
                1 & 1
            \end{bmatrix}
        \]
        故 $V$ 的一个标准正交基为
        \[
            \bm{X}_1 = \begin{bmatrix}
                1 & 0 \\
                0 & 0
            \end{bmatrix},\quad \bm{X}_2 = \begin{bmatrix}
                0 & 1 \\
                0 & 0
            \end{bmatrix},\quad \bm{X}_3 = \dfrac{1}{\sqrt{2}}\begin{bmatrix}
                0 & 0 \\
                1 & 1
            \end{bmatrix}
        \]
    \item[(2)]
        计算基象组:
        \begin{align*}
            T(\bm{X}_1) & =
            \begin{bmatrix}
                1 & 2 \\
                0 & 0
            \end{bmatrix} = 1\bm{X}_1 + 2\bm{X}_2 + 0\bm{X}_3 \\
            T(\bm{X}_2) & =
            \begin{bmatrix}
                2 & 1 \\
                0 & 0
            \end{bmatrix} = 2\bm{X}_1 + 1\bm{X}_2 + 0\bm{X}_3 \\
            T(\bm{X}_3) & = \dfrac{1}{\sqrt{2}}
            \begin{bmatrix}
                0 & 0 \\
                3 & 3
            \end{bmatrix} = 0\bm{X}_1 + 0\bm{X}_2 + 3\bm{X}_3
        \end{align*}
        设$T(\bm{X}_1,\bm{X}_2,\bm{X}_3) = (\bm{X}_1,\bm{X}_2,\bm{X}_3)\bm{A}$,则
        \[
            \bm{A} = \begin{bmatrix}
                1 & 2 & 0 \\
                2 & 1 & 0 \\
                0 & 0 & 3
            \end{bmatrix}
        \]
        $\bm{A}$为对称矩阵,因此$T$是对称变换.
    \item[(3)]
        求正交矩阵$\bm{Q}$使得$\bm{Q}^{-1}\bm{AQ} = \bm{A}$,解得
        \[
            \bm{A} = \begin{bmatrix}
                3 &   &    \\
                  & 3 &    \\
                  &   & -1
            \end{bmatrix} \quad \bm{Q} = \begin{bmatrix}
                0 & \frac{1}{\sqrt{2}} & -\frac{1}{\sqrt{2}} \\
                0 & \frac{1}{\sqrt{2}} & \frac{1}{\sqrt{2}}  \\
                1 & 0                  & 0
            \end{bmatrix}
        \]
        令$(\bm{Y}_1,\bm{Y}_2,\bm{Y}_3) = (\bm{X}_1,\bm{X}_2,\bm{X}_3)Q$,求得标准正交基
        \[
            \bm{Y}_1 = \dfrac{1}{\sqrt{2}}\begin{bmatrix}
                0 & 0 \\
                1 & 1
            \end{bmatrix},\quad \bm{Y}_2 = \dfrac{1}{\sqrt{2}}\begin{bmatrix}
                1 & 1 \\
                0 & 0
            \end{bmatrix},\quad \bm{Y}_3 = -\dfrac{1}{\sqrt{2}}\begin{bmatrix}
                -1 & 1 \\
                0  & 0
            \end{bmatrix}
        \]
        且有 $T(\bm{Y}_1,\bm{Y_2},\bm{Y_3}) = (\bm{Y}_1,\bm{Y_2},\bm{Y_3})\bm{A}$.
\end{enumerate}
\par \ \par

\centerline{\large{\textbf{习题}}} \ \par

\paragraph*{习题 1.3.15} 在欧式空间 $R^{2\times 2}$中,矩阵 $\bm{A}$和$\bm{B}$的内积定义为$(\bm{A},\bm{B}) = tr(\bm{A}^T\bm{B})$,子空间
\[
    V = \biggl\{\bm{X} = \begin{bmatrix}
        x_1 & x_2 \\
        x_3 & x_4
    \end{bmatrix}\ \bigg| \ \begin{matrix}
        x_1 - x_4 = 0 \\
        x_2 - x_3 = 0
    \end{matrix}\biggr\}
\]
$V$中的线性变换为
\[
    T(\bm{X}) = \bm{XP} + \bm{X}^T \quad (\forall \bm{X} \in V), \quad \bm{P} = \begin{bmatrix}
        0 & 1 \\
        1 & 0
    \end{bmatrix}
\]
\begin{enumerate}
    \item[(1)] 求$V$的一个标准正交基;
    \item[(2)] 验证$T$是 $V$中的对称变换;
    \item[(3)] 求$V$的一个标准正交基,使$T$在该基下的矩阵为对角矩阵.
\end{enumerate}

\paragraph*{解}
\begin{enumerate}
    \item[(1)]
        设 $\bm{X} \in V$,则
        \[
            \bm{X} = \begin{bmatrix}
                x_1 & x_2 \\
                x_2 & x_1
            \end{bmatrix} = x_1 \begin{bmatrix}
                1 & 0 \\
                0 & 1
            \end{bmatrix} + x_2 \begin{bmatrix}
                0 & 1 \\
                1 & 0
            \end{bmatrix}
        \]
        故$V$的一个标准正交基为
        \[
            \bm{X}_1 = \dfrac{1}{\sqrt{2}}\begin{bmatrix}
                1 & 0 \\
                0 & 1
            \end{bmatrix},\quad \bm{X}_2 = \dfrac{1}{\sqrt{2}}\begin{bmatrix}
                0 & 1 \\
                1 & 0
            \end{bmatrix}
        \]
    \item[(2)]
        计算基象组
        \begin{align*}
            T(X_1) & =
            \begin{bmatrix}
                1 & 1 \\
                1 & 1
            \end{bmatrix} = 1\bm{X}_1 + 1\bm{X}_2 \\
            T(X_2) & =
            \begin{bmatrix}
                1 & 1 \\
                1 & 1
            \end{bmatrix} = 1\bm{X}_1 + 1\bm{X}_2
        \end{align*}
        设 $T(\bm{X}_1,\bm{X}_2) = (\bm{X}_1,\bm{X}_2)\bm{A}$,则
        \[
            \bm{A} = \begin{bmatrix}
                1 & 1 \\
                1 & 1
            \end{bmatrix}
        \]
    \item[(3)]
        \begin{align*}
            \lambda I - A & = \begin{bmatrix}
                                  \lambda - 1 & -1         \\
                                  -1          & \lambda -1
                              \end{bmatrix} \\
                          & \to \begin{bmatrix}
                                    \lambda & -\lambda    \\
                                    -1      & \lambda - 1
                                \end{bmatrix}
            \to \begin{bmatrix}
                    \lambda & 0          \\
                    -1      & \lambda -2
                \end{bmatrix}                   \\
                          & \to \begin{bmatrix}
                                    \lambda & 0           \\
                                    0       & \lambda - 2
                                \end{bmatrix}
        \end{align*}
        所以
        \begin{align*}
            (-\bm{A})x_1     & = \begin{bmatrix}
                                     -1 & -1 \\
                                     -1 & -1
                                 \end{bmatrix}x_1 = 0 \\
            (2I - \bm{A})x_2 & = \begin{bmatrix}
                                     1  & -1 \\
                                     -1 & 1
                                 \end{bmatrix}x_2 = 0
        \end{align*}
        解得$x_1 = \dfrac{1}{\sqrt{2}}
            \begin{bmatrix}
                1 \\
                -1
            \end{bmatrix},\quad x_2 = \dfrac{1}{\sqrt{2}}
            \begin{bmatrix}
                1 \\
                1
            \end{bmatrix}$\\
        所以
        \[
            \varLambda =
            \begin{bmatrix}
                0 &   \\
                  & 2
            \end{bmatrix}\quad
            P = \dfrac{1}{\sqrt{2}}
            \begin{bmatrix}
                1  & 1 \\
                -1 & 1
            \end{bmatrix}
        \]
        于是有
        \begin{align*}
            \bm{Y}_1 & = \dfrac{1}{2}
            \begin{bmatrix}
                -1 & 1  \\
                1  & -1
            \end{bmatrix}            \\
            \bm{Y}_2 & = \dfrac{1}{2}
            \begin{bmatrix}
                1 & 1 \\
                1 & 1
            \end{bmatrix}
        \end{align*}

\end{enumerate}

\end{document}