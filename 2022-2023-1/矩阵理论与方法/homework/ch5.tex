\documentclass[12pt, a4paper, oneside, fontset=none]{ctexart}
\usepackage{amsmath, amsthm, amssymb, graphicx, color, fontspec, float, pgfplots}
\usepackage{bm}
\usepackage[bookmarks=true, colorlinks, citecolor=blue, linkcolor=black]{hyperref}
\pgfplotsset{compat=1.16}
\usepackage{xeCJK, CJKnumb}
\xeCJKsetup{CJKmath=true,CheckSingle=true}
\setCJKmainfont[ItalicFont=KaiTi]{微软雅黑}
\usepackage{geometry}
\geometry{left=1.8cm, right=1.8cm, top=2.18cm, bottom=2.18cm}
\author{}
\date{}
\linespread{1.25}
\title{\vspace{-3em}\textbf{矩阵论 \quad 第五次作业}\vspace{-3em}}

\begin{document}

\maketitle

\section*{第1章 \quad 线性空间和线性变换}

\subsection*{1.2 \quad 线性变换及其矩阵}

\centerline{\large{\textbf{定义}}} \ \par

\paragraph*{定义 1.19} 首项系数是$1$(简称首$1$),次数最小,且以矩阵$A$为根$\lambda$的多项式,称为$A$的\textbf{最
    小多项式,}常用$m(\lambda)$表示. \par
根据定理$1.18$,显然$A$的最小多项式$m(\lambda)$的次数不大于它的特征多项式$\varphi(\lambda)$的次数. \par \ \par

\centerline{\large{\textbf{定理}}} \ \par

\paragraph*{定理 1.17} 任意$n$阶矩阵与三角矩阵相似.

\paragraph*{定理 1.18(Hamilton-Cayley)} $n$阶矩阵$A$是其特征多项式的矩阵根(零点),即令
\[
    \varphi(\lambda) = \det(\lambda I - A) = \lambda^n + a_1\lambda^{n-1} + \cdots + a_{n-1}\lambda + a_n
\]
则有
\[
    \varphi(A) = A^n + a_1A^{n-1} + \cdots + a_{n-1}A + A_nI = O
\]

\paragraph*{定理 1.20} 矩阵$A$的最小多项式$m(\lambda)$与其特征多项式$\varphi(\lambda)$的零点相同(不计重数).
\paragraph*{证} 有定理$1.18$知$\varphi(A) = O$,再知$m(\lambda)$能够整除$\varphi(\lambda)$,所以$m(\lambda)$的零点是
$\varphi(\lambda)$的零点. \par
又设$\lambda_0$是$\varphi(\lambda)$的一个零点,也是$A$的一个特征值,那么有非零向量$\bm{x}\in C^n$,使得
\[
    A\bm{x} = \lambda_0 \bm{x} \qquad or \qquad m(A)\bm{x} = m(\lambda_0) \bm{x}
\]
因为$m(A) = O$,所以$m(\lambda_0)\bm{x} = 0$,从而$m(\lambda_0) = 0$,故$\varphi(\lambda)$的零点也是$m(\lambda)$的零点.

\paragraph*{定理 1.21} 设$n$阶矩阵$A$的特征多项式为$\varphi(\lambda)$,特征矩阵$\lambda I - A$的全体$n - 1$阶子式的最
大公因式为$d(\lambda)$,则$A$的最小多项式为
\[
    m(\lambda) = \dfrac{\varphi(\lambda)}{d(\lambda)}
\]
\par \ \par

\centerline{\large{\textbf{例题}}} \ \par

\paragraph*{例题 1.20} 计算矩阵多项式$A^{100} + 2A^{50}$,其中
\begin{equation*}
    A = \begin{bmatrix}
        1 & 1  & -1 \\
        1 & 1  & 1  \\
        0 & -1 & 2
    \end{bmatrix}
\end{equation*}

\paragraph*{解} 令$\phi(\lambda) = \lambda^{100} + 2\lambda^{50}$,可求得$A$的特征多项式为
\[
    \varphi(\lambda) = \det(\lambda I - A) = (\lambda - 1)^2(\lambda-2)
\]
用$\varphi(\lambda)$除$\phi(\lambda)$。可得
\[
    \phi(\lambda) = \varphi(\lambda)q(\lambda) + b_0 + b_1 \lambda + b_2\lambda^2
\]
将$\lambda = 1, 2$分别代入上式,则有
\begin{equation*}
    \begin{cases}
        b_0 + b_1 + b_2 = 3 \\
        b_0 + 2b_1 + 4b_2 = 2^{100} + 2^{51}
    \end{cases}
\end{equation*}
对$\phi(\lambda)$求导,得到
\[
    \phi^{'}(\lambda) = [2(\lambda - 1)(\lambda - 2) + (\lambda - 1)^2]q(\lambda) + \varphi(\lambda)q^{'}(\lambda) + b_1 + 2b_2 \lambda
\]
将$\lambda = 1$代入上式,可得
\[
    b_1 + 2b_2 = \varphi^{'}(1) = 200
\]
从而求得
\[
    \begin{cases}
        b_0 = 2^{100} + 2^{51} - 400 \\
        b_1 = 606 - 2^{101} - 2^{52} \\
        b_2 = -203 + 2^{100} + 2^{51}
    \end{cases}
\]
故
\[
    A^{100} + 2A^{50} = b_0 I + b_1 A + b_2 A^2
\]

\paragraph*{例题 1.21} 求矩阵
\begin{equation*}
    A = \begin{bmatrix}
        3  & -3 & 2  \\
        -1 & 5  & -2 \\
        -1 & 3  & 0
    \end{bmatrix}
\end{equation*}
的最小多项式.

\paragraph*{解} 设$f(\lambda) = \lambda + k\ (k \in R)$,由于$f(A) = A + kI \neq O$,所以任何一次多项式都不是$A$的最
小多项式,注意到$A$的特征多项式
\[
    \varphi(\lambda) = (\lambda - 2)^2(\lambda - 4)
\]
且对于它的二次因式
\[
    \phi(\lambda) = (\lambda - 2)(\lambda - 4) = \lambda^2 - 6 \lambda + 8
\]
有
\[
    \phi(A) = A^2 - 6A + 8I = O
\]
故,$m(\lambda) = \phi(\lambda).$
\par \ \par

\centerline{\large{\textbf{ppt例题}}} \ \par

\paragraph*{P5} 设$\bm{x}_1,\ \bm{x}_2$为线性空间$V$一组基,线性变换$T$在这组基下的矩阵为
\begin{equation*}
    A = \begin{bmatrix}
        2  & 1 \\
        -1 & 0
    \end{bmatrix}
\end{equation*}
$\bm{y}_1,\ \bm{y}_2$为$V$的另一组基,且
\begin{equation*}
    (\bm{y}_1,\ \bm{y}_2) = (\bm{x}_1,\ \bm{x_2}) \begin{bmatrix}
        1  & -1 \\
        -1 & 2
    \end{bmatrix}
\end{equation*}
\begin{enumerate}
    \item[(1)] 求$T$在$\bm{y}_1,\ \bm{y}_2$下的矩阵$B.$
    \item[(2)] 求$A^k.$
\end{enumerate}

\paragraph*{解}
\begin{enumerate}
    \paragraph*{解}
    \item[(1)] $T$在基$\bm{y}_1,\ \bm{y}_2$下的矩阵
        \begin{align*}
            B & = \begin{bmatrix}
                      1  & -1 \\
                      -1 & 2
                  \end{bmatrix}^{-1} \begin{bmatrix}
                                         2  & 1 \\
                                         -1 & 0
                                     \end{bmatrix} \begin{bmatrix}
                                                       1  & -1 \\
                                                       -1 & 2
                                                   \end{bmatrix}            \\
              & = \begin{bmatrix}
                      2 & 1 \\
                      1 & 1
                  \end{bmatrix} \begin{bmatrix}
                                    2  & 1 \\
                                    -1 & 0
                                \end{bmatrix} \begin{bmatrix}
                                                  1  & -1 \\
                                                  -1 & 2
                                              \end{bmatrix} = \begin{bmatrix}
                                                                  1 & 1 \\
                                                                  0 & 1
                                                              \end{bmatrix}
        \end{align*}
    \item[(2)] 由$B = C^{-1} A C$,有$A=CBC^{-1}$,于是$A^k=CB^kC^{-1}$
        \begin{align*}
            A^k & = \begin{bmatrix}
                        1  & -1 \\
                        -1 & 2
                    \end{bmatrix}\begin{bmatrix}
                                     1 & 1 \\
                                     0 & 1
                                 \end{bmatrix}^k \begin{bmatrix}
                                                     1  & -1 \\
                                                     -1 & 2
                                                 \end{bmatrix}^{-1}           \\
                & = \begin{bmatrix}
                        1  & -1 \\
                        -1 & 2
                    \end{bmatrix}\begin{bmatrix}
                                     1 & k \\
                                     0 & 1
                                 \end{bmatrix} \begin{bmatrix}
                                                   2 & 1 \\
                                                   1 & 1
                                               \end{bmatrix} = \begin{bmatrix}
                                                                   k + 1 & k      \\
                                                                   -k    & -k + 1
                                                               \end{bmatrix}
        \end{align*}
\end{enumerate}

\paragraph*{P17} 设
\begin{equation*}
    A = \begin{bmatrix}
        1 & 0  & 2 \\
        0 & -1 & 1 \\
        0 & 1  & 0
    \end{bmatrix}
\end{equation*}
求\ $2A^8 - 3A^5 + A^4 + A^2 -4I.$

\paragraph*{解} $A$的特征多项式\ $\varphi(\lambda) = |\lambda I - A| = \lambda^3 - 2\lambda + 1$ \\
用$\varphi(\lambda)$去除$2\lambda^8-3\lambda^5 + \lambda^4 + \lambda^2 - 4\lambda = g(\lambda)$,得
\[
    g(\lambda) = \varphi(\lambda)(2\lambda^5 + 4\lambda^3 - 5\lambda^2 + 9\lambda - 14) + (24\lambda^2 - 37\lambda + 10)
\]
\begin{flalign*}
    \because   \quad & \varphi(A) = 0                                     & \\
    \therefore \quad & 2A^8 - 3A^5 + A^4 + A^2 - 4I = 24 A^2 - 37 A + 10I & \\
                     & = \begin{bmatrix}
                             -3 & 48  & -26 \\
                             0  & 95  & -61 \\
                             0  & -61 & 34
                         \end{bmatrix}
\end{flalign*}

\paragraph*{P22} 求
\begin{equation*}
    A = \begin{bmatrix}
        1 & 1 & 0 \\
        0 & 1 & 0 \\
        0 & 0 & 1
    \end{bmatrix}
\end{equation*}
的最小多项式.

\paragraph*{解} $A$的特征多项式为
\[
    \varphi(\lambda) = |\lambda I - A| = \begin{vmatrix}
        \lambda - 1 & -1          & 0           \\
        0           & \lambda - 1 & 0           \\
        0           & 0           & \lambda - 1
    \end{vmatrix} = (\lambda - 1)^3
\]
又$A - I \neq 0$,
\begin{align*}
    (A - I)^2 & = A^2 - 2A + I                                   \\
              & = \begin{bmatrix}
                      1 & 2 & 0 \\
                      0 & 1 & 0 \\
                      0 & 0 & 1
                  \end{bmatrix} - \begin{bmatrix}
                                      2 & 2 & 0 \\
                                      0 & 2 & 0 \\
                                      0 & 0 & 2
                                  \end{bmatrix} + \begin{bmatrix}
                                                      1 & 0 & 0 \\
                                                      0 & 1 & 0 \\
                                                      0 & 0 & 1
                                                  \end{bmatrix}
\end{align*}
因此,$A$的最小多项式为$(\lambda - 1)^2.$

\paragraph*{P30} 试用初等变换化多项式矩阵
\begin{equation*}
    A(\lambda) = \begin{bmatrix}
        -\lambda + 1  & 2 \lambda - 1           & \lambda    \\
        \lambda       & \lambda^2               & -\lambda   \\
        \lambda^2 + 1 & \lambda^2 + \lambda - 1 & -\lambda^2
    \end{bmatrix}
\end{equation*}
为标准形.

\paragraph*{解}
\begin{align*}
    A(\lambda) & \xrightarrow[{[3]+[1]}]{} \begin{bmatrix}
                                               -\lambda + 1  & 2 \lambda - 1           & 1 \\
                                               \lambda       & \lambda^2               & 0 \\
                                               \lambda^2 + 1 & \lambda^2 + \lambda - 1 & 1
                                           \end{bmatrix} \xrightarrow[{[3]\leftrightarrow[1]}]{} \\
               & \begin{bmatrix}
                     1 & 2 \lambda - 1           & -\lambda+1    \\
                     0 & \lambda^2               & \lambda       \\
                     1 & \lambda^2 + \lambda - 1 & \lambda^2 + 1
                 \end{bmatrix} \xrightarrow[(3) - (1)]{}                                     \\
               & \begin{bmatrix}
                     1 & 2 \lambda - 1       & -\lambda+1          \\
                     0 & \lambda^2           & \lambda             \\
                     1 & \lambda^2 - \lambda & \lambda^2 + \lambda
                 \end{bmatrix} \xrightarrow[{[3] + (\lambda - 1)[1]}]{[2] - (2\lambda - 2)[1]}   \\
               & \begin{bmatrix}
                     1 & 0                   & 0                   \\
                     0 & \lambda^2           & \lambda             \\
                     0 & \lambda^2 - \lambda & \lambda^2 + \lambda
                 \end{bmatrix} \xrightarrow[{[2]\leftrightarrow[3]}]{}                           \\
               & \begin{bmatrix}
                     1 & 0                   & 0                   \\
                     0 & \lambda             & \lambda^2           \\
                     0 & \lambda^2 + \lambda & \lambda^2 + \lambda
                 \end{bmatrix} \xrightarrow[{[3]-\lambda[2]}]{}                                  \\
               & \begin{bmatrix}
                     1 & 0                   & 0                    \\
                     0 & \lambda             & 0                    \\
                     0 & \lambda^2 + \lambda & -\lambda^3 - \lambda
                 \end{bmatrix} \xrightarrow[{(3)-(\lambda+1)(2)}]{(-1)[3]}                       \\
               & \begin{bmatrix}
                     1 & 0       & 0                   \\
                     0 & \lambda & 0                   \\
                     0 & 0       & \lambda^3 + \lambda
                 \end{bmatrix}
\end{align*}
最后所得矩阵是$A(\lambda)$的标准形,此时,$d_1(\lambda) = 1$,$d_2(\lambda)=\lambda$,$d_3(\lambda) = \lambda^3 + \lambda.$

\paragraph*{P35} 求矩阵
\begin{equation*}
    A = \begin{bmatrix}
        -1 & 1 & 0 \\
        -4 & 3 & 0 \\
        1  & 0 & 2
    \end{bmatrix}
\end{equation*}
的最小多项式.

\paragraph*{解} 求$\lambda I - A$的初等因子组.由于
\begin{align*}
    \lambda I - A = & \begin{bmatrix}
                          \lambda + 1 & -1          & 0         \\
                          4           & \lambda - 3 & 0         \\
                          -1          & 0           & \lambda-2
                      \end{bmatrix} \to                      \\
                    & \begin{bmatrix}
                          -1          & 0                              & 0           \\
                          \lambda - 3 & (\lambda + 1)(\lambda - 3) + 4 & 0           \\
                          0           & -1                             & \lambda - 2
                      \end{bmatrix} \to \\
                    & \begin{bmatrix}
                          1 & 0               & 0           \\
                          0 & (\lambda - 1)^2 & 0           \\
                          0 & -1              & \lambda - 2
                      \end{bmatrix} \to                          \\
                    & \begin{bmatrix}
                          1 & 0               & 0                        \\
                          0 & (\lambda - 1)^2 & (\lambda-2)(\lambda-1)^2 \\
                          0 & -1              & 0
                      \end{bmatrix} \to             \\
                    & \begin{bmatrix}
                          1 & 0 & 0                         \\
                          0 & 1 & 0                         \\
                          0 & 0 & (\lambda -2)(\lambda-1)^2
                      \end{bmatrix}
\end{align*}
不变因子为$d_1(\lambda)=1$,$d_2(\lambda) = 2$,$d_3(\lambda) = (\lambda -2)(\lambda-1)^2.$\\
最后一个不变因子就是$A$的最小多项式:$m(\lambda) = d_3(\lambda)$\\
有$m(A) = 0.$

\paragraph*{P38} 求矩阵
\begin{equation*}
    A = \begin{bmatrix}
        1 & -1 & 2 \\
        3 & -3 & 6 \\
        2 & -2 & 4
    \end{bmatrix}
\end{equation*}
的最小多项式.

\paragraph*{解}
\begin{align*}
    \lambda I - A = & \begin{bmatrix}
                          \lambda - 1 & 1           & -2          \\
                          -3          & \lambda + 3 & -6          \\
                          -2          & 2           & \lambda - 4
                      \end{bmatrix} \to                 \\
                    & \begin{bmatrix}
                          \lambda - 1           & 1 & -2       \\
                          -\lambda^2 - 2\lambda & 0 & 2\lambda \\
                          -2\lambda             & 0 & \lambda
                      \end{bmatrix} \to  \begin{bmatrix}
                                             1 & -2       & \lambda-1             \\
                                             0 & 2\lambda & -\lambda^2 - 2\lambda \\
                                             0 & \lambda  & -2\lambda
                                         \end{bmatrix} \to \\
                    & \begin{bmatrix}
                          1 & 0        & 0                    \\
                          0 & 2\lambda & -\lambda^2 -2\lambda \\
                          0 & \lambda  & -2\lambda
                      \end{bmatrix} \to  \begin{bmatrix}
                                             1 & 0       & 0                     \\
                                             0 & 0       & -\lambda^2 - 2\lambda \\
                                             0 & \lambda & -2\lambda
                                         \end{bmatrix} \to  \\
                    & \begin{bmatrix}
                          1 & 0       & 0                     \\
                          0 & 0       & -\lambda^2 + 2\lambda \\
                          0 & \lambda & 0
                      \end{bmatrix} \to \begin{bmatrix}
                                            1 & 0       & 0                    \\
                                            0 & \lambda & 0                    \\
                                            0 & 0       & \lambda(\lambda - 2)
                                        \end{bmatrix}
\end{align*}
不变因子为$d_1(\lambda)=1$,$d_2(\lambda) = \lambda$,$d_3(\lambda) = \lambda(\lambda -2).$\\
最后一个不变因子就是$A$的最小多项式:$m(\lambda) = \lambda(\lambda -2)$\\
有$m(A) = 0.$
\par \ \par

\centerline{\large{\textbf{习题}}} \ \par

\paragraph*{1.2.14} 计算$2A^8 - 3A^5 + A^4 + A^2 - 4I$,其中
\begin{equation*}
    A = \begin{bmatrix}
        1 & 0  & 2 \\
        0 & -1 & 1 \\
        0 & 1  & 0
    \end{bmatrix}
\end{equation*}

\paragraph*{解}
\[
    \varphi(\lambda) = \det(\lambda I -A) = \lambda^3 - 2\lambda + 1
\]
有
\[
    A^3 -2A + I = 0
\]
那么
\begin{align*}
      & 2A^8 - 3A^5 + A^4 + A^2 - 4I                                      \\
    = & (A^3 -2A + I)(2A^5 + 4A^3 - 5A^2 + 9A -14 I) +(24A^2 - 37A + 10I) \\
    = & 24A^2 -37A + 10I                                                  \\
    = & \begin{bmatrix}
            -3 & 48  & -26 \\
            0  & 95  & -61 \\
            0  & -61 & 34
        \end{bmatrix}
\end{align*}

\end{document}