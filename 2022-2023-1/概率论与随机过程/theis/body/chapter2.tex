\section{问题的探究}

长期以来, “贝特朗问题”的上述三种“流行”解法像魔咒一样, 禁锢着人们的思
维, 尽管人们的争论“喋喋不休”, 论战“硝烟四起”, 结果“众说纷纭”, 有人认为三种
解法都对;论据是“等可能角度不同,概率的结果就不同”.有人说其中的一种解法正
确, 其它都是错误的;但理由却不能令人信服;有人说问题早已经解决, 又说不出是如何
解决的;林林总总, 不一而足.最具代表性的论据就是“等可能角度不同, 概率的结果就不同”.

\subsection{问题的分析}

下面,从几何概型的基本要素(样本空间的构造和均匀)对以上争议进行深入的分析.

\subsubsection{样本空间构造的合理性}

样本空间中的元素是试验的基本结果.贝特朗悖论的题意是要求在圆内任意作弦,
直观看来,试验的结果应该是做出的弦.但是几何概率问题中试验的基本结果应该用
“点” 来描述,这个点根据实际情况可以是一维数轴上的点,也可以是二维平面上的点,
或者是三维空间上的点.故求解贝特朗问题的首要任务是要把做出的弦转化成相应的
“点” ,即样本空间的元素.而弦和点之间应存在对应关系.

\par 另外,当对作弦附加了条件,比如指定弦的端点(方法1)和指定弦的方向(方法2),
会否缩小样本空间?

\par 方法1中,在圆周上任取一点,作为弦的一端,然后讨论弦的另一端点的位置.根据
圆的对称性,以及取点的任意性,固定一端点后所作的弦的性质与固定其他点所作的相
应弦的性质相同,当然包括概率这个性质.故虽然表面上看起来方法1的样本空间比解方
法2的样本空间要小,但所求的概率是合理的.

\par 方法3中,预先任意确定弦的方向,考虑在这个方向上的弦的性质.由于这个方向也
是任意的,在这个方向的弦的性质与其他方向上的弦的性质相同.因此虽然样本空间比
方法2的样本空间小,但所求的概率也是合理的. \cite{张敏2015贝特朗悖论之争的终结}

\subsubsection{样本空间的均匀性}

这意味着我们要选定一个区域,而在这个区域中,所有的样本都是均匀分布的. 而贝特朗悖论的题干中并没有明确的定义出这个”区域“到底是那一部分, 所以由于解题者对于使随机事件均匀分布的这部分”区域“的定义不同, 因此纠产生了分歧.

\par 方法一是假设弦的端点在这块”区域“上均匀分布.

\par 方法二是假定弦的中点在圆的直径上均匀分布.

\par 方法三是假定弦的中点在圆形内部均匀分布.

\par 这就是贝特朗悖论的成因, 在几何概型中所有随机事件都是需要在一块区域中均匀
分布的. 而在贝特朗悖论并没有没有明确的定义出让所有随机事件可以均匀分布的那
块”区域“, 导致解题者划定了不同的”区域“, 由此而产生了悖论.

\subsection{分析结果}

\subsubsection{贝特朗悖论并不奇}

所谓“悖论”一点也不悖. 这只是反映了选择不同的坐标会导致不同的概率分配这
一事实. 至于哪一个分配是“正确”的, 决定于事先确定的模型的如何应用或阐释.只是
有的人对任意作弦的方式有个人偏好,因此倾向于某种等可能性假设,而偏向于某种解
法.而实际上,这种假定甚至还不限于本文所提及的3种,所以贝特朗悖论的答案非但不
唯一, 甚至是无数个解. 当然,当等可能性条件补充完整后,贝特朗问题的解就唯一了.

\subsubsection{概率公理化}

贝特朗悖论说明在以往的定义中等可能性要求并不明确, 而要明确指出等可能的含
义, 又要因具体试验而定. 因此, 采用等可能性来定义一般的概率是困难的. 贝特朗悖论
的出现使得人们对什么是概率的疑惑放大到了极致. 人们明白必须要解决这个问题, 解
决问题的方法就是给出概率的严密的定义, 再在此基础上推演概率的理论体系, 即将概
率论公理化. 这极大地推动了概率论概念和方法的发展, 促使数学家们寻找另外的途径
来定义事件的概率, 从而最终建立了概率的公理化体系.