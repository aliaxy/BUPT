\section*{\zihao{2} \centering 摘要}

\vskip0.5cm
和所有的数学分支类似,概率论的也是经历了从直觉到严格的过程,其中的一个转
折点就是贝特朗悖论.在概率论的发展史上, 贝特朗悖论起了揭示问题促使人们思考概率
理论体系严密性的作用. 最后, 前苏联数学家柯尔莫哥洛夫建立了概率论的公理化体
系. 在概率论的公理化以及数学的发展, 悖论起到了重要的作用.


\textbf{关键词:}  贝特朗悖论,几何概率, 样本空间, 等可能假设, 概率的公理化
\addcontentsline{toc}{section}{摘要}

\section*{\zihao{2} \centering 介绍}

本主要对悖论及贝特朗悖论做出了阐释, 并列出其3种不同的解法. 其内涵是对“等可能”这一论述的不同解释, 从而引出概率公理化的必要性.

\addcontentsline{toc}{section}{介绍}