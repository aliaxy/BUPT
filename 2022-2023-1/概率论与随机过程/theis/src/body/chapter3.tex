\section{概率的公理化}

1900年, Hilbert在数学家大会上提出了23个20世纪应该解决的数学问题,建立
概率的公理化体系就在列其中. 希尔伯特建议用数学的公理化方法推演出全部物理, 首
先是概率和力学. 1993年, 前苏联数学家柯尔莫哥洛夫建立了概率论的公理化体系.\cite{冯变英2008贝特朗悖论与概率论的公理化}

\subsection{概率的公理化定义}

从贝特朗悖论我们看到, 概率的统一定义不能从统一其计算方法入手. 我们注意到,
虽然不同类型的随即问题中事件概率的计算公式是各式各样的, 但进一步分析可发现它
们有以下共同特点:
\par (1) 事件是样本空间的子集;
\par (2) 事件的概率$P(A)$是事件$A$的函数. 由于事件$A$是$\Omega$的子集, 因此$P(A)$是以$\Omega$中部分子集为变元的集合函数;
\par (3) 表达事件概率的集合函数$P(A)$均满足非负性、规范性、有限可加性或可列可加性.

\par 因此事件的概率的集合函数$P(A)$可用定义域为$\Omega$中的部分子集(事件), 函数值域
为$[0,1]$且满足非负性、规范性、可列可加性的集合函数来定义. 概率的这种定义被称
为\textbf{公理化定义}, 是前苏联科学家\textbf{柯尔莫哥洛夫}在1933年提出的, 即通过规定概率应具备
的基本性质(公理)来定义概率. 下面是其内容.

\paragraph*{概率的公理化定义}\cite{北京邮电大学理学院数学系概率教研室2010概率论与随机过程} 设$\Omega$是试验$E$的样本空间, 对于$E$的每一个事件$A$有一个实数与之对应, 记为$P(A)$, 且具有:

\par \textbf{性质1(非负性)} \ $0\leqslant P(A) \leqslant 1$.

\par \textbf{性质2(规范性)} \ $P(\Omega) = 1$.

\par \textbf{性质3(可列可加性)} \ 设$A_1,A_2,\cdots$是两两互不相容的事件, 即$A_iA_j = \varnothing(i \neq j,i,j=1,2,\cdots)$, 有
$$
    P(\bigcup_{i=1}^\infty A_i) = \sum\limits_{i=0}^n P(A_i).
$$
则称$P(A)$为事件$A$的概率.

\subsection{概率的可应用性}

柯尔莫哥洛夫的概率论公理化体系将概率论建立成了具有严谨演绎逻辑的一个纯
数学分支. 然而,公理化的概率论, 并没能解决一个非常重要的问题: 随机性的本质究
竟是什么? 早在拉普拉斯(1812)就讨论过明天太阳升起的概率, 爱因斯坦确信上帝是不
会掷骰子的, 海森堡(1927)提出测不准原理. 萨维齐秉承德·菲尼蒂的个人主观概率
观点, 在他的《统计学基础》(1954)中认为概率是一个普遍存在的概念, 人们天生就
会使用概率来管理其生活, 无须将概率与柯尔莫哥洛夫的数学公理体系联系起来, 只需确
定通用原则, 保持内在一致性规则. 柯尔莫哥洛夫在他提出公理化概率论后30年, 也
再次回到概率的可应用性问题研究, 在1965年提出了“公理化随机性”概念和“算法
复杂性”概念的两种解决思路.