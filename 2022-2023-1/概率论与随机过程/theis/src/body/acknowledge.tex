\section*{讨论}
\addcontentsline{toc}{section}{讨论}

公理化体系的建立, 离不开集合论和测度论的发展, 公理化体系的建立, 沟通了概率
论与其它数学分支的联系。其中, 悖论的出现对公理化体系的建立起了不可低估的作用.
不过遗憾的是, 对于贝特朗悖论的之争, 至今还在研究, 始终没有给定一个确定的结果.
\par 在数学的发展史上, 每一次悖论的出现, 都是数学理论体系漏洞的暴露, 每一次悖论
的出现, 都使得人们重新思考和完善数学基础. 数学史上三次大的数学危机都与悖论的
出现有关, 都使得数学前进了一大步. 希帕索斯悖论让当时统治数学界的“任何量都可
以用整数和整数的比来表示”化为泡影, 产生了无理数; 贝克莱悖论的出现, 使得极限概
念由模糊变得清晰, 由直觉变得严密;罗素悖论的出现, 使得建立在集合论基础上的数学
大厦摇摇欲坠,人们在不断的做出努力来构建和加固数学大厦, 数学大厦也就在人们的
不断加固和不断的扩建的过程中变得越来越光辉灿烂.