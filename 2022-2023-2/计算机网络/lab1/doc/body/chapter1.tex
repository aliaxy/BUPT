\pagenumbering{arabic}

\section{实验内容}

\subsection{实验内容}

\par 利用所学数据链路层原理,自己设计一个滑动窗口协议,在仿真环境下编程实现有噪音信道环境下
两站点之间无差错双工通信。信道模型为8000bps全双工卫星信道,信道传播时延270毫秒,信道误码率
为 $10^{-5}$,信道提供字节流传输服务,网络层分组长度固定为256字节。

\subsection{实验目的}

\par 通过该实验,进一步巩固和深刻理解数据链路层误码检测的CRC校验技术,以及滑动窗口的工作机
理。滑动窗口机制的两个主要目标:(1) 实现有噪音信道环境下的无差错传输;(2)充分利用传输信道的带
宽。在程序能够稳定运行并成功实现第一个目标之后,运行程序并检查在信道没有误码和存在误码两种
情况下的信道利用率。为实现第二个目标,提高滑动窗口协议信道利用率,需要根据信道实际情况合理
地为协议配置工作参数,包括滑动窗口的大小和重传定时器时限以及ACK搭载定时器的时限。这些参数
的设计,需要充分理解滑动窗口协议的工作原理并利用所学的理论知识,经过认真的推算,计算出最优
取值,并通过程序的运行进行验证。

\subsection{实验要求}

\par 实验题目给出了物理层信道模型和分组层数据的大小,链路层协议的设计有很大的自由度。由组内
同学共同讨论完成,包括帧控制字段的设计,滑动窗口的过程控制。从易到难,可选的协议类型为“不
搭载 ACK 的 Go-Back-N 协议“,“使用搭载 ACK 技术的 Go-Back-N 协议”,“选择重传协议”,要求必须
是全双工通信协议。本次实验中,选取了后两种协议进行实现。