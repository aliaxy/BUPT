\section{研究和探索的问题}

\subsection{CRC校验能力}

本次实验使用的 32 位的 CRC 校验码. 在理论上, 可以检测出: 所有的奇数个错
误、所有双比特错误、所有小于等于 32 位的突发错误。但是检测不错大于 32 位的突发错误.
因此如果出现 CRC32 不能检测出的错误, 至少需要出现 33 位突发错误。
而这一概率为 $5.160 \times 10^{-93}$ 可以认为这一事件为不可能事件, 客户每天实际可以发
送的帧数目为 $94.6\% * 8000 / 256 / 8 * 60 * 60 * 24 / 2 = 159638$ 帧.
所以发生这一事件至少需要 $ 1.937 \times 10^{92} / 159638 / 365 = 5.3091 \times 10^{91}$ 年.


\subsection{软件测试方面的问题}

\par 不同的测试方案可以检测软件在不同状况下的稳定性与性能, 以模拟真实环境.
\par 无误码信道数据传输可以用于验证程序基础功能的正确性, 这样可以先摆脱校验模块进行测试, 提高开发效率.
\par 平缓方式和洪水式产生分组则是模拟真实网络环境中的空闲期与高峰期.
\par 高误码率情况则可检测程序健壮性, 就网络而言, 一般来说人们愿意牺牲一定的性能来换来更高的稳定性.
\par 此外, 认为测试还应该覆盖信道完全空闲, 无数据传输时的情况, 以检测程序是否表现正常.


\subsection{对等协议实体之间的流量控制}

\par 鉴于选择重传协议有发送方窗口和接收方窗口限制, 当发送方流量过大时接收方窗
口将拒绝接收数据进而使发送方计时器超时而选择重传, 同时发送方受发送方窗口大小
限制不能突发发送大量数据. 因此可以认为对等协议实体之间有流量控制.