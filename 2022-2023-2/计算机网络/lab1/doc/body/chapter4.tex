\section{实验结果}

\subsection{简单描述}

我所实现的协议软件实现了有误码信道中无差错传输功能

\subsection{健壮性}

\par 在文档中给出的所有性能测试命令下, 程序均可正常运行超过 20 分钟, 且性能接近参
考数据, 这证明了程序的正确性与健壮性, 实现了有误码信道环境中无差错传输功能.

\subsection{协议参数的选取}

\par 考虑滑动窗口的大小, 重传定时器的时限, ACK搭载定时器的时限.
根据信道特性数据, 分组层分组的大小, 以及的滑动窗口机制, 给出定量分析, 详细列举
出选择这些参数值的具体原因.

\par 信道带宽为 8000 bps, 单程传播时延为 270ms, 使用稍带确认

\begin{align*}
  u = \dfrac{W}{2(1 + \alpha)}        \\
  \alpha = \dfrac{t_{prop}}{t_{tran}} \\
  p_{tran} = \dfrac{packet\_len}{transmission_rate}
\end{align*}

解得, $u = W / 4.1 \geqslant 100\% \Rightarrow W \geqslant 5$.

\par 经过测试, 选取ACK计时器为5100ms, 重传计时器为6000ms.(选择重传)
\subsection{理论分析}

\subsubsection{无误码情况下最大信道利用率}

\par 在数据帧中, 每一帧263字节, 帧头3个字节, 帧尾4个字节, 所以信道利用率

\begin{align*}
  \dfrac{256}{263} \times 100 \% \approx 97.3\%
\end{align*}

\subsubsection{有误码情况下最大信道利用率}

\par 这里假设重传操作及时, 重传的数据帧的回馈, ACK 帧可以100\%正确传输.
\par 当误码率为$10^{-5}$, 即发送$10^5$个比特, 出现一个错误, 则平均发送଴$\dfrac{100\ 000}{263\times 8} = 47.53$
个帧会出现一个错误. 先假设重传的帧必然正确, 则发送 47.5 帧需要发送 47.5 + 1 + 1 = 49.5 帧(NAK帧和重发的帧).
\par 此时信道利用率为:
\[
  47.5 / 49.5 * 97.3 \% = 95.3\%
\]
\par 当误码率为$10^{-4}$, 即发送$10^4$个比特, 出现一个错误, 则信号利用率为:
\[
  \dfrac{5}{6} \times 97.3 \% = 81.0 \%
\]

\subsection{结果分析}

\begin{figure}[h]
  \centering
  \includegraphics[width=1\textwidth]{figure/data.png}
  \caption*{数据结果记录表}
  \label{fig:data}
\end{figure}

\par 总体来说, 结果和理论值(参考数据)很接近, 效果很好.

\subsection{存在的问题}

\par 程序虽然经过了所有测试方案, 但在低误码率的情况下理论值与实际数据均有改进空间. 另一个是参数可能不是最优情况, 只是得到了一个较好的结果, 并没有继续向后测试.

